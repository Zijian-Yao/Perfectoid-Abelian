\documentclass[10pt,oneside]{amsart}
\usepackage{amsmath}
\usepackage{amsthm}
\usepackage{amsfonts}
\usepackage{amssymb,amscd,epsf,verbatim}
\usepackage{mathrsfs}
\usepackage{graphicx}
\usepackage{latexsym}
\usepackage{standalone}
\usepackage{lscape}
\usepackage[colorlinks=true]{hyperref}
\hypersetup{colorlinks, citecolor=blue, filecolor=black, linkcolor=red, urlcolor=green}
\usepackage{epstopdf}
\usepackage{tikz}
\usetikzlibrary{calc}
\usetikzlibrary{matrix,arrows,decorations.pathmorphing}
\usepackage{tikz-cd}
\usepackage{color}
\usepackage{geometry}
\usepackage{multirow}
\usepackage{enumitem}
\usepackage{framed}

\newcommand{\dstyle}{\displaystyle}

%%%%%%% please do NOT add any new command 
%%%%%%(unless it is absolutely necessary, in which case please send everyone an email about it.
%%%%%%%

%\theoremstyle{theorem}
\newtheorem{theorem}{Theorem}[section]
\newtheorem{lemma}[theorem]{Lemma}
\newtheorem{step}[theorem]{step}
\newtheorem{proposition}[theorem]{Proposition}
\newtheorem{corollary}[theorem]{Corollary}
\newtheorem{claim}[theorem]{Claim}
\newtheorem{conjecture}[theorem]{Conjecture}
\newtheorem*{outline}{Outline of Proof}
\newtheorem{mainthm}{Theorem} 
\newtheorem{lemma*}[mainthm]{Lemma}
\newtheorem{proposition*}[mainthm]{Proposition}



\theoremstyle{definition}
\newtheorem{definition}[theorem]{Definition}
\newtheorem{notation}[theorem]{Notation}
\newtheorem{question}[mainthm]{Question}
\newtheorem{remark}[theorem]{Remark}
\newtheorem*{example}{Example}
\newtheorem{remark*}[mainthm]{Remark}
\newtheorem{definition*}[mainthm]{Definition}




\title[perfectoid covers of abelian varieties]{perfectoid  covers of abelian varieties} 
%\date{August 2017}
\author{
	Clifford Blakestad \and
	Dami\'an Gvirtz \and
	Ben Heuer \and 
	Daria Shchedrina \and
	Koji Shimizu \and 
	Peter Wear \and
	Zijian Yao}

\begin{document}
	
	\maketitle
	
	\begin{abstract}
For an abelian variety $A$ over an algebraically closed non-archimedean field which is either $p$-adic or of characteristic $p$, we show that there exists a perfectoid space which is the tilde-limit of $\varprojlim_{[p]}A$.
	\end{abstract}
	

	
	
%	\tableofcontents
 	
	

%%%%%%%%%%%%
%%      Section 1
%%%%%%%%%%%%
	\section{Introduction} 

Let $p$ be a prime and $K$ a non-archimedean field which is either $p$-adic or characteristic $p$. For simplicity we assume that $K$ is  algebraically closed in this introduction. The primary goal of this article is to show that the ``inverse limit'' of an abelian variety $A$ over $K$ under the multiplication by $p$ map carries a natural structure of perfectoid spaces. %Before we state the precise version of the theorem, we remark that the main assertion already implicitly appeared in [?] [?] without justification. Nevertheless we decide to fill in this gap in the literature, since the proof, despite being straight-forward, is quite subtle to spell out. 
More precisely, we prove 

         \begin{mainthm} \label{thm:main_thm_intro}
		Let $A$ be an abelian variety over $K$. Then there exists a unique perfectoid space $A_\infty$ over $K$ such that
				$A_\infty \sim \varprojlim_{[p]} A.$
	\end{mainthm}
	
\begin{remark*}	 \label{remark:inverse_limit_adic}
We will recall the definition of the tilde-limit, namely the meaning of $A_\infty \sim \varprojlim_{[p]} A$, in Section \ref{section:tilde_limit} of the paper. This notion appears in \cite{SW} to remedy the problem that inverse limits rarely exist in the category of adic spaces in the sense of \textit{\small ibid} (even if transition maps are affine). 
\end{remark*}

\begin{remark*}	 
{\color{red}  char p case }
\end{remark*}

 To prove the main theorem,  we need to 
\begin{enumerate}
\item construct a tilde-limit $A_\infty$ for the system $ \cdots \xrightarrow{[p]} A \xrightarrow{[p]} A \xrightarrow{[p]} A $;
\item show that the adic space $A_\infty$ is perfectoid. 
\end{enumerate}
In fact, we will achieve (1) and (2) simultaneously. 

\begin{remark*}	\label{remark:good_reduction}
Let us consider the motivating example of the case of good reduction. This example is generalized from Exercise 4 -- 6 in \cite{Bhatt} and spelt out in details in Subsection \ref{subsection:perfectoid_tilde_limit} [Corollary \ref{tilde-limit exists and is perfectoid in the good reduction case}]. 

Let $\mathcal A$ be a formal abelian scheme over $\mathcal O_K$ with generic fiber $A$. Then on the mod $\pi$ special fiber $\mathcal A_s = \mathcal A \times \mathcal O_K/\pi$,  multiplication by $p$ factors through the relative Frobenius map, forcing the inverse limit $ \varprojlim_{[p]} \mathcal A_s $ (which exists in the category of schemes) to be relatively perfect over $\mathcal O_K/\pi$. Let $\mathcal A_{\infty} = \varprojlim_{[p]} \mathcal A$ (which exists by Lemma \ref{lemma:inverse_limit_formal}), then its adic generic fiber $A_\infty$ is perfectoid and a tilde-limit of $\varprojlim_{[p]}A$.
\end{remark*}

Now we explain the outline of the proof of Theorem \ref{thm:main_thm_intro}. Our approach relies on constructing various formal models of relevant objects and take their (adic) generic fibres. 

{\color{red} Explain outline}

\begin{remark*}	
{\color{red} Remark on Neron models -- why we cannot directly get the results using Neron models and also the problems of existence when $A$ is over $K = \overline K$ etc. }
\end{remark*}

\vspace*{0.2cm}


 
Now we describe the content of each section. 



 
	\begin{question} \label{question_intro}
	    \begin{enumerate} 
	    \item		Given a rigid group $G$, when is there an adic space $G_\infty$ such that $G_\infty \sim  \varprojlim_{[p]} G ?$
	    \item If it exists, and $K$ is perfectoid, when is $G_\infty$ perfectoid?
	    \end{enumerate}
	\end{question}
 
 
{\color{red} To finish}
 
\begin{comment} 
	
	
	
	But before we give proofs for examples of rigid groups $G$ for which a perfectoid tilde-limit exists, we first note that the second question certainly doesn't have an affirmative answer for all rigid group varieties:
	\begin{example}
		For the additive group $\mathbb G_a^{\operatorname{an}}$, we know that $[p]$ is an isomorphism and therefore $\varprojlim_{[p]} \mathbb G_a=\mathbb G_a$ exists (even as an actual limit in the category of adic spaces) but is certainly not perfectoid.
	\end{example}


We first more generally consider a rigid group  $G$ over a non-archimedean field $K$. While inverse limits usually don't exist in the rigid analytic category, limits are much better behaved in formal schemes over the ring of integers $\mathcal O_K$ of $K$. One can therefore give a simple criterion in terms of formal models that guarantees that a tilde-limit $G_\infty \sim\varprojlim_{[p]} G$ exists, namely that there is a well-behaved formal model of the $[p]$-multiplication tower.
If $K$ is perfectoid, we give a stronger criterion involving a Frobenius factorisation condition, which implies that $G_\infty$ is perfectoid.

In the case of a rigid analytic split torus $T$, one can use a family of explicit covers by affinoids to construct formal models for which both of these conditions are satisfied. 

Next we consider the case of the Raynaud extension $E$ associated to a semistable abelian variety $A$ over a perfectoid field $K$. One can construct $E$ by extending the rigid fibre of a formal group scheme $\overline{E}$ by a rigid torus $T$. In order to construct a formal model of $E$ one therefore just needs to extend $\overline{E}$ by a formal model of $T$. While this can be done explicitly using affinoid covers, the language of formal and rigid fibre bundles allows for a more conceptual treatment. Using the associated fibre construction we then show that there is a formal model of the $[p]$-multiplication tower of $E$ which satisfies all the necessary criteria to show that $E_\infty$ exists and is perfectoid.  

We then construct a tilde-limit of $\varprojlim_{[p]} A$ from $E_\infty$: By Raynaud uniformisation, $A$ is naturally isomorphic to the rigid analytic quotient of $E$ by a lattice $M$. After a choice of $\Gamma_0(p^\infty)$-structure, the $[p]$-multiplication tower of $E/M$ factors in a ``ramified'' and an ``\'etale'' part. By a careful choice of charts of $E/M$ in terms of subspaces of $E$ that behave well under $[p^n]:E\rightarrow E$, one can explicitly construct first a perfectoid tilde-limit of the ``ramified'' tower, and then in a second step the space $A_\infty$. This space is independent of the choice of $\Gamma_0(p^\infty)$-structure but remembers it as a pro-\'etale subgroup $D_\infty \subseteq A_\infty$. The construction shows that the perfectoid tilde-limit $A_\infty$ still exists under the weaker assumption that $K$ is a perfectoid field over which there exist lattices $M^{1/p^n}$ for all $n$ whose $p^n$-th multiple is $M$.

The approach via explicit covers finally gives an explicit description of $A_\infty$ in terms of open subspaces of $E_\infty$ which we use in the last section to study the induced map $E_\infty\rightarrow A_\infty$. We show that $E_\infty$ is in fact an open subspace of $A_\infty$ and thus obtain a second description of $A_\infty$ as a quotient group $(D_\infty\times E_\infty)/M_\infty$.


\newpage

\end{comment}




\addtocontents{toc}{\protect\setcounter{tocdepth}{0}} %some hack to hide the acknowledgements in the toc
\section*{Notation}
\addtocontents{toc}{\protect\setcounter{tocdepth}{2}} % end hack
	We fix the prime $p$.  Let $K$ be a perfectoid field (either $p$-adic or of characteristic $p$), with the ring of integers  $\mathcal O_K$  and  a fixed pseudo-uniformiser $\pi\in \mathcal O_K$. 
	
	By adic spaces over $\operatorname{Spa}(K,\mathcal O_K)$, we mean adic spaces in the sense of \cite{SW}, and we adopt the notion of perfectoid spaces defined in \S2 \textit{\small ibid}. In their language, adic spaces in the sense of Huber are referred to as \textit{honest} adic spaces. Throughout the article, we make no distinction between rigid spaces (resp. formal schemes) and their corresponding honest adic spaces (resp. adic spaces). 
	
	If a rigid space is obtained from a $K$-scheme $X$ via rigid-analytification $X\mapsto X^{\operatorname{an}}$, we will often denote both by the same symbol $X$.
		%?In particular, any cover of a rigid space will be a cover as an adic space, and therefore be an admissible cover in the sense of rigid analytic geometry.  		{\color{red} I don't understand what this remark means. }
 
	For a formal scheme $\mathfrak X$ over $\operatorname{Spf}(\mathcal O_K)$ with the $\pi$-adic topology, we denote by $\mathfrak X_\eta$ its adic generic fibre. 
	We denote by $\tilde{X}=\mathfrak X\times_{\operatorname{Spf}\mathcal O_K}\operatorname{Spf}\mathcal O_K/\pi$ its mod $\pi$ special fibre, considered as a scheme over $\operatorname{Spec}\mathcal O_K/\pi$. 

	Finally, recall the following standard terminology: 
		\begin{enumerate}
			\item Let $X$ be a rigid space over $K$. A \textbf{formal model} of $X$ is an admissible topologically finite type formal scheme $\mathfrak X$ over $\mathcal O_K$ together with an isomorphism of its generic fibre $\mathfrak X_\eta \xrightarrow{\sim} X$ (which is often suppressed from notation).
			\item Let $\phi:  X\rightarrow  Y$ be a morphism of rigid spaces over $K$, with formal models $\mathfrak X$ and $\mathfrak Y$	respectively. A morphism of formal schemes $\Phi:\mathfrak X \rightarrow \mathfrak Y$ is a \textbf{formal model} of $\phi$ if it agrees with $\phi$ on the adic generic fiber. 
		\end{enumerate}


%%%%%%%%%%%%
%%      Section 2
%%%%%%%%%%%%	
	
\numberwithin{theorem}{section}
	\section{Tilde-limits of rigid groups} \label{section:tilde_limit}
  
	

		\subsection{Tilde-limits and formal models} 
		
	As we have seen in Remark \ref{remark:inverse_limit_adic} in the introduction, inverse limits often do not exist in the category of adic spaces (and neither in in rigid spaces). Instead we use the notion of tilde-limit from \cite{SW}:
	
	\begin{definition} 
Let $X_i$ be a filtered inverse system of adic spaces with quasi-compact and quasi-separated transition maps, let $X$ be an adic space with a compatible system of morphisms $f_i: X \rightarrow X_i$. We write $X \sim \varprojlim X_i$ ($X$ is a \textbf{tilde-limit} of $X_i$) if the map of underlying topological spaces $|X| \rightarrow \varprojlim |X_i|$ is a homeomorphism, and there exists an open cover of $X$ by affinoids $\operatorname{Spa} (A, A^+) \subset X$ such that the map 
$$ \varinjlim_{\operatorname{Spa}(A_i, A_i^+) \subset X_i} A_i \rightarrow A$$
has dense image, where the direct limit runs over all open affinoids $\operatorname{Spa}(A_i, A_i^+) \subset X_i$ containing the image of $\operatorname{Spa}(A, A^+) \subset X$.
	\end{definition}
	
	\begin{remark} \label{remark:tilde_limit_non_unique}
As pointed out after Proposition 2.4.4.\ of \cite{SW}, tilde-limits (if they exist) are in general not unique. For example, consider the inverse system consisting of a single affinoid pre-perfectoid space $X = \operatorname{Spa}(A, A^+)$, then its strong completion $\hat X = \operatorname{Spa}(\hat A,\hat A^+) $ is also a tilde-limit of $X$. 
	\end{remark}
	
One way to construct tilde-limits is by constructing certain formal models. First we observe that inverse limits are much better behaved in the category of formal schemes. 

	\begin{lemma} \label{lemma:inverse_limit_formal}
		Let $(\mathfrak X_i,\phi_{ij})_{i\in I}$ be an inverse system of formal schemes $\mathfrak X_i$ over $\mathcal O_K$ with affine transition maps $\phi_{ij}:\mathfrak X_j\rightarrow \mathfrak X_i$. Then the inverse limit $\mathfrak X=\varprojlim \mathfrak X_i$ exists in the category of formal schemes over $\mathcal O_K$. If all the $\mathfrak X_i$ are flat formal schemes, so is $\mathfrak X$.
	\end{lemma}
	\begin{proof}
	In the affine case, if the inverse system is $\operatorname{Spf} A_i$, take $A$ to be the $\pi$-adic completion of $\varinjlim A_i$, then  $\operatorname{Spf} A$ is the inverse limit of the $\operatorname{Spf}A_i$. In general, we can use the fact that the transition maps are affine to reduce to the affine case. 
	\end{proof}
	
	In the lemma above, $\mathfrak X$ is also a tilde-limit  $\mathfrak X\sim \varprojlim \mathfrak X_i$. This remains true after passing to the adic generic fibre after base-change $\operatorname{Spa}(K,\mathcal O_K)\hookrightarrow \operatorname{Spa}(\mathcal O_K,\mathcal O_K)$.
	
	\begin{lemma}\label{tilde-limit from adic generic fibre of formal schemes}
		Let $(\mathfrak X_i,\phi_{ij})_{i\in I}$ be an inverse system of formal schemes $\mathfrak X_i$ over $\mathcal O_K$ with affine transition maps $\phi_{ij}$ and let $\mathfrak X=\varprojlim_{\phi_j} \mathfrak X_i$ be the limit. Let $\mathcal X_i =(\mathfrak X_i)_\eta$ and  $\mathcal X = (\mathfrak X)_\eta$ be the adic generic fibres. Then
		\[\mathcal X \sim \varprojlim \mathcal X_i.\]
	\end{lemma}
	\begin{proof}
		This is a consequence of \cite{SW}, Proposition 2.4.2: The transition maps in the system are affine, hence quasi-separated quasi-compact. In order to prove the Lemma, we can restrict to an affine open subset $\operatorname{Spf}(A)$ of $\mathfrak X$ that arises as the inverse limit of affine open subsets $\operatorname{Spf}(A_i)\subseteq \mathfrak X_i$. Here all formal schemes are considered with the $\pi$-adic topology and $A$ is the $\pi$-adic completion of $\varinjlim A_i$. 
		On the generic fibre, $A_i$ with ideal of definition $I_i=\pi A_i$ is an open subring of definition of $A_i[1/\pi]$. We then clearly have $I_iA_j = A_j$ for any $j\geq i$. The inverse system therefore satisfies the conditions of \cite{SW}, Proposition 2.4.2, and we conclude that $\operatorname{Spf}(A)_\eta \sim \varprojlim \operatorname{Spf}(A_i)_\eta$ as desired.
	\end{proof}
	
	\begin{remark}
	This lemma essentially says that one may construct a tilde-limit of an inverse system of rigid spaces $\mathcal X_i$ if it arises from an inverse system of formal schemes $\mathfrak X_i$ with affine transition maps. This is precisely what Scholze uses in \cite{torsion} to construct the space $\mathcal X_{\Gamma_0(p^\infty)}(\epsilon)_a$ (see Corollary III.2.19 in \cite{torsion} and its proof).
	\end{remark}
	
	%Therefore, one way to construct  a tilde-limit $\varprojlim \mathcal X_i$ for an inverse system of rigid spaces $\mathcal X_i$ is to look for formal models $\mathfrak X_i$ of the system with affine transition maps. If such data exist, Lemma~\ref{tilde-limit from adic generic fibre of formal schemes} produces a tilde-limit $\mathcal X \sim \varprojlim \mathcal X_i$. 
	
 	\begin{remark} \label{Raynaud theory main theorem}
			
	Let us recall Raynaud's theory of formal models: under mild assumptions,
	one can always find formal models of rigid spaces, and (possibly after formal blow ups) of morphisms between them. More precisely, Raynaud's theorem \cite[section 8.4]{Bosch lectures} states that
		
		\begin{enumerate}
			\item Let $X$ be a quasi-separated quasi-paracompact rigid space over $K$. Then there exist an admissible quasi-paracompact formal model $\mathfrak X$ for $X$.
			\item If $\mathfrak X'\rightarrow \mathfrak X$ is an admissible blow-up of admissible formal schemes, then it induces an isomorphism on the generic fibre  $\mathfrak X'_\eta \xrightarrow{\sim} \mathfrak X_\eta$.
			\item Let $\mathfrak X$ and $\mathfrak Y$ be admissible quasi-paracompact formal schemes over $\mathcal O_K$ and let $f:\mathfrak X_\eta \rightarrow \mathfrak Y_\eta$ be a morphism of their associated rigid spaces. Then there exist an admissible blow-up $\pi:\mathfrak X'\rightarrow \mathfrak X$ and a map $\mathfrak f:\mathfrak X'\rightarrow \mathfrak Y$ such that $\mathfrak f_\eta = f\circ \pi_\eta$.
		\end{enumerate}
		In particular, given an inverse system ($\mathcal X_i,\phi_{ij})$ of rigid spaces, one can always choose formal models $\mathfrak X_i$ for the $\mathcal X_i$, and by successive admissible blow-ups one can also find models for the transition maps $\phi_{ij}$ (\textit{which might not be affine morphisms}). 
		\end{remark}

	
\subsection{Tilde-limits for rigid groups}

Let $G$ be a rigid group over $K$, that is, a group object in the category of rigid spaces over $K$.
	We are primarily interested in the following examples 
\begin{enumerate}	 
\item Analytifications of finite type group schemes over $K$. Examples include the analytification of an abelian variety $A$ over $K$, of tori $T$ over $K$, \textit{\small etc.}
\item Generic fibres of topologically finite type formal group schemes over $\mathcal O_K$. 
\item Raynaud's covering space $E$  of an abelian variety with semi-stable reduction.
\end{enumerate}

	Lemma \ref{lemma:inverse_limit_formal} and  \ref{tilde-limit from adic generic fibre of formal schemes} motivate the following definition. 
	\begin{definition}  \indent 
	
	\begin{enumerate}
	\item	Let $G$ be a rigid group. A \textbf{$[p]$-formal tower} for $G$ is  a family of formal models $\{\mathfrak G_n\}_{n\in \mathbb N}$ of $G$, together with affine transition maps $[\mathfrak p]_{n+1}:\mathfrak G_{n+1}\rightarrow \mathfrak G_{n}$ which are formal models of $[p]:G\rightarrow G$. Sometimes we suppress notation and write $[\mathfrak p]$ for $[\mathfrak p]_{n}$ on $\mathfrak G_n$. 
	\item	 More generally, let $U\subseteq G$ be an admissible open subset, a $[p]$-formal tower for $U \subseteq G$ is a tower of formal models for 
$$ \cdots \xrightarrow{[p]} [p^2]^{-1} (U)   \xrightarrow{[p]} [p]^{-1} (U)  \xrightarrow{[p]} U. $$  
	\end{enumerate}
	\end{definition}
	

In particular, a $[p]$-formal tower for $G$ gives rise to a $[p]$-formal tower for $U$ for each admissible open $U \subseteq G$ by pullback. 

\begin{comment}	\begin{center}
		\begin{tikzcd}
			\dots \arrow[r] & \left [p^2\right]^{-1}(U) \arrow[r, "{[p]}"] \arrow[d, hook'] & \left [p\right]^{-1}(U) \arrow[r, "{[p]}"] \arrow[d, hook'] & U \arrow[d, hook'] \\
			\dots \arrow[r, "{[p]}"] & G \arrow[r, "{[p]}"] & G \arrow[r, "{[p]}"] & G
		\end{tikzcd}
	\end{center}  \end{comment}
	
	We now summarise our discussion in the previous subsection by the following
	\begin{corollary}
		Let $G$ be a rigid group. If $G$ admits a $[p]$-formal tower, then there exists a rigid space $G_\infty$ such that $G_\infty \sim \varprojlim_{[p]}G$.
	\end{corollary}
	For example, for any formal group $\mathfrak G$ where $[p]$ is affine, the $[p]$-multiplication tower of $\mathfrak G$ gives rise to a $[p]$-formal tower for its generic fibre $\mathfrak G_\eta$. 
	
	
	
	
	\subsection{Perfectoidness}  \label{subsection:perfectoid_tilde_limit}

In this subsection we define a formalism to address part (2) of Question \ref{question_intro} in the introduction. In light of Remark \ref{remark:good_reduction}, we first make the following definition
	
	\begin{definition}
		Let $G$ be a rigid analytic group, a \textbf{$[p]$-$F$-formal tower} for $G$ is a $[p]$-formal tower 
		$$(\{\mathfrak G_n\}_{n\in \mathbb N}, [\mathfrak p]_{n+1}:\mathfrak G_{n+1}\rightarrow \mathfrak G_{n})$$ such that each $\mathfrak G_n$ is flat over $\mathcal O_K$, and on the mod $\pi$ special fibre  $\tilde{G}_n$, each   $[\mathfrak p]_{n+1}$ factors through the relative Frobenius:
				\begin{center}
					\begin{tikzcd}
						& \tilde G_{n+1}^{(p)} \arrow[rd, dashed] &  \\
						\tilde G_{n+1} \arrow[rr, "{[p]}"] \arrow[ru, "F_{rel}"] &  & \tilde G_{n}
					\end{tikzcd}
				\end{center}
	 
	\end{definition}	
	
	 	\begin{proposition}\label{existence of p-F-formal tower implies perfectoid}
		Let $G$ be a rigid analytic group over a perfectoid field $K$. If $G$ admits a $[p]$-$F$-formal tower, then $G_\infty$ exists and is perfectoid. 
	\end{proposition}
	 
	\begin{remark} \label{remark:tilde_limit_unique}
In contrast to Remark \ref{remark:tilde_limit_non_unique}, by Proposition 2.4.5 of \cite{SW}, if $X \sim \varprojlim X_i$ and $X$ is perfectoid, then $X$ is the unique perfectoid tilde-limit of $ \varprojlim X_i$ up to unique isomorphism. In such situations we will refer to $X$ by \textit{the} tilde-limit of $ \varprojlim X_i$. 
	\end{remark} 
	 
	\begin{proof}
	 
 
		Let $(\{\mathfrak G_n\}_{n\in \mathbb N}, [\mathfrak p]_{n+1}:\mathfrak G_{n+1}\rightarrow \mathfrak G_{n})$ be a $[p]$-$F$-formal tower for $G$.  By Lemma~\ref{tilde-limit from adic generic fibre of formal schemes} we therefore have 
		$$G_\infty := (\varprojlim_{[\mathfrak p]}\mathfrak G)_\eta \sim \varprojlim_{[p]}G. $$
 
		
		To see that $G_\infty$ is perfectoid, we proceed as the proof of \cite{torsion}, Corollary III.2.19. It suffices to prove that $\mathfrak G_\infty = \varprojlim_{[\mathfrak p]} \mathfrak G$ can be covered by formal schemes of the form $\operatorname{Spf}(S)$ where $S$ is a flat $\mathcal O_K$-algebra such that the Frobenius map \[S/\pi^{1/p} \rightarrow  S/\pi\] is an isomorphism. Lemma~5.6 of \cite{perfectoid} then shows that $S[1/\pi]$ is perfectoid.
		
		By assumption, on the mod $\pi$ special fibre  $\tilde{G}_n$,  $[\mathfrak p]_{n+1}$ factors through the relative Frobenius. Now take   any affine open subspace $\operatorname{Spf}(S_0) \subseteq \mathfrak G_0$.  Let $[\mathfrak p^i]:   \mathfrak G_i \rightarrow  \mathfrak G_0$ be the composition $[\mathfrak p]_{i} \circ \cdots \circ [\mathfrak p]_{1}$, and let $\operatorname {Spf}S_i \subset \mathfrak G_i$ be the pullback of $\operatorname {Spf}S_0$ via $[p^i]$. Then we have a commutative diagram:
		\begin{center}
			\begin{tikzcd}[row sep = small]
				&  & \tilde{S}_{i}^{(p)} \arrow[rd, "F_{rel}"] &  & \tilde{S}_{i+1}^{(p)} \arrow[rd, "F_{rel}"] &  &  \\
				\dots \arrow[r] & \tilde{S}_{i-1} \arrow[rr] \arrow[ru, "V", dashed] &  & \tilde{S}_i \arrow[ru, "V", dashed] \arrow[rr] &  & \tilde{S}_{i+1} \arrow[r] & \dots
			\end{tikzcd}
		\end{center} where the horizontal maps are induced from $[\mathfrak p] \mod \pi$. 
		
		From this we can check on elements that relative Frobenius is an isomorphism on $\tilde{S}_\infty := \varinjlim_i \tilde{S}_i$. Since $K$ is perfectoid, we moreover have an isomorphism $\mathcal O_K/\pi^{1/p}\rightarrow \mathcal O_K/\pi$ from the absolute Frobenius on $\mathcal O_K/\pi$. Therefore absolute Frobenius on $S_\infty/\pi$ induces an isomorphism
		\[S_\infty/\pi^{1/p}\xrightarrow{\sim} S_\infty/\pi.\]
		Since each $\mathfrak G_i$ is flat, so are the $S_i$ and thus so is $S_\infty$. Thus $S_\infty[1/\pi]$ is a perfectoid $K$-algebra.
		Since $G_\infty$ is covered by affinoids of the form $\operatorname{Spf}(S_\infty)_\eta$, this shows that $G_\infty$ is perfectoid.
	\end{proof}
	
This in particular implies what we promised in Remark \ref{remark:good_reduction}: 
	\begin{corollary}\label{tilde-limit exists and is perfectoid in the good reduction case}
		Let $A$ be an abelian variety of good reduction over a perfectoid field $K$. Then $A_\infty$ exists and is perfectoid.
	\end{corollary}
	 	
More generally, If $\mathfrak G$ is a flat commutative formal group scheme such that $p$-multiplication is affine, then multiplication by $p$ on $\mathfrak G$ defines a $[p]$-$F$-formal tower for the rigid analytic group $G=\mathfrak G_\eta$, and  $G_\infty := (\varprojlim_{[p]}\mathfrak G)_\eta$ is the perfectoid tilde-limit of $\varprojlim_{[p]} G$. 

%\begin{remark} What we aim to prove in the rest of this write-up is that for a Raynaud extension $0\rightarrow T\rightarrow E\rightarrow B\rightarrow 0$, there is a $[p]$-$F$-model for $T$ which induces a $[p]$-$F$-model for $E$. This will prove that tilde-limits $T_\infty$ and $E_\infty$ exist and are perfectoid if $K$ is perfectoid.  
%\end{remark}
		
\subsection{Examples}				
		
	\begin{example}
		Let $\mathfrak G$ be the $p$-adic completion of the affine group scheme $\mathbb G_m$ over $\mathcal O_K$. The underlying formal scheme of $\mathfrak G$ is $\operatorname {Spf} \mathcal O_K\langle X^{\pm 1} \rangle$.  Multiplication by $p$ on $\mathbb G_m$ gives a $[p]$-$F$-formal tower for $G$, so for the generic fibre $G=\mathfrak G_\eta$ we obtain the perfectoid tilde-limit $G_\infty := (\mathfrak G_\infty)_{\eta} \sim \varprojlim_{[p]} G$. More precisely,  multiplication by $p$ corresponds to the homomorphism
		\[[p]:\mathcal O_K\langle X^{\pm 1} \rangle\rightarrow  \mathcal O_K\langle X^{\pm 1} \rangle, \quad X\rightarrow X^{p}.\]
		In the direct limit, we obtain $   (\varinjlim_{[p]} \mathcal O_K\langle X^{\pm 1} \rangle  )^{\wedge} = \mathcal O_K\langle  X^{\pm 1/p^\infty} \rangle$.  Therefore, taking the generic fiber we get
		$$G_\infty = \operatorname{Spa}(K\langle X^{\pm 1/p^\infty} \rangle,\mathcal O_K\langle X^{\pm 1/p^\infty} \rangle)$$
		and one can verify directly that we indeed have $G_\infty \sim \varprojlim_{[p]} G$.
	\end{example}
	
	
	\begin{example}
		An example of a very different flavour is  the $p$-adic completion  $\mathfrak G$  of the affine group scheme $\mathbb G_a$ over $\mathcal O_K$. Note that $G=\mathfrak G_\eta$ is not equal to $\mathbb G_a^{an}$, but is the closed unit disc in the latter.
		
		The underlying formal scheme of $\mathfrak G$ is $\operatorname {Spf} S$ with $S=\mathcal O_K\langle X \rangle$, and the $[p]$-multiplication is now given by
		\[[p]:\mathcal O_K\langle X \rangle\rightarrow  \mathcal O_K\langle X \rangle, \quad X\rightarrow pX.\]
		In the direct limit, we first obtain the algebra $S'_\infty  = \mathcal O_K\langle \frac{1}{p^\infty}X \rangle$, consisting of power series $f=\sum_{n=0}^\infty  a_nX^n\in \mathcal O_K[[X]]$ for which there exists an $m\in \mathbb Z_{\geq 0}$ so that $|p^{nm}a_n|\to 0$. Next we need to take the $p$-adic completion to form $S_\infty$. But we have 
		\[p^n\mathcal O_K\langle \frac{1}{p^\infty}X \rangle = p^n\mathcal O_K + X \mathcal O_K\langle \frac{1}{p^\infty}X \rangle\]
		and therefore $S'_\infty/\pi^n=\mathcal O_K/\pi^n\mathcal O_K$. Consequently, the completion is just $S_\infty = \mathcal O_K$ and thus its adic generic fiber is $G_\infty = \operatorname{Spa}(K,\mathcal O_K)$.  This is still perfectoid but is just one point!
		
		Let us explain the heuristic geometrically: on the level of $K$-points, the formal scheme $G$ is the closed unit disc and $[p]$ is scaling by $p$. A $K$-point in $\varprojlim_{[p]} G(K)$ therefore corresponds to a sequence of $K$-points of the closed unit disc of the form 
		$$x,\frac{1}{p}x,\frac{1}{p^2}x,\dots.$$ But for this to be contained in the closed unit disc, we must have $x=0$. 
		
	\end{example}
	
 
 
	
 
	
		
	\subsection{A few consequences}
	
	One reason why perfectoid limits along group morphisms are particularly interesting is that the perfectoidness ensures that the limit has again a group structure:
	
	\begin{definition}
		A \textbf{perfectoid group} is a group object in the category of perfectoid spaces.
	\end{definition}
	
			Note that the category of perfectoid spaces over $K$ has finite products, so the notion of a group object makes sense. 
	
	\begin{proposition}\label{perfectoid tilde-limit is perfectoid group in a functorial way}
		Let $G$ be a rigid group with a perfectoid tilde-limit $G_\infty$. Then
		\begin{enumerate}
		\item  there is a unique way to endow $G_\infty$ with the structure of a perfectoid group in such a way that all projections $G_\infty\rightarrow G$ are group homomorphisms
		\item given a rigid group $H$ with perfectoid tilde-limit $H_\infty\sim\varprojlim_{[p]}H$ and a group homomorphism $H\rightarrow G$, there is a unique group homomorphism $H_\infty\rightarrow G_\infty$
		commuting with all projection maps. In particular, formation of $\varprojlim_{[p]}$ is functorial.
	\end{enumerate}
	\end{proposition}
	\begin{proof}
		These are all consequences of the universal property of the perfectoid tilde-limit, cf Proposition~2.4.5 of \cite{SW}, which shows that one can argue like in the case of usual limits.
	\end{proof}


 
 
  %%%%%%%%%%%%
%%      Section 3
%%%%%%%%%%%%	
	\section{Formal models for tori}
	
	In this section we want to show that for a split rigid torus $T$ over $K$, a tilde-limit $T_\infty$ exists and is perfectoid. We do this by exhibiting a $[p]$-$F$-model of $T$.
	
	As a preparation, we consider the torus $\mathbb G_m^{\operatorname{an}}$ over $K$. Recall that it arises from rigid analytification of the affine torus $\mathbb G_m$ over $K$. Note however that $\mathbb G_m^{\operatorname{an}}$ is not affinoid (and not even quasi-compact). It contains the generic fibre of the $p$-adic completion of $\mathbb G_m$ as an open subspace. If we see $\mathbb G_m^{\operatorname{an}}$ as the rigid affine line with the origin removed, this subspace $\widehat{\mathbb G}_m$ can be identified with the open annulus of radius 1. In other words, on the level of points it corresponds to $\mathcal O_K^\times \subseteq K^\times$.
	
	Finally, recall that for every $x\in K^\times$ we have a translation map
	\[\mathbb G_m^{\operatorname{an}}\xrightarrow{x\cdot} \mathbb G_m^{\operatorname{an}}\]
	that is an isomorphism of rigid spaces sending the point $1$ to $x$.
	
	\subsection{A family of explicit covers}
	We briefly recall how $\mathbb G_m^{\operatorname{an}}$ is constructed: The following is inspired by \cite{Bosch lectures}, \S 9.2, although we choose slightly different constructions. Throughout we use the following shorthand notation: For any $a\in K$ we write
	\[K\langle X,a/X\rangle = K\langle X,Z\rangle/(X\cdot Z - a). \]
	
	Let $q \in K^\times$ with $|q|\leq 1$. Consider the annulus $\mathcal B(q,1)$ of radii $|q|$ and $1$ inside $\mathbb A_K^{\operatorname{an}}$:
	\[\mathcal B(q,1) = \operatorname{Sp}(L_q),\quad \text{ where } L_q = K\langle X,q/X\rangle. \]
	Similarly, for $q\in K^\times$ with $|q|\geq 1$ one constructs the annulus $\mathcal B(1,q)$ by
	\[\mathcal B(1,q) = \operatorname{Sp}(L_{q}),\quad \text{ where } L_{q} = K\langle X/q,1/X\rangle\]
	where $K\langle X/q\rangle$ denotes the ring of those power series $f=\sum c_nX^m\in K[[X]]$ for which $|c_nq^n|\to 0$ for $n\to \infty$. In particular, we have isomorphisms
	 \[K\langle X',q^{-1}/X'\rangle\cong K\langle X/q,1/X\rangle,\quad X'\mapsto q^{-1}X.\]
	One can now construct $\mathbb G_m^{\operatorname{an}}$ as follows: Choose sequences $a_n,b_n\in K^\times$ with $a_0=1=b_0$ such that $|a_n|<|a_{n-1}|<\cdots<1$ and $|a_n| \to 0$ and similarly $|b_n|>|b_{n-1}|>\cdots>1$ and $|b_n| \to \infty$. Then one can glue the annuli $\mathcal B(a_n,1)$ and $\mathcal B(1,b_n)$ using the following maps:
	\begin{equation}\label{torus explicit cover glue map 1}
	\begin{alignedat}{2}
	\mathcal B(a_{n},1)&\hookleftarrow&& \mathcal B(a_{n-1},1)\\
	L_{a_n}=K\langle X,a_n/X\rangle&\rightarrow &&K\langle X,a_{n-1}/X\rangle=L_{a_{n-1}}\\
	X,a_{n}/X&\mapsto&& X, \frac{a_{n}}{a_{n-1}}a_{n-1}/X
	\end{alignedat}
	\end{equation}
	and similarly 
	\begin{equation}\label{torus explicit cover glue map 2}
	\begin{alignedat}{2}
	\mathcal B(1,b_n)&\hookleftarrow&& \mathcal B(1,b_{n-1})\\
	L_{b_n}=K\langle  X/b_{n},1/X\rangle&\rightarrow &&K\langle X/b_{n-1},1/X\rangle=L_{b_{n-1}}\\
	X/b_{n},1/X&\mapsto&& \frac{b_{n-1}}{b_{n}} X/b_{n-1}, 1/X.
	\end{alignedat}
	\end{equation}
	
	Also, via the above maps, the annuli $\mathcal B(a_n,1)$ and $\mathcal B(1,b_m)$ are glued along $\mathcal B(a_0,1)=\mathcal B(1,1)=\mathcal B(1,b_0).$ This gives the desired space $\mathbb{G}_m^{\operatorname{an}}$.
	
	Since we are mainly interested in the $p$-multiplication map, we will more precisely use the following cover on which $[p]$ can be seen directly: Choose $q\in K^\times$ with $|q|<1$. Then for the sequences $a_n$ and $b_n$ from above we take $a_n = q^n$, $b_n = q^{-n}$. 
	We call this cover $\mathfrak U_q$.
	
	Assume now that $q$ has a $p$-th root $q^{1/p}$ in $K$. The above then gives a finer cover $\mathfrak U_{q^{1/p}}$ of $\mathbb G_m^{\operatorname{an}}$. Using both covers $\mathfrak U_q$ and $\mathfrak U_{q^{1/p}}$, we can easily see the $[p]$-multiplication $[p]:\mathbb G_m^{\operatorname{an}}\rightarrow \mathbb G_m^{\operatorname{an}}$ as follows: Consider the affinoid open subsets $\mathcal B(q^{1/p},1)$ of the source and  $\mathcal B(q,1)$ of the target. Then $[p]$ restricts to
	\begin{equation}
	\begin{alignedat}{2} \label{torus explicit [p] map 1}
	\mathcal B(q,1)&\xleftarrow{[p]}&& \mathcal B(q^{1/p},1)\\
	K\langle X,q/X\rangle&\rightarrow &&K\langle X,q^{1/p}/X\rangle\\
	X,q/X&\mapsto&& X^p, (q^{1/p}/X)^p
	\end{alignedat}
	\end{equation}
	and similarly, on $\mathcal B(1,q^{-1/p})$ and $\mathcal B(1,q^{-1})$ the map is 
	\begin{equation}
	\begin{alignedat}{2} \label{torus explicit [p] map 2}
	\mathcal B(1,q^{-1})&\xleftarrow{[p]}&& \mathcal B(1,q^{-1/p})\\
	K\langle X/q,1/X\rangle&\rightarrow &&K\langle X/q^{1/p},1/X\rangle\\
	X/q,1/X&\mapsto&& (X/q^{1/p})^p, (1/X)^p.
	\end{alignedat}
	\end{equation}
	The same works for the other affinoid open subspaces $\mathcal B(q^{n},1)\xleftarrow{[p]} \mathcal B(q^{n/p},1)$ and for $\mathcal B(1,q^{-n})\xleftarrow{[p]} \mathcal B(1,q^{-n/p})$.
	One can then show that the maps~(\ref{torus explicit [p] map 1}) and (\ref{torus explicit [p] map 2}) are compatible with the gluing maps~(\ref{torus explicit cover glue map 1}) and~(\ref{torus explicit cover glue map 2}). In the case of~(\ref{torus explicit [p] map 1}) this is basically because $a_n/a_{n-1} = q$ or $a_n/a_{n-1}=q^{1/p}$ depending on whether we work with $\mathfrak U_q$ or $\mathfrak U_{q^{1/p}}$ respectively, and the only thing to check is that the following diagram commutes:
	\begin{center}
		\begin{equation}\label{diagram showing that [p]-multiplication on torus glues together}
		\begin{tikzcd}
			\mathcal B(q^{n},1) & \mathcal B(q^{n-1},1) \arrow[l, hook'] & q^n/X \arrow[r, maps to] \arrow[d, maps to] & q\cdot q^{n-1}/X \arrow[d, maps to] \\
			\mathcal B(q^{n/p},1) \arrow[u, "{[p]}"] & \mathcal B(q^{(n-1)/p},1) \arrow[u, "{[p]}"'] \arrow[l, hook'] & (q^{n/p}/X)^p \arrow[r, maps to] & q\cdot(q^{(n-1)/p}/X)^p. 
		\end{tikzcd}
		\end{equation}
	\end{center} 
	The case of~(\ref{torus explicit [p] map 2}) is very similar.
	
	\subsection{A family of formal models}
	At this point we have constructed a cover $\mathfrak U_q$ of $\mathbb G_m^{\operatorname{an}}$ depending on a choice of $q\in K^\times$ with $|q|<1$. 
	The affinoid subspaces $\mathcal B(q^n,1)$ that we have used for this admit natural formal models: Namely, consider the $\mathcal O_K$-algebra
	\[L_q^\circ := \mathcal O_K\langle X,Z\rangle/(XZ-q).\]
	This is clearly of topologically finite type over $\mathcal O_K$. It is moreover flat as an $\mathcal O_K$-algebra (this should follow from Lemma 8.2.1 in \cite{Bosch lectures}). For the same reason (or by $L_{q^{-1}} \cong L_q$) we see that \[L_{q^{-1}}^\circ := \mathcal O_K\langle X',Z\rangle/(X'Z-q)\] is a flat topologically finite type $\mathcal O_K$-algebra. Consequently, we have flat formal models 
	\begin{alignat*}{4}
		\mathfrak B(q,1)&:=&& \operatorname{Spf}(L_{q}^\circ), &\quad&\mathfrak B(q,1)_\eta &=& \mathcal B(q,1)\\ 
		\mathfrak B(1,q)&:=&& \operatorname{Spf}(L_{q^{-1}}^\circ), &\quad&\mathfrak  B(1,q)_\eta &=& \mathcal B(1,q).
	\end{alignat*}
	For the glueing maps~(\ref{torus explicit cover glue map 1}) and (\ref{torus explicit cover glue map 2}) it is clear from $a_n/a_{n-1} = b_{n-1}/b_n = q$ that these extend to glueing maps $\mathfrak B(q^n,1)\hookleftarrow \mathfrak B(q^{n-1},1)$ and $\mathfrak B(1,q^{-n})\hookleftarrow \mathfrak B(1,q^{-(n-1)})$. We conclude:

	\begin{lemma}\label{formal model of torus}
		The affine formal schemes $\mathfrak B(q^n,1)$ and $\mathfrak B(1,q^n)$ glue together to a flat formal scheme $\mathfrak G_q$ such that $(\mathfrak G_q)_\eta = \mathbb G_m^{\operatorname{an}}$. In other words, $\mathfrak G_q$ is a formal model for $\mathbb G_m^{\operatorname{an}}$.
	\end{lemma}
	\subsection{A family of formal models for $p$-multiplication}
	As before choose $q\in K^\times$ such that $|q|<1$ and such that there exists a $p$-th root $q^{1/p} \in K$. A closer look at the maps~(\ref{torus explicit [p] map 1}) and~(\ref{torus explicit [p] map 2}) shows that the $[p]$-multiplication extends to a morphism of formal schemes
	\[\mathfrak B(q,1)\xleftarrow{[p]} \mathfrak B(q^{1/p},1):[\mathfrak p]\]
	and similarly for $\mathfrak B(1,q^{-1})$. The diagram~(\ref{diagram showing that [p]-multiplication on torus glues together}) shows that these maps glue to a morphism
	\[[\mathfrak p]: \mathfrak G_{q^{1/p}}\rightarrow  \mathfrak G_q.\]
	By construction, after tensoring $-\otimes_{\mathcal O_K} K$ all morphisms on algebras coincide with those defined in ~(\ref{torus explicit cover glue map 1}),~(\ref{torus explicit cover glue map 2}),~(\ref{torus explicit [p] map 1}), (\ref{torus explicit [p] map 2}) respectively. We conclude:
	\begin{proposition}
		The map $[\mathfrak p]: \mathfrak G_{q^{1/p}}\rightarrow  \mathfrak G_q$ is a formal model of $[p]:\mathbb G_m^{\operatorname{an}}\rightarrow \mathbb G_m^{\operatorname{an}}$.
	\end{proposition}
	We moreover see directly from the construction:
	\begin{proposition}
		The map $[\mathfrak p]: \mathfrak G_{q^{1/p}}\rightarrow  \mathfrak G_q$ reduces mod $\pi$ to the relative Frobenius map.
	\end{proposition}
	We now have everything together to finish our proof that $(\mathbb G_m^{\operatorname{an}})_\infty$ is perfectoid:
	\begin{proposition}
		The space $\mathbb G_m^{\operatorname{an}}$ has a $[p]$-$F$-formal tower. In particular, there exists a perfectoid space $(\mathbb G_m^{\operatorname{an}})_\infty$ such that $(\mathbb G_m^{\operatorname{an}})_\infty \sim \varprojlim_{[p]} \mathbb G_m^{\operatorname{an}}$.
	\end{proposition}
	\begin{proof}
		Since $K$ is perfectoid, we can find $q\in K^\times$ such that $|q|<1$ for which there exist arbitrary $p^n$-th roots. We choose such a $q$ and roots $q^{1/p^n}$ for all $n$. Then the two Propositions above combine to show that 
		\[\dots \xrightarrow{[\mathfrak p]} \mathfrak G_{q^{1/p^2}}\xrightarrow{[\mathfrak p]} \mathfrak G_{q^{1/p}}\xrightarrow{[\mathfrak p]} \mathfrak G_q\]
		is a $[p]$-$F$-formal tower.
		By Proposition~\ref{existence of p-F-formal tower implies perfectoid} we obtain a perfectoid space $(\mathbb G_m^{\operatorname{an}})_\infty$ as desired.
	\end{proof}
	
	\subsection{The action of $\overline{T}$}
	The multiplication $\mathbb G_m^{\operatorname{an}}\times \mathbb G_m^{\operatorname{an}}\rightarrow \mathbb G_m^{\operatorname{an}}$ can locally be described in terms of the rigid analytic cover that we have defined above as follows  : Let $a,b \in K^\times$ such that $|a|,|b|\leq 1$, then the multiplication map restricts to
	\begin{equation}
	\begin{alignedat}{2} \label{torus multiplication map}
	\mathcal B(a,1)\times \mathcal B(b,1)&\xrightarrow{m}&& \mathcal B(ab,1)\\
	K\langle X,ab/X\rangle&\rightarrow &&K\langle X,a/X\rangle\widehat{\otimes} K\langle X,b/X\rangle\\
	X&\mapsto&& X\otimes X\\
	ab/X&\mapsto&& b/X\otimes a/X
	\end{alignedat}
	\end{equation}
	and similarly on the $\mathcal B(1,a)\times \mathcal B(1,b)$ for $|a|,|b|\geq 1$. Multiplication on the $\mathcal B(a,1)\times \mathcal B(1,b)$ for $|a|< 1 < |b|$ is more difficult to see on the cover that we have chosen.
	
	The same arguments as in the last section show that the map described in~(\ref{torus multiplication map}) has a flat formal model
	\[\mathfrak B(a,1)\times \mathfrak B(b,1)\rightarrow \mathfrak B(ab,1).\]
	This does \textit{not} mean that multiplication has a formal model $\mathfrak G\times \mathfrak G\rightarrow \mathfrak G$. Indeed, the chosen description has different covers on source and target which in the formal case give rise to different formal schemes (the inversion map $i:\mathbb G_m^{\operatorname{an}}\rightarrow \mathbb G_m^{\operatorname{an}}$ on the other hand does have a formal model). Nevertheless, if we take $a=1$ in the above, we see that we do have an action of the torus $\overline{T}:=\mathfrak B(1,1)$ on each of $\mathfrak B(b,1)$ and $\mathfrak B(1,b)$. Using the formal models from the last section, we conclude:
	
	\begin{proposition}\label{action on formal model of torus}
		For any $q\in K^\times$ with $|q|<1$, the formal torus $\overline{T}:=\mathfrak B(1,1)$ has a natural action on $\mathfrak G_q$ via a map
		\[\mathfrak m:\overline{T}\times \mathfrak G_q\rightarrow \mathfrak G_q.\]
		This map is a formal model of the action of the annulus $\mathcal B(1,1)$ on $\mathbb G_m^{\operatorname{an}}$. Furthermore, this action is compatible with the models for $[p]$ in the sense that if there is a $p$-th root $q^{1/p}\in K$, then the following diagram commutes.
		\begin{center}
		\begin{tikzcd}
			\overline{T}\times \mathfrak G_{q^{1/p}} \arrow[d, "{[p]\times [\mathfrak p]}"'] \arrow[r, "\mathfrak m"] & \mathfrak G_{q^{1/p}} \arrow[d, "{[\mathfrak p]}"] \\
			\overline{T}\times \mathfrak G_{q} \arrow[r, "\mathfrak m"] & \mathfrak G_{q}.
		\end{tikzcd}
		\end{center}
	\end{proposition} 
	\begin{proof}
		The existence of $\mathfrak m$ follows from the above consideration concerning the map~(\ref{torus multiplication map}). The rest is clear from the construction: All adic rings we have used in the construction are $\mathcal O_K$-subalgebras of the affinoid $K$-algebras used to define $\mathbb G_m^{\operatorname{an}}$, so the equalities hold because they hold for $\mathbb G_m^{\operatorname{an}}$.
	\end{proof}
	
	\subsection{The case of general tori}
	By taking products everywhere, all of the statements in this section immediately generalise to split tori:
	\begin{corollary}\label{torus has formal models}
		Let $T$ be a split torus over $K$ of the form $T=(\mathbb G_m^{\operatorname{an}})^d$. Then for any $q\in K^\times$ with $|q|<1$ the formal scheme $\mathfrak T_q := (\mathfrak G_q)^d$ is a formal model of $T$. If there is a $p$-th root $q^{1/p}\in K$, the $p$-multiplication map has a formal model $[\mathfrak p]:\mathfrak T_{q^{1/p}}\rightarrow \mathfrak T_{q}$ that locally on polyannuli is of the form $[\mathfrak p]:\mathfrak B(q^{1/p},1)^d\rightarrow \mathfrak B(q,1)^d$. Moreover this map reduces mod p to the relative Frobenius morphism.
	\end{corollary}
	\begin{corollary}\label{torus has p-F-formal tower and has perfectoid tilde-limit}
		Let $T$ be a split torus over $K$, considered as a rigid space. Then $T$ has a $[p]$-$F$-formal tower. In particular, there exists a perfectoid space $T_\infty$ such that $T_\infty \sim \varprojlim_{[p]} T$. 
	\end{corollary}
	
	\begin{corollary}\label{action on formal model of torus}
		Let $T$ be any split torus over $K$. For any $q\in K^\times$ with $|q|<1$, the formal completion $\overline{T}$ has a natural action on $\mathfrak T_q$ via a map
		\[\mathfrak m:\overline{T}\times \mathfrak T_q\rightarrow \mathfrak T_q.\]
		This map is a formal model of the action of the annulus $\overline{T}$ on $T$. Furthermore, this action is compatible with the models for $[p]$ in the sense that if there is a $p$-th root $q^{1/p}\in K$, then the map $[\mathfrak p]:\mathfrak T_q^{1/p}\rightarrow \mathfrak T_q$ is semi-linear with respect to $[p]:\overline{T}\rightarrow \overline{T}$.
	\end{corollary} 
	
%%%%%%%%%%%%
%%      Section 4
%%%%%%%%%%%%
	\section{Raynaud extensions as principal bundles of formal and rigid spaces}\label{Raynaud extensions as principal bundles of formal and rigid spaces}
	As the next step towards studying the $p$-multiplication tower of an abelian variety, we now look at the case of rigid groups arising from the Raynaud extensions associated to abelian varieties over an algebraically closed non-archimedean complete field $K$.
	
	While we expect that our main theorem still holds over any perfectoid field, it is easier to work in the algebraically closed case since this simplifies the theory of Raynaud uniformisation, in particular Lemma~\ref{formal Raynaud sequence is locally split} below.	We sketch the theory here, see \S 5.6 of \cite{rigid geometry of curves} for more details on the setup.
	
	Let now $A$ be an abelian variety over $K$. There exists a unique connected open rigid analytic subgroup $\overline A$ of $A$ which extends to a formal smooth $\mathcal O_K$-group scheme $\overline E$ with semi-abelian reduction. Then $\overline E$ fits into a short exact sequence of formal group schemes
	\begin{equation}\label{formal Raynaud extension}
	0\rightarrow \overline T \rightarrow \overline E \xrightarrow{\pi} \overline{B}\rightarrow 0
	\end{equation}
	where $\overline{B}$ is the completion of an abelian variety $B$ over $K$ of good reduction (we also denote by $B$ the rigid space associated to it), and $\overline{T}$ is the completion of a torus of rank $r$ over $K$.
	The rigid generic fibre $\overline{T}_\eta$ of the torus $\overline{T}$ canonically embeds into the torus $T^{\operatorname{an}}$ which again we simply denote by $T$. One can show that this induces a pushout exact sequence in the category of rigid groups. More precisely, there exists a rigid group variety $E$ such that the following diagram commutes and the left square is a pushout.
	\begin{center}
		\begin{equation}
		\begin{tikzcd}
			0 \arrow[r] & \overline{T}_\eta \arrow[d, hook] \arrow[r] & \overline{E}_\eta \arrow[d, hook] \arrow[r] & \overline{B}_\eta \arrow[d,equal] \arrow[r] & 0 \\
			0 \arrow[r] & T \arrow[r] & E \arrow[r] & B \arrow[r] & 0
		\end{tikzcd}
		\end{equation}
	\end{center}
	
	The abelian variety $A$ we started with can then be uniformized in terms of $E$ as follows:
	
	\begin{definition}
		A subset $M$ of a rigid space $G$ is called \textbf{discrete} if the intersection of $M$ with any affinoid open subset of $G$ is a finite set of points.
		Let $G$ be a rigid group, then a \textbf{lattice} in $G$ of rank $r$ is a discrete subgroup $M$ of $G$ which is isomorphic to the constant rigid group $\underline{\mathbb Z^r}$. 
	\end{definition}
	
	\begin{proposition}\label{Raynaud uniformisation}
		There exists a lattice $M \subseteq E$ of rank equal to the rank $r$ of the torus for which the quotient $E/M$ exists as a rigid space and has a group structure such that $E\rightarrow E/M$ is a rigid group homomorphism. Moreover, there is a natural isomorphism
		\[A=E/M.\]
	\end{proposition}
	
	Since $M$ is discrete, the local geometry of $A$ is thus determined by the local geometry of $E$. More precisely, we will first study the $[p]$-multiplication tower of $E$ and will then deduce properties of the $[p]$-multiplication tower of $A$.
	
	In order to do so, we would like to study the local geometry of $E$ and $\overline{E}$ via $T$ and $B$. An obstacle in doing this is that the categories of formal or rigid groups are not abelian, which makes working with short exact sequences a subtle issue. Another issue is that one cannot direcly study short exact sequences locally on $T$, $E$ or $B$. An important tool is therefore the following Lemma:

	\begin{lemma}\label{formal Raynaud sequence is locally split}
		The short exact sequence (\ref{formal Raynaud extension}) admits local sections, that is there is a cover of $B$ by formal open subschemes $U_i$ such that there exist sections $s:U_i\rightarrow \overline{E}$ of $\pi$. In particular, one can cover $\overline{E}$ by formal open subschemes of the form $\overline{T}\times U_i\hookrightarrow E$.
	\end{lemma}
	\begin{proof}
		This is proved in Proposition A.2.5 in~\cite{rigid geometry of curves}, where it is fomulated in terms of the group $\operatorname{Ext}(B,T)$. Also see \cite{BL}, \S 1.
	\end{proof}
	
	This Lemma suggests that instead of considering Raynaud extensions from the abelian category viewpoint, one should consider them as fibre bundles of formal schemes with structure group $T$, or more precisely as principal $T$-bundles of formal schemes, which are also called $T$-torsors. This is the language we want to use in the following: We will work with fibre bundles of formal schemes, rigid spaces and schemes. The main technical tool we will need is the associated fibre construction in these settings. For a rigorous  treatment of these we refer to the Appendix.
	
	First of all, we note that the sequence~(\ref{formal Raynaud extension}) from the last section gives rise to a principal $\overline{T}$-bundle
	$\overline{E}\rightarrow \overline{B}$. The fact that $E$ is obtained from $\overline{E}_\eta$ via push-out from $\overline{T}_\eta\rightarrow T$ can now conveniently be expressed in terms of the associated fibre bundle by saying that $E_\eta = T\times^{\overline{T}_\eta}\overline{E}_\eta$ in the sense of Definition~\ref{definition of Borel construction}. We have the following description of $[p]$:
	\begin{lemma}\label{p-multiplication is induced from Borel construction}
		The map $[p]:E\rightarrow E$ coincides with the morphism 
		\[[p]\times^{[p]}[p]: T\times^{\overline{T}_\eta}\overline{E}_\eta\rightarrow T\times^{\overline{T}_\eta}\overline{E}_\eta\]
		induced by the different $[p]$-multiplication maps by Proposition~\ref{associated bundle construction in the semi-linear case is a sort of fibered bifunctor}.
	\end{lemma}
	\begin{proof}
		Lemma~\ref{universal property of associated fibre construction in the semilinear case} in light of Remark~\ref{appendix in the case of rigid spaces and schemes} applied to the maps $g=[p]:\overline{T}_\eta\rightarrow \overline{T}_\eta$, $h=[p]:T\rightarrow T$ and $f=[p]:\overline{E}_\eta\rightarrow \overline{E}_\eta$ says that there is a unique morphism of fibre bundles $E\rightarrow E$ making the following diagram commute:
		\begin{center}
			\begin{equation}\label{rigid p-multiplication cube}
			\begin{tikzcd}[row sep = small, column sep = small]
				& T \arrow[rr] &  & E \\
				T \arrow[ru, "{[p]}"] \arrow[rr,crossing over] &  & E \arrow[ru, "\exists!", dotted] &  \\
				& \overline{T}_\eta \arrow[uu, hook] \arrow[rr,crossing over] &  & \overline{E}_\eta \arrow[uu] \\
				\overline{T}_\eta \arrow[ru, "{[p]}"] \arrow[uu, hook] \arrow[rr] &  & \overline{E}_\eta \arrow[ru, "{[p]}"] \arrow[uu] & 
			\end{tikzcd}
			\end{equation}
		\end{center}
		Since $[p]:E\rightarrow E$ is such a map, the Lemma follows.
	\end{proof}

 
%%%%%%%%%%%%
%%      Section 5
%%%%%%%%%%%%	
	
	\section{Formal models for $E$}
	In this subsection we prove that $E$ admits a $[p]$-$F$-tower model. The first step is to construct a family of formal models for $E$. We do this by using the formal models $\mathfrak T_q$.
	\begin{proposition}
	Let $q\in K^\times$ with $|q|<1$. Let $\mathfrak T_q$ be the formal model from Corollary~\ref{torus has formal models}. Then there is a formal scheme $\mathfrak E_q :=\mathfrak T_q \times^{\overline{T}}\overline{E}$ that is a formal model of the rigid space $E$. Furthermore, there exists a morphism
	\[\mathfrak E_q :=\mathfrak T_q \times^{\overline{T}} \overline{E} \rightarrow \overline{B} \]
	which is a fibre bundle and a formal model of $E\rightarrow B$.
	\end{proposition}
	\begin{proof}
		Recall from Proposition~\ref{action on formal model of torus} that $\mathfrak T_q$ has a $\overline{T}$-action that is a model of the $\overline{T}_\eta$-action on $T$. In particular, the associated fibre construction for the principal $\overline{T}$-bundle $\overline{E}$ gives a fibre bundle $\mathfrak E_q :=\mathfrak T_q \times^{\overline{T}} \overline{E} \rightarrow \overline{B}$. Since $\mathfrak T_q$ is a formal model of $T$, one sees by Lemma~\ref{associated bundle commutes with generic fibre} that this is a formal model of $T\times^{\overline{T}_\eta}\overline{E}_\eta$ which by definition is equal to $E$.
	\end{proof}
	Next we want to construct a model for the $[p]$-multiplication map. Here we can use again that $[p]$ exists on $\overline{E}$ and on $\mathfrak T_{q^{1/p}}$.
	\begin{proposition}\label{formal model of p-multiplication on E}
		Let $q\in K^\times$ be such that $|q|<1$ and let $q^{1/p}\in K$ be a $p$-th root of $q$. Then there is an affine morphism
		\[[\mathfrak p]:\mathfrak E_{q^{1/p}} \rightarrow  \mathfrak E_{q}\]
		which is a formal model of $[p]:E\rightarrow E$.
	\end{proposition}
		Note that we can always choose a $p$-th root of $q$ in $K$ due to our running assumption that $K$ is algebraically closed.
	\begin{proof}
		Recall that the multiplication map $[p]:T\rightarrow T$ has a formal model $[\mathfrak p]:\mathfrak T_{q^{1/p}}\rightarrow \mathfrak T_q$ by Corollary~\ref{torus has formal models}. This fits into a commutative diagram
		\begin{center}
			\begin{tikzcd}
				\mathfrak T_{q^{1/p}} \arrow[r, "{[\mathfrak p]}"] & \mathfrak T_q \\
				\overline{T} \arrow[u, hook] \arrow[r, "{[p]}"] & \overline{T} \arrow[u, hook].
			\end{tikzcd}
		\end{center}
		
		Functoriality of the associated fibre construction in the general case, Proposition~\ref{associated bundle construction in the semi-linear case is a sort of fibered bifunctor}, applied to the diagram below then gives a natural map $\mathfrak E_{q^{1/p}}\rightarrow \mathfrak E$ making the diagram commute:
		\begin{center}
			\begin{equation}\label{formal model of p-multiplication cube}
			\begin{tikzcd}[column sep={1.3cm,between origins},row sep={1.3cm,between origins}]
				& \mathfrak T_{q} \arrow[rr] &  & \mathfrak E_q \\
				\mathfrak T_{q^{1/p}} \arrow[ru, "{[\mathfrak p]}"] \arrow[rr] &  & \mathfrak T_{q^{1/p}}\times^{\overline{T}}\overline{E} \arrow[ru, "\exists", dotted] &  \\
				& \overline{T} \arrow[uu] \arrow[rr] &  & \overline{E} \arrow[uu] \\
				\overline{T} \arrow[uu] \arrow[rr] \arrow[ru, "{[p]}"] &  & \overline{E} \arrow[uu] \arrow[ru, "{[p]}"] & 
			\end{tikzcd}
			\end{equation}
		\end{center}
		By Lemma~\ref{p-multiplication is induced from Borel construction} this diagram equals diagram~(\ref{rigid p-multiplication cube}) on the generic fibre. 
	
		To see that the morphism $[\mathfrak p]:\mathfrak E_{q^{1/p}} \rightarrow  \mathfrak E_{q}$ is affine, first note that $[p]:\overline{B}\rightarrow \overline{B}$ is an affine morphism. The map $[\mathfrak p]:\mathfrak T_{q^{1/p}}\rightarrow \mathfrak T_{q}$ is affine by construction, namely by Corollary~\ref{torus has formal models} it is locally on $\mathfrak T_{q}$ of the form $[\mathfrak p]:\mathfrak B(q^{1/p},1)^d\rightarrow \mathfrak B(q,1)^d$. Note that both of these affine open subsets are fixed by the action of $\overline{T}$.
		We conclude from the construction in the proof of Proposition~\ref{associated bundle construction in the semi-linear case is a sort of fibered bifunctor} that the morphism  $[\mathfrak p]:\mathfrak E_{q^{1/p}} \rightarrow  \mathfrak E_{q}$ locally on the target is of the form
		\[[\mathfrak p]:\mathfrak B(q^{1/p},1)^d \times U' \rightarrow \mathfrak B(q,1)^d \times U\]
		for an affine open formal subscheme $U\subseteq \overline{B}$ with affine preimage $U'$ under $[p]:\overline{B}\rightarrow \overline{B}$. This shows that the morphism is affine locally on the target, and therefore is affine.
	\end{proof}
	
	We have thus proved the first part of what we want to show about tilde-limits of $E$:
	\begin{proposition}\label{p-formal tower exists for E}
		Let $K$ be perfectoid. Then $E$ has a $[p]$-formal tower of the form
		\[\dots \xrightarrow{[\mathfrak p]} \mathfrak E_{q^{1/p^2}}\xrightarrow{[\mathfrak p]} \mathfrak E_{q^{1/p}}\xrightarrow{[\mathfrak p]} \mathfrak E_q\]
		for some $q\in K^\times$. In particular, there exists a space $E_\infty$ such that $E_\infty\sim \varprojlim_{[p]}E$.
	\end{proposition}
	\begin{proof}
		By Proposition~\ref{formal model of p-multiplication on E}, any choice of $q\in K^\times$ with $|q|<1$ for which there exists a compatible system of $p^n$-th roots $q^{1/p^n}\in K^\times$ gives a tower
		\[\dots \xrightarrow{[\mathfrak p]} \mathfrak E_{q^{1/p^2}}\xrightarrow{[\mathfrak p]} \mathfrak E_{q^{1/p}}\xrightarrow{[\mathfrak p]} \mathfrak E_q\]
		that on the generic fibre equals $\dots\xrightarrow{[p]} E\xrightarrow{[p]} E$. This is the desired $[p]$-formal tower.
	\end{proof}
	
	We are now ready to prove the main result of this section, namely that $E_\infty$ is perfectoid.
	%\begin{framed}
	\begin{theorem}\label{p-F-formal tower exists for E}
		Let $K$ be perfectoid. Then the $[p]$-formal tower from Proposition~\ref{p-formal tower exists for E}
		\[\dots \xrightarrow{[\mathfrak p]} \mathfrak E_{q^{1/p^2}}\xrightarrow{[\mathfrak p]} \mathfrak E_{q^{1/p}}\xrightarrow{[\mathfrak p]} \mathfrak E_q\]
		 is already a $[p]$-$F$-formal tower.
		In particular, the space $E_\infty$ is perfectoid.
	\end{theorem}
	%\end{framed}
	\begin{proof}
	
	It suffices to prove that for any $q\in K^\times$ with $|q|<1$ and a $p$-th root $q^{1/p}$, the map $[\mathfrak p]:\mathfrak E_{q^{1/p}}\xrightarrow{} \mathfrak E_q$ upon reduction mod $\pi$ factors through relative Frobenius.
	
	In the following we denote reduction of a formal scheme by a $\sim$ over the formal scheme, for example the reductions of $\overline{T}$, $\overline{E}$ and $\mathfrak T$ are denoted by $\tilde{T}$, $\tilde{E}$ and $\tilde{\mathfrak{T}}$.
	
		
	Recall that $[\mathfrak p]:\mathfrak E_{q^{1/p}}\xrightarrow{} \mathfrak E_q$ was constructed using the $[p]$-multiplication cube in diagram~(\ref{formal model of p-multiplication cube}) and functoriality of the associated bundle. 	
	Also recall that all statements we have used about fibre bundles also hold when we replace formal schemes over $\mathcal O_K$ by schemes over $\mathcal O_K/\pi$, and formation of the associated bundle commutes with this reduction. In particular,
	\[\tilde{\mathfrak{E}}_q = \tilde{\mathfrak T}_q\times^{\tilde{T}}\tilde E.\]
	By  Corollary~\ref{torus has formal models}, the model of the multiplication map $[\mathfrak p]:\mathfrak T_{q^{1/p}} \rightarrow \mathfrak T_{q}$ reduces to relative Frobenius over $p$. In particular, we have a natural isomorphism
	\[\tilde{\mathfrak T}_{q^{1/p}}^{(p)} \cong \tilde{\mathfrak T}_{q}\]
	and we can identify $\tilde{\mathfrak T}_{q^{1/p}}^{(p)} = \tilde{\mathfrak T}_{q}$ in the following. The same is true for $\tilde{T}^{(p)} = \tilde{T}$.
	
	Since $\tilde{E}$ and $\tilde{T}$ are group schemes, the reduction of $[p]$ on them factors through the relative Frobenius maps $F_E$ and $F_T$ respectively. By functoriality of relative Frobenius ("Frobenius commutes with any map") we have a commutative diagram
	\begin{center}
		\begin{tikzcd}
			\tilde{E} \arrow[r, "F_{\tilde E}"] &  \tilde{E}^{(p)} \\
			\tilde{T} \arrow[r, "F_{\tilde T}"] \arrow[u, hook] &  \tilde{T}^{(p)} \arrow[u, hook].
		\end{tikzcd}
	\end{center}
	In other words, $F_{\tilde E}$ is an $F_{\tilde T}$-linear morphism of fibre bundles. Again by functoriality of Frobenius we also have a commutative diagram
	\begin{center}
	\begin{tikzcd}
	\tilde{\mathfrak T}_{q^{1/p}} \arrow[r, "F_{\tilde { \mathfrak T}}"]  & \tilde{\mathfrak T}_{q^{1/p}}^{(p)} \\
	\tilde T \arrow[r, "F_{\tilde T}"] \arrow[u,hook] & \tilde{T}^{(p)} \arrow[u,hook].
	\end{tikzcd}
	\end{center}
	
	By Proposition~\ref{associated bundle construction in the semi-linear case is a sort of fibered bifunctor}, we thus obtain a natural morphism
	\[F_{\tilde{\mathfrak{T}}}\times^{F_{\tilde{T}}} F_{\tilde
		E}:\tilde{\mathfrak T}_{q^{1/p}}\times^{\tilde T}\tilde E \rightarrow \tilde{\mathfrak T}_{q^{1/p}}^{(p)}\times^{\tilde T^{(p)}}\tilde E^{(p)}. \]
	Using the explicit description of $F_{\tilde{\mathfrak{T}}}\times^{F_{\tilde{T}}} F_{\tilde
		E}$ in the proof of Proposition~\ref{associated bundle construction in the semi-linear case is a sort of fibered bifunctor}, we easily check that this morphism is just the relative Frobenius of $\tilde{\mathfrak{E}}_{q^{1/p}}$: This is a consequence of the fact that relative Frobenius on the fibre product $\tilde{T}\times \tilde{U}$ for any $\tilde{U}\subseteq \tilde{B}$ is just the product of the relative Frobenius morphisms of $\tilde T$ and $\tilde U$, and thus the morphisms $\theta_i$ from Lemma~\ref{equivalent characterisation of morphisms of principal T-bundles} are all trivial. 
	
	But this means that again by Proposition~\ref{associated bundle construction in the semi-linear case is a sort of fibered bifunctor}, the reduction of the formal model of the $p$-multiplication cube in diagram~\ref{formal model of p-multiplication cube} admits the following factorisation:
	
	\begin{center}
		\begin{tikzcd}[column sep={1cm,between origins},row sep={1cm,between origins}]
			&  &  &  & \tilde{\mathfrak T}_{q} \arrow[rrr] &  &  & \tilde{\mathfrak E}_{q} \\
			&  & \tilde{\mathfrak T}_{q}^{(p)} \arrow[rru,equal] \arrow[rrr] &  &  & \tilde{\mathfrak E}^{(p)}_{q^{1/p}} \arrow[rru, dotted] &  &  \\
			\tilde{\mathfrak T}_{q^{1/p}} \arrow[rru, "\, F"'] \arrow[rrr] &  &  & \tilde{\mathfrak E}_{q^{1/p}} \arrow[rru, "\, F"'] &  &  &  &  \\
			&  &  &  & \tilde{T} \arrow[rrr] \arrow[uuu] &  &  & \tilde{E} \arrow[uuu] \\
			&  & \tilde{T}^{(p)} \arrow[rrr] \arrow[rru,equal] \arrow[uuu] &  &  & \tilde{E}^{(p)} \arrow[rru] \arrow[uuu] &  &  \\
			\tilde{T} \arrow[rrr] \arrow[uuu] \arrow[rru, "\, F"'] &  &  & \tilde{E} \arrow[rru, "\, F"'] \arrow[uuu] &  &  &  & 
				\end{tikzcd}
	\end{center}
	Since the composed maps $\tilde{E}\rightarrow \tilde{E}$ on the bottom right, $\tilde{T}\rightarrow \tilde{T}$ on the bottom left and  $\tilde{\mathfrak T}_{q^{1/p}}\rightarrow \tilde{\mathfrak T}_{q}$ on the upper left by construction are the reductions of the respective $p$-multiplication maps $[p]$, the functoriality of the associated bundle construction in Proposition~\ref{associated bundle construction in the semi-linear case is a sort of fibered bifunctor} implies that the two maps on the upper right compose to the reduction of $[\mathfrak p]\times^{[p]}[p]$. But $[\mathfrak p]\times^{[p]}[p]$ is equal to $[\mathfrak p]:\mathfrak E_{q^{1/p}}\xrightarrow{} \mathfrak E_q$ by definition of the latter. 	This completes the proof that the reduction of $[\mathfrak p]:\mathfrak E_{q^{1/p}}\xrightarrow{} \mathfrak E_q$ factors through the relative Frobenius on $\tilde{\mathfrak E}_{q^{1/p}}$.
	
	The conclusion that $E_\infty$ is perfectoid then follows from Proposition~\ref{existence of p-F-formal tower implies perfectoid}.
	\end{proof}
	
	
%%%%%%%%%%%%%%%%%
%%%%%%%%%%%%%%%%%
%%%  Section 6
%%%%%%%%%%%%%%%%%
%%%%%%%%%%%%%%%%%
	
	\section{The case of abelian varieties with semi-stable reduction}
	We now want to prove the main result of this work, building on the preceding chapters:
	
	\begin{theorem}\label{main theorem}
		Let $A$ be an abelian variety over an algebraically closed perfectoid field $K$. Then there is a perfectoid space $A_\infty$ such that
				\[A_\infty \sim \varprojlim_{[p]} A.\]
	\end{theorem}
	
	We have seen this already for abelian varieties with good reduction in Corollary~\ref{tilde-limit exists and is perfectoid in the good reduction case}. It thus suffices to deal with the bad reduction case. Since $K$ is algebraically closed, this means that $A$ has semi-stable reduction.
	
	The proof of Theorem~\ref{main theorem}  in the case of semi-stable reduction will be completed in several steps over the following sections.  We first recall some notation:  Let $g$ be the dimension of $A$. Let $E$ be the Raynaud extension associated to $A$ from Proposition~\ref{Raynaud uniformisation}, which is an extension of a split rigid torus $T$ of rank $r$ by an abelian variety $B$ of good reduction. Moreover there is a lattice $M\subseteq E$ of rank $r$ such that $A=E/M$. Our strategy is to describe the $[p]$-multiplication tower of $E/M$ locally in terms of the $[p]$-multiplication tower of $E$.
	As a first step towards this goal, in the following section, we want to give a cover of $E/M$ by subspaces of $E$ that behaves well under $[p]$-multiplication.
	
	\subsection{Covering $A$ by subspaces of $E$}
	As a first step we recall how to relate the lattice $M$ to a Euclidean lattice in $\mathbb R^r$, cf \S2.7 and \S6.2 in  \cite{rigid geometry of curves}. On the level of points, $\mathbb{G}_m^{\operatorname{an}}$ has an absolute value map
	\[|-|:\mathbb{G}_m^{\operatorname{an}}(K)=K^\times\rightarrow \mathbb R^\times, \quad x\mapsto |x|\]
	which induces the following group homomorphism from the torus $T$:
	\[|-|:T(K)=(K^\times)^r\rightarrow (\mathbb R^\times)^r, \quad (x_1,\dots,x_n)\mapsto (|x_1|,\dots,|x_n|)\]
	Since when working with lattices we prefer additive notation, we also consider the map
	\[\ell:T(K)=(K^\times)^r\rightarrow \mathbb R^r, \quad x_1,\dots,x_n\mapsto (-\log |x_1|,\dots,-\log |x_n|).\]
	Note that this map has dense image due to our assumptions on $K$.
	
		The formal torus $\overline{T}$ corresponds on $K$-points to $\overline{T}_\eta(K) = (\mathcal O_K^\times)^r$ and is thus in the kernel of $\ell$. We can therefore extend $\ell$ to $E(K)$ as follows: Locally over an open subspace $U\subseteq B$ we have $E|_U = T\times^{\overline{T}_\eta}\overline{E}_\eta|_{U}$ and we define $\ell$ by projection from the first factor. The different $E|_U$ are then glued on intersections using the $\overline{T}_\eta$-action on $T$. But since $\ell$ on $T$ is invariant under the $\overline{T}_\eta$-action, the maps glue together to a group homomorphism 
	\[\ell:E(K)\rightarrow \mathbb R^r.\]
	
	Since $A=E/M$ is proper, the lattice $M$ is sent by $\ell$ to a Euclidean lattice $\Lambda \subset \mathbb R^r$ of full rank $r$ (see Proposition 6.1.4 in \cite{rigid geometry of curves}). In particular, this induces an isomorphism of discrete torsionfree groups
	\[\ell:M\xrightarrow{\sim} \Lambda\subseteq\mathbb R^r.\]
	
	The idea is now that one can understand the quotient $E/M$ in terms of the quotient $\mathbb R^r/\Lambda$. We are going to make this precise in the following:
	
	For any $d\in \mathbb R_{> 0}^r$, consider the cuboid with length $2d$ centered at the origin:
	\[S(d) = \{(s_1,\dots,s_r)\in \mathbb R^r | |s_i|\leq d_i \}\]
	Choose $d$ small enough such that $S(d)$ intersects $\Lambda$ only in $0\in \Lambda$. Choose
	 $q_1,\dots,q_r\in K$ such that $|q_i|=\exp(-d_i)$. We denote by $\mathcal B(q,q^{-1})$ the affinoid open rigid multi-annulus in $T$ centered at $1$ of radii $|q_i|< 1 < |q_i|^{-1}$ in every direction.
	
	\begin{lemma}\label{cube around origin gives local chart for E/M}
		The inverse image $\ell^{-1}(S(d))\subseteq E(K)$ is the underlying set of the admissible open subset $E(q):=\mathcal B(q,q^{-1})\times^{\overline{T}_\eta}\overline{E}_\eta$.
	\end{lemma}
	\begin{proof}
		One shows this first for the map $T\rightarrow \mathbb R^r$, where it is clear that the preimage is $\mathcal B(q,q^{-1})$. This is also described in \S 6.4 of \cite{FvdP}. The statement for $\mathcal B(q,q^{-1})\times^{\overline{T}_\eta}\overline{E}_\eta$ follows by direct inspection on local trivialisations $\mathcal B(q,q^{-1})\times U$ for $U\subseteq B$.
	\end{proof}
	
	
	Note that by our choice of $d$, the map $\mathbb R^r\rightarrow \mathbb R^r/\Lambda$ maps $S(d)$ bijectively onto its image. 
	Lemma~\ref{cube around origin gives local chart for E/M} says that we can use $\mathcal B(q,q^{-1})\times^{\overline{T}_\eta}\overline{E}_\eta$ as a chart for $E/M$ around the origin.
	
	In order to obtain charts around other points of $E/M$, we simply need to consider translations: Recall that for every $c\in T(K)$, the translation map
	\[T\xrightarrow{\cdot c}T\]
	is an isomorphism of rigid spaces that sends the unit $1$ to $c$. We denote the image of any admissible open set $U$ under translation by $c\cdot U$.
	
	\begin{lemma}\label{cube around point gives local chart for E/M}
		With notation as before, let $c \in T(K)$ be any point and let $s=l(c)$. Then the inverse image $\ell^{-1}(s+S(d))\subseteq E(K)$ of the translation of $S(d)$ by $s$ is the underlying set of the admissible open subset $E(c,q) := (c\cdot \mathcal B(q,q^{-1}))\times^{\overline{T}_\eta}\overline{E}_\eta \subseteq E$. We can choose $c\in T(K)$ and $q\in T(K)$ in such a way that they admit arbitrary $p^n$-th roots.
	\end{lemma}
	
	\begin{proof}
		Since $\ell$ commutes with the translations
		\begin{center}
			\begin{tikzcd}
				T(K) \arrow[r, "\ell"] \arrow[d, "\cdot c"'] & \mathbb R^r \arrow[d, "+s"] \\
				T(K) \arrow[r, "\ell"] & \mathbb R^r,
			\end{tikzcd}
		\end{center}
		the first part is an immediate consequence of Lemma~\ref{cube around origin gives local chart for E/M}. For the second, note that for any $c'\in T(K)$ with $l(c')=l(c)$ and $q\in T(K)$ with $l(q')=l(q)$ we have $c\cdot \mathcal B(q,q^{-1}) = c'\cdot \mathcal B(q,q^{-1})$, and thus $E(c,q)=E(c',q')$. The statement therefore follows from $K$ being perfectoid, for instance using $\mathcal O_K^\flat = \varprojlim_{x\mapsto x^p}\mathcal O_K$.
	\end{proof}
    
    \begin{definition}\label{defininition of cuboid}
		We call spaces of the form $E(c,q)\subseteq E$ \textbf{cuboids} centered at $c$. More precisely they are locally a cuboid   $c\cdot\mathcal B(q,q^{-1})\subseteq T$ times an admissible open subset of the abelian variety $B$. 
	\end{definition}
    
	\begin{lemma}\label{an admissible cover of A by cuboids of E}
	There exist finitely many admissible open cuboids $E_1,\dots,E_k\subseteq E$ which map isomorphically onto their images in $A=E/M$ and which cover $A$ admissibly. 
	
	One can reconstruct $A$ from any such cover by glueing $E_1,\dots,E_k$ as follows: By construction, for any $E_i$ the translates $m\cdot E_i$ by $m\in M$ are pairwise disjoint and we thus have a canonical projection $\pi$ from the union $\cup_{m\in M} (m\cdot E_i)\subseteq E$ to $E_i$.	Let $E_{ij}:=(\bigcup_{m\in M} m\cdot E_i)\cap E_j \subseteq E$. Then we glue $E_j$ to $E_i$ via the map
	\[E_{ij}\rightarrow \bigcup_{m\in M} m\cdot E_i \xrightarrow{\pi} E_i.\]
	
	\begin{figure}
		\includestandalone[]{glue-cover-tikzpicture}%
		% or use \input{mytikz}
		\caption{Given two charts $E_i$ and $E_j$, the chart $E_j$ is glued to $E_i$ along intersections with all translates of $E_i$ by $q\in M$.}
		\label{glue-cover-tikzpicture}
	\end{figure}
	
	\end{lemma} 
	\begin{proof}
	Since $\mathbb R^r/\Lambda$ is compact, we can find finitely many $s_1,\dots,s_k$ and $d_1,\dots,d_k \in \mathbb{R}^r_{>0}$ such that $\mathbb R^r/\Lambda$ is covered by the $s_i+S(d_i)$. When we choose corresponding $c_1,\dots,c_k \in T(K)$ and $q_1,\dots,q_k$ as in Lemma~\ref{cube around point gives local chart for E/M}, then the corresponding $E_i:=E(c_i,q_i)$ are an atlas of $A=E/M$ by admissible open subsets of $E$.
	
	In order to reconstruct $A$, note that $\cup_{m\in M} m\cdot E_i$ is precisely the preimage of $E_i$ under the projection $E\rightarrow E/M$. In particular, the subspace $E_{ij}$ is precisely the preimage of $E_i$ under the composition $E_j\hookrightarrow E \rightarrow E/M$. In other words, the subspace $E_{ij}\subseteq E_j$ is the intersection of $E_i$ and $E_j$ when considered as subspaces of $A$. This shows that as charts of $A$, the spaces $E_i$ and $E_j$ are glued via $E_{ij}$ as described.
	\end{proof}
	Finally, we need some control about what happens to the cubes under $[p]$-multiplication. Recall from Lemma~\ref{cube around point gives local chart for E/M} that we can always assume that $c$ admits $p$-th roots.
	\begin{lemma}\label{pullback of cuboid is cuboid}
		Let $c^{1/p}$ be a $p$-th root of $c$ and let $q^{1/p}$ be a $p$-th root of $q$ in $(K^\times)^r$. Then under $[p]:E\rightarrow E$, the admissible open $E(c_i,q)$ pulls back to the admissible open $E(c_i^{1/p},q^{1/p})$.
	\end{lemma}
	\begin{proof}
		It is clear that under $[p]:T\rightarrow T$, the admissible open cuboid $c\cdot \mathcal B(q,q^{-1})$ centered at $c$ pulls back to $c^{1/p}\cdot \mathcal B(q^{1/p},q^{-1/p})$. Note that this is independent of the choices of $c^{1/p}$ and $q^{1/p}$. Now recall that in terms of fibre bundles, multiplication $[p]:E\rightarrow E$ is
		\[[p]\times^{[p]}[p]: T\times^{\overline{T}_\eta}\overline{E}_\eta\rightarrow T\times^{\overline{T}_\eta}\overline{E}_\eta \]
		by Lemma~\ref{p-multiplication is induced from Borel construction}. Thus $(c\cdot \mathcal B(q,q^{-1}))\times^{\overline{T}_\eta}\overline{E}_\eta$ pulls back to $(c^{1/p}\cdot \mathcal B(q^{1/p},q^{-1/p}))\times^{\overline{T}_\eta}\overline{E_\eta}$.
	\end{proof}

	\subsection{The two towers}
	In this section we want to separate the $[p]$-multiplication of $A$ into two different towers, which we think of as being a ``ramified'' tower and an ``\'etale'' tower. Of course in characteristic $0$ both towers will actually be \'etale, and these words are only meant to describe the behaviour of the maps relative to the lattice $M$.
	
	For the ramified tower, we first make an auxiliary choice of certain torsion subgroups of $A$: Since $K$ is algebraically closed, we can choose lattices $M^{1/p^n}\subseteq E$ such that $[p]:E\rightarrow E$ restricts to isomorphisms $M^{1/p^{n+1}}\rightarrow M^{1/p^n}$ for all $n$.
	
	\begin{remark}
		Such a choice is equivalent to the choice of subgroups $D_n\subseteq A[p^n]$ of rank $p^{rn}$ for all $n$ such that $pD_{n+1}=D_n$ and $D_n+E[p^n]=A[p^n]$. Namely,
		given the lattices $M^{1/p^{n+1}}$, we obtain torsion subgroups by setting $D_n:=M^{1/p^{n+1}}/M$. This is because any such lattice gives a splitting of the short exact sequence $0\rightarrow E[p^n]\rightarrow A[p^n]\rightarrow M/M^{p^n} \rightarrow 0$.
		
		Conversely, given subgroups $D_n\subseteq A[p^n]$ that form a partial anticanonical $\Gamma_0(p^\infty)$ structure, we recover $M^{1/p^n}$ as the kernel of $E\rightarrow A\rightarrow A/D_n$.
		
		One might call the choice of $D_n$ for all $n$ a partial anticanonical $\Gamma_0(p^\infty)$-structure, because if $B$ admits a canonical subgroup (that is, satisfies a condition on its Hasse invariant), the choice of a (full) anticanonical $\Gamma_0(p^\infty)$-structure on $A$ is equivalent to the choice of a partial anticanonical $\Gamma_0(p^\infty)$-structure on $A$ and an anticanonical $\Gamma_0(p^\infty)$-structure on $B$. Note however that $A$ always has a partial anticanonical subgroup even if $B$ does not have a canonical subgroup.
		
		Note that in the case of $K$ perfectoid but not necessarily algebraically closed, one can still carry out the constructions in the following using partial anticanonical $\Gamma_0(p^\infty)$-structures, whereas the lattices $M^{1/p^n}$ might not be defined over $K$.
	\end{remark}
	
	Following the remark, denote by $D_n$ the torsion subgroup $M^{1/p^n}/M\subseteq A$. The quotient $A/D_n = E/M^{1/p^n}$ is then another abelian variety over $K$ and the quotient map $v^n:E/M\rightarrow E/M^{1/p^n}$ is an isogeny of degree $p^{2gn}$  through which  $[p^n]:A\rightarrow A$ factors: 
		\begin{center}
			\begin{equation}\label{factorisation of [p] over E/M}
			\begin{tikzcd}
				& E/M^{1/p} \arrow[rd, "{[p^n]}"] &  \\
				E/M \arrow[rr, "{[p^n]}"] \arrow[ru,"v^n"] &  & E/M.
			\end{tikzcd}
			\end{equation}
		\end{center}
		We think of these maps as being an analogue of Frobenius and Verschiebung, which is why we denote the left map by $v$.
		Putting everything together, the $[p]$-multiplication tower splits into two towers
		\begin{center}
		\begin{equation}\label{p-multiplication tower of E/M splits into vertical and horizontal tower}
		\begin{tikzcd}[column sep={1.1cm,between origins},row sep={0.7cm,between origins}]
			\ddots \arrow[rd] &  &  & \vdots \arrow[d] &  & \vdots \arrow[d] \\
			& E/M \arrow[rr,"v"] \arrow[rrdd, "{[p]}"'] &  & E/M^{1/p} \arrow[rr,"v"] \arrow[dd,"{[p]}"] &  & E/M^{1/p^2} \arrow[dd,"{[p]}"] \\
			&  &  &  &  &  \\
			&  &  & E/M \arrow[rrdd, "{[p]}"'] \arrow[rr,"v"] &  & E/M^{1/p} \arrow[dd,"{[p]}"] \\
			&  &  &  &  &  \\.
			&  &  &  &  & E/M
		\end{tikzcd}
		\end{equation}
		\end{center}
		Since each quotient $M^{1/p^n}/M$ is a finite \'etale group scheme, all horizontal maps are finite \'etale. The vertical tower on the other hand fits into a second commutative diagram of rigid groups which compares it to the $[p]$-tower of $E$:
		
		\begin{center}
		\begin{equation}\label{F-tower for E/M}
		\begin{tikzcd}
			& \vdots \arrow[d] & \vdots \arrow[d] & \vdots \arrow[d] &  \\
			0 \arrow[r] & M^{1/p^2} \arrow[d, "\cong"] \arrow[r] & E \arrow[d, "{[p]}"] \arrow[r] & E/M^{1/p^2} \arrow[d, "{[p]}"] \arrow[r] & 0 \\
			0 \arrow[r] & M^{1/p} \arrow[d, "\cong"] \arrow[r] & E \arrow[d, "{[p]}"] \arrow[r] & E/M^{1/p} \arrow[d, "{[p]}"] \arrow[r] & 0 \\
			0 \arrow[r] & M \arrow[r] & E \arrow[r] & E/M \arrow[r] & 0
		\end{tikzcd}
		\end{equation}
		\end{center}
		
		\subsection{Constructing a limit of the vertical tower}
		
		Our first step is to show that the tower on the right has a perfectoid tilde-limit.
		Recall from Lemma~\ref{an admissible cover of A by cuboids of E} that $E/M$ can be covered by admissible open subspaces $E_1,\dots,E_k\subseteq E$ which map isomorphically onto an admissible open via $E\to E/M$. Denote by $E_i^{1/p^n}\subseteq E$ the pullback along $[p^n]:E\rightarrow E$. Also denote by $E_{ij}^{1/p^n}\subseteq E$ the pullback of $E_{ij}$. We can then reconstruct the space $E/M^{1/p^n}$ from the $E_i^{1/p^n}$ as follows:
		\begin{lemma}\label{compatible cuboid charts for the tower over E/M}
			\leavevmode
			\begin{enumerate}
		\item The restriction to $E_{i}^{1/p^n}\subseteq E$ of $E\rightarrow E/M^{1/p^n}$ is an isomorphism onto its image. In particular, we can view $E_{i}^{1/p^n}$ as a chart of $E/M^{1/p^n}$, and this is the preimage of $E_i$ under $E/M^{1/p^n}\rightarrow E/M$.
		\item The collection of  $E_{i}^{1/p^n}$ is an atlas for  $E/M^{1/p^n}$. 
		\item We can reconstruct $E/M^{1/p^n}$ from glueing the $E_{i}^{1/p^n}$ along the $E_{ij}^{1/p^n}$.
		\item The map $[p^n]:E/M^{1/p^n}\rightarrow E/M$ can be glued from the restrictions of $[p^n]:E\rightarrow E$ to $E_{i}^{1/p^n}\rightarrow E_{i}$, that is these maps commute with the glueing maps on $E_{ij}^{1/p^n}$.
		\end{enumerate}
		The situation is thus like in Figure~\ref{transform-glue-cover-tikzpicture}.
		\end{lemma}
			\begin{figure}
				\includestandalone{transform-glue-cover-tikzpicture}
				\caption{Illustration of how $[p^n]:E/M^{1/p^n}\rightarrow E/M$ can be glued from the maps $E_j^{1/p^n}\rightarrow E_j$. Here $E_j$ on bottom and $E_j^{1/p^n}$ on top are represented by the grey cuboids in the middle. On the left they are embedded into $E$ whereas on the right they are considered as charts for $E/M$ and $E/M^{1/p}$.}
				\label{transform-glue-cover-tikzpicture}
			\end{figure}
		\begin{proof}
		The first part follows because the map on the left of diagram~\ref{F-tower for E/M} is an isomorphism. The second follows from the pullback of the $E_i$ along $E/M^{1/p^n}\rightarrow E/M$, using that the diagram commutes.
		We thus obtain an admissible cover by cuboids $E_1^{1/p^n},\dots,E_k^{1/p^n}$ of $E/M^{1/p^n}$.
		The second part of Lemma~\ref{an admissible cover of A by cuboids of E} applied to $E/M^{1/p^n}$ then shows that $E/M^{1/p^n}$ can be reconstructed by glueing along subspaces $E_{ij}^{1/p^n}$.
		
		Finally, in order to see that one can glue together the map $[p^n]:E/M^{1/p^n}\rightarrow E/M$ from the $E_i^{1/p^n}$, use that intersection of cuboids are again cuboids, and so $E_{ij}^{1/p^n}$ is a disjoint union of cuboids. It then follows from Lemma~\ref{pullback of cuboid is cuboid} that $E_{ij}$ pulls back under $[p^n]$ to the intersection $E_{ij}^{1/p^n}\subseteq E/M^{1/p^n}$. That $[p^n]$ commutes with the glueing maps is clear because we know from diagram (\ref{factorisation of [p] over E/M}) that $[p^n]:E\rightarrow E$ induces a morphism $[p^n]:E/M^{1/p^n}\rightarrow E/M$.
		\end{proof}
		We are now ready to prove:
		\begin{proposition}\label{explicit construction of vertical tilde-limit}
			There is a perfectoid space $E/M^{1/p^\infty}$ such that\[E/M^{1/p^\infty}\sim \varprojlim_{n}E/M^{1/p^n}. \]
		\end{proposition}
		\begin{proof}
		 Denote by $E_i^{1/p^\infty}$ the pullback of $E_i\subseteq E$ to $E_\infty$. This is an open subspace of a perfectoid space and hence perfectoid. Moreover, by Proposition 2.4.3 in \cite{SW} we have  \[ E_i^{1/p^\infty}\sim \varprojlim E_i^{1/p^n}.\] 
		Given two different $E_i$, $E_j$, we know by Lemma~\ref{compatible cuboid charts for the tower over E/M} that at every step in the tower, the pullbacks $E_i^{1/p^n}$ and $E_j^{1/p^n}$ to $E/M^{1/p^n}$ intersect in  $E_{ij}^{1/p^n}$.
		We can thus glue the $E_i^{1/p^\infty}$ along pullbacks $E_{ij}^{1/p^\infty}$ of the intersections $E_{ij}=E_i\cap E_j$ to $E_\infty$ and thus obtain a perfectoid space $E/M^{1/p^\infty}$. This is a tilde-limit for $\varprojlim_{[p]}E/M^{1/p^n}$ because by construction it is so locally, and the definition of tilde-limits in Definition 2.4.1 of \cite{SW} is local on the source.
		\end{proof}
	 
	\subsection{Constructing a limit of the horizontal tower}
	In order to construct a tilde-limit for $\varprojlim A$, we can now use that the horizontal maps in diagram~(\ref{p-multiplication tower of E/M splits into vertical and horizontal tower}) are all finite \'etale. They are even finite covering maps, in the following sense:
	\begin{lemma}\label{horizontal map is covering map}
		For any $0\leq m\leq n$, the preimage of $E_i^{1/p^n}$ from Lemma~\ref{compatible cuboid charts for the tower over E/M} under the horizontal map $v^{n-m}:E/M^{1/p^{m}}\rightarrow E/M^{1/p^n}$ is isomorphic to $p^{r(n-m)}$ disjoint copies of $E_i^{1/p^n}$. More canonically, it can be described as the isomorphic image of $M^{1/p^n}/M^{1/p^m}\times E_i^{1/p^m}$ under the multiplication map $E/M^{1/p^m}\times E/M^{1/p^m}\rightarrow E/M^{1/p^m}$.
	\end{lemma}
	\begin{proof}
		By the first part of Lemma~\ref{compatible cuboid charts for the tower over E/M}, we know that the preimage of $E_i^{1/p^n}$ under the projection $E\rightarrow E/M^{1/p^n}$ is a disjoint union of translates of $E_i^{1/p^n}$ by $M^{1/p^{n}}$. The result then follows because $M^{1/p^{n}}/M^{1/p^{m}} =  \underline{(\mathbb Z/p^{n-m}\mathbb Z)^r}$.
	\end{proof}
	We also record the following immediate consequence:
	
	\begin{lemma}\label{preimage of E_i under p^n is disjoint copies}
		The preimage of $E_i$ under $[p^n]:A\rightarrow A$ is isomorphic to $p^{rn}$ disjoint copies of $E_i^{1/p^n}$. More canonically, we can describe the preimage as the isomorphic image of  $D_n \times E_i^{1/p^n}$ under the multiplication $A\times A\rightarrow A$. The situation is thus as in figure~\ref{local-glueing-[p]-tikzpicture}.
	\end{lemma}
	\begin{figure}
		\includestandalone{local-glueing-[p]-tikzpicture}
		\caption{Illustration of how $[p]:E/M\rightarrow E/M$ factors in a part that is ``ramified'' (the vertical tower) and a part that is ``\'etale'' (the horizontal tower) with respect to our cover.}
		\label{local-glueing-[p]-tikzpicture}
	\end{figure}
	\begin{proof}
		This follows from the first part of Lemma~\ref{compatible cuboid charts for the tower over E/M} combined with Lemma~\ref{horizontal map is covering map} in the case of $m=0$.
	\end{proof}
	
	
	\begin{lemma}\label{squares in double tower are pullbacks}
		The squares in diagram~(\ref{p-multiplication tower of E/M splits into vertical and horizontal tower}) are all pullback diagrams.
		\begin{center}
			\begin{tikzcd}
				E/M^{1/p^n} \arrow[d, "{[p]}"] \arrow[r,"v"] & E/M^{1/p^{n+1}} \arrow[d, "{[p]}"] \\
				E/M^{1/p^{n-1}} \arrow[r,"v"] & E/M^{1/p^n}
			\end{tikzcd}
		\end{center}
	\end{lemma}
	\begin{proof}
		This can for instance be checked locally: The admissible open subset $E_i^{1/p^n}\subseteq E/M^{1/p^n}$ from Lemma~\ref{compatible cuboid charts for the tower over E/M} is pulled back to $E_i^{1/p^{n+1}}$ under the vertical map $[p]:E/M^{1/p^{n+1}}\rightarrow E/M^{1/p^n}$. The preimage of $E_i^{1/p^n}$ under the horizontal map $E/M^{1/p^{n-1}}\rightarrow E/M^{1/p^n}$ is $p^r$ disjoint copies of $E_i^{1/p^n}$ by Lemma~\ref{horizontal map is covering map}. The pullback of $E_i^{1/p^n}$ to the upper right is thus $p^r$ disjoint copies of $E_i^{1/p^{n+1}}$, which is clearly the fibre product.
	\end{proof}
	\begin{lemma}\label{horizontal etale map pulls back to vertical limit}
	The horizontal maps in diagram~(\ref{p-multiplication tower of E/M splits into vertical and horizontal tower}) induce natural finite \'etale morphisms $v:E/M^{1/p^\infty}\rightarrow E/M^{1/p^\infty}$ that fit into Cartesian diagrams
	\begin{center}
		\begin{tikzcd}
			E/M^{1/p^\infty} \arrow[d, "{}"] \arrow[r,"v^m"] & E/M^{1/p^\infty} \arrow[d, "{}"] \\
			E/M^{1/p^{n-m}} \arrow[r,"v^m"] & E/M^{1/p^{n}}
		\end{tikzcd}
	\end{center}
	In particular, the preimage of $E_i^{1/p^\infty}$ under $v^m$ is isomorphic to $p^{rm}$ copies of $E_i^{1/p^\infty}$.
	\end{lemma}
	\begin{proof}
		Since $E/M \rightarrow E/M^{1/p}$ is finite \'etale, the fibre product with $E/M^{1/p^\infty} \rightarrow E/M^{1/p}$ exists and is perfectoid by Proposition~7.10 of \cite{perfectoid}.
		
		The universal property of the fibre product then gives a unique map 
		\[E/M^{1/p^\infty}\rightarrow E/M\times_ {E/M^{1/p}}E/M^{1/p^\infty}\]
		making the natural diagrams commute.
		On the other hand, using Lemma~\ref{squares in double tower are pullbacks} we see that the fibre product has compatible maps into the vertical inverse system over $E/M$. Since by Proposition~2.4.5 of \cite{SW} the perfectoid tilde-limit $E/M^{1/p^\infty}$ is universal for maps from perfectoid spaces to the inverse system, we obtain a unique map into the other direction.
	\end{proof}
	We thus obtain a pro-\'etale tower
	\begin{equation}\label{proetale tower in the vertical limit}
	\dots \xrightarrow{v}E/M^{1/p^\infty}\xrightarrow{v} E/M^{1/p^\infty}\xrightarrow{v} E/M^{1/p^\infty}
	\end{equation}
	which we think of as being a kind of vertical ``limit'' of diagram~\ref{p-multiplication tower of E/M splits into vertical and horizontal tower}. 
	\begin{remark}\label{projective limits of perfectoid spaces as adic spaces}
	It is clear at this point that the limit of this tower exists, even as a categorical limit of honest adic spaces, since limits of affinoid perfectoid spaces exist in honest adic spaces and are affinoid perfectoid:
	
	If $(\operatorname{Spa}(A_i,A_i^{+}))_{i\in I}$ is an inverse system of affinoid perfectoid spaces, then it is easy to check that for $A^+$ the $\pi$-adic completion of $\varinjlim A_i$ and for $A = A^+[1/\pi]$, the affinoid algebra $(A,A^{+})$ is again perfectoid. It then follows from topological algebra that $\operatorname {Spa}(A,A^{+})$ has the desired universal property for maps from honest adic spaces into the inverse system $(\operatorname{Spa}(A_i,A_i^{+}))_{i\in I}$.
	
	From the construction it follows that this is a tilde-limit (the condition on the underlying sets can be checked using the argument of Lemma 6.13.(ii) in \cite{perfectoid}). The affinoid case implies the result for the above tower by a glueing argument.
	\end{remark}
	
	Despite the remark, we now want to give a slightly more explicit construction of this limit that also says something about its local geometry. We first looks at the tower
	\[\dots \xrightarrow{[p]}D_{n+1}\xrightarrow{[p]}D_n\rightarrow\dots.\]
	\begin{lemma}
		There is a perfectoid space $D_\infty$ such that $D_\infty = \varprojlim_{[p]} D_n$ as a limit of adic spaces.
	\end{lemma}
	\begin{proof}
		The $D_n$ are \'etale over $\operatorname{Spa}(K)$ and thus perfectoid.
		Like in Remark~\ref{projective limits of perfectoid spaces as adic spaces}, we then obtain a perfectoid space $D_\infty$. More explicitly, since the $D_n$ are isomorphic to $\underline {(\mathbb Z_p/p^{n}\mathbb Z)^r}$, we can write \[D_\infty \cong\operatorname{Spa}(\operatorname{Map}_{\operatorname{cts}}(\mathbb Z_p^r,K),\operatorname{Map}_{\operatorname{cts}}(\mathbb Z_p^r,\mathcal O_K)).\]
	\end{proof}
	\begin{proposition}\label{horizontal limit of vertical limit}
	There exists a perfectoid space $(E/M^{1/p^\infty})_\infty$ such that \[(E/M^{1/p^\infty})_\infty= \varprojlim_{v}(E/M^{1/p^\infty}).\]
	 Moreover, the projection map $\pi:(E/M^{1/p^\infty})_\infty\rightarrow E/M^{1/p^\infty}$ is a profinite covering map, that is every point of $E/M^{1/p^\infty}$ has an open neighbourhood $U$ such that $\pi^{-1}(U)$ is isomorphic to $D_\infty \times U$.
	\end{proposition}
	\begin{proof}
	By Lemma~\ref{horizontal etale map pulls back to vertical limit}, the preimage of $E_i^{1/p^\infty}$ under $v^m$ of $E/M^{1/p^\infty}$ is isomorphic to $D_m\times E_i^{1/p^\infty}$. Since projective limits commute with products, the restriction of the tower to $E_i^{1/p^\infty}$ therefore has limit $D_\infty \times E_i^{1/p^\infty}$, which is perfectoid as a product of perfectoid spaces. 
	
	One can then glue using the same arguments as in the proof of Proposition~\ref{explicit construction of vertical tilde-limit}.
	\end{proof}
	
	\subsection{The diagonal tower: proof of the main theorem}
	We now want to show that $(E/M^{1/p^\infty})_\infty$ is in fact a tilde-limit of the $[p]$-multiplication tower. In other words, this says that the horizontal limit of the vertical tilde-limits in diagram~\ref{p-multiplication tower of E/M splits into vertical and horizontal tower} is also a diagonal tilde-limit.
	This isn't just a formal consequence since tilde-limits aren't limits. But using the local geometry of the maps in the tower in terms of cuboids, it is still easy to see:
	
	\begin{proposition}\label{tilde-limit of tilde-limits of partial towers is tilde-limit of whole tower}
		The perfectoid space  $A_\infty:=(E/M^{1/p^\infty})_\infty$ is a tilde-limit of $\varprojlim_{[p]}A$.	It is independent up to unique isomorphism of the choice of partial anticanonical $\Gamma_0(p^\infty)$-structure, but it remembers the choice as a pro-finite \'etale closed subgroup $D_\infty \subseteq A_\infty$. 
		
		The preimage of $E_i\subseteq A$ under the projection $A_\infty \rightarrow A$ is isomorphic to $D_\infty \times E_i^{1/p^\infty}$.
	\end{proposition}
	\begin{proof}
	It is clear from $(E/M^{1/p^\infty})_\infty \sim \varprojlim_v E/M^{1/p^\infty}$ and $E/M^{1/p^\infty}\sim \varprojlim_{[p]} E/M^{1/p^n}$ that the underlying topological space of $(E/M^{1/p^\infty})_\infty$ is indeed isomorphic to $\varprojlim_{[p]}|E/M|$.

	In order to show that it is a tilde-limit of $\varprojlim_{[p]}E/M$, it thus suffices to give an explicit cover of $(E/M^{1/p^\infty})_\infty$ by open affinoids satisfying the tilde-limit property. 
	
	Recall that by construction of $(E/M^{1/p^\infty})$ we have a cover of $E/M$ by open subsets $E_i$ that pull back to perfectoid open subspaces $E_i^{1/p^\infty}$ for which $E_i^{1/p^\infty}\sim \varprojlim E_i^{1/p^n}$.
	 Moreover, by the second part of Proposition~\label{horizontal etale map pulls back to vertical limit} we know that the pullback of $E_i^{1/p^\infty}$ to $(E/M^{1/p^\infty})_\infty$ is $D_\infty \times E_i^{1/p^\infty}$. 
	 
	 On the other hand, when we go along the diagonal tower, we obtain the inverse system 
	 \[\dots\rightarrow D_{n+1}\times E_i^{1/p^{n+1}}\rightarrow D_{n}\times E_i^{1/p^{n}}\rightarrow \dots.\]
	 More explicitly, after choosing compatible isomorphisms $D_n\cong \underline{(\mathbb Z/p^n\mathbb Z)^r}$ for all $n$, we can for any affinoid open $V=\operatorname{Spa}(A_n,A_n^{+})\subseteq E_i^{1/p^{n}}$ write the affinoid open subset $D_n\times V\subseteq D_n\times E_i^{1/p^{n}}$ as
	 \[D_{n}\times V \cong \operatorname{Spa}(\operatorname{Map}(\mathbb (\mathbb Z/p^n\mathbb Z)^r,A_n),\operatorname{Map}((\mathbb Z/p^n\mathbb Z)^r,A_n^{+}))\]
	 
	 We claim that this tower has tilde-limit $D_\infty\times E_i^{1/p^\infty} \sim \varprojlim D_n\times E_i^{1/p^n}$.
	 To see this, cover $E_i^{1/p^\infty}$ by open affinoids $U=\operatorname{Spa}(A,A^{+})$ such that 
	 \[\varinjlim_{\operatorname{Spa}(A_{j},A_{j}^{+})\subseteq E_i^{1/p^n}} A_{j} \rightarrow A \]
	 has dense image, where the direct limit runs through all affinoid open subspaces $\operatorname{Spa}(A_{j},A_{j}^{+})\subseteq E_i^{1/p^n}$ for all $n$ through which $U\subseteq E_i^{1/p^\infty}\rightarrow E_i^{1/p^n}$ factors, as in the definition of tilde-limits. Then by the construction of products of affinoid perfectoid spaces, the pullback of $U$ to $(E/M^{1/p^\infty})_\infty$ is 
	\[D_\infty\times U \cong \operatorname{Spa}(\operatorname{Map}_{cts}(\mathbb Z_p^r,A),\operatorname{Map}_{cts}(\mathbb Z_p^r,A^+))\]
	 It thus suffices to show that $\varinjlim \operatorname{Map}((\mathbb Z/p^n\mathbb Z)^r,A_{j})$ is dense in $\operatorname{Map}_{cts}(\mathbb Z_p^r,A)$.
	 As a first step, even though in the above limit $n$ depends on $j$, we may write this as two separate limits, 
	 \[\varinjlim \operatorname{Map}((\mathbb Z/p^n\mathbb Z)^r,A_{j}) = \varinjlim_{j} \varinjlim_n \operatorname{Map}(\mathbb (\mathbb Z/p^n\mathbb Z)^r,A_{j}).\]
	 For fixed $j$, we then see that 
	 \[\varinjlim_{n} \operatorname{Map}((\mathbb Z/p^n\mathbb Z)^r,A_j)=\operatorname{Map}_{lc}(\mathbb Z_p^r,A_j)\] where the right hand side denotes locally constant morphisms. But it then follows from a pointwise approximation argument that $\varinjlim\operatorname{Map}_{lc}(\mathbb Z_p^r,A_j)$ has dense image in $\operatorname{Map}_{lc}(\mathbb Z_p^r,A)$. It is then clear that the latter has dense image in $\operatorname{Map}_{cts}(\mathbb Z_p^r,A)$. We conclude that $D_\infty\times E_i^{1/p^\infty} \sim \varprojlim D_n\times E_i^{1/p^n}$.
	 
	That $A_\infty$ is independent of the $\Gamma_0(p^\infty)$-structure up to unique isomorphism is a consequence of the universal property of the perfectoid tilde-limit. To see that $D_\infty$ is a closed subgroup of $A_\infty$, choose $i$ such that the unit section of $E/M$ lies in $E_i$. Then the unit section $\operatorname{Spa}(K,\mathcal O_K)\rightarrow E_i^{1/p^\infty}$ induces a closed immersion $D_\infty\hookrightarrow D_\infty \times E_i^{1/p^\infty}\hookrightarrow A_\infty$.
	\end{proof}
	
	This finishes the proof of Theorem~\ref{main theorem}.
	
	Note that while the approach via cuboids $E_i$ may look a bit technical on first glance, it has the advantage of giving an explicit description of $(E/M)_\infty$ as being glued from pieces of $E_\infty$ by glueing data that is controlled by the lattices $M^{1/p^n}$. This might be interesting for applications, and in particular for computing the tilt. 
	
	
	\section{Limits of the covering maps}
	In this section we use the explicit constructions of the space $A_\infty$ to study its geometry more closely. We retain notation and assumptions from the last chapter.
	
	Over the course of the proof of Theorem~\ref{main theorem}, we have used three different towers: The tower $\dots \rightarrow E\xrightarrow{[p]} E$, the tower $\dots \rightarrow E/M \xrightarrow{[p]} E/M$ and the tower $\dots \rightarrow E/M^{1/p} \xrightarrow{[p]} E/M$. The three are related by covering maps which fit into a commutative diagram of towers
	\begin{center}
		\begin{tikzcd}
			E \arrow[d, "{[p]}"] \arrow[r] & E/M \arrow[d, "{[p]}"] \arrow[r] & E/M^{1/p^{n+1}} \arrow[d, "{[p]}"] \\
			E \arrow[r] & E/M \arrow[r] & E/M^{1/p^n}
		\end{tikzcd}
	\end{center}
	As we have seen in the last sections, all three towers have perfectoid tilde-limits, that we have denoted by $E_\infty$, $A_\infty$ and $E/M^{1/p^\infty}$.
	
	By Proposition~\ref{perfectoid tilde-limit is perfectoid group in a functorial way} the map $\pi:E\rightarrow A=E/M$ in the limit induces a natural group homomorphism $\iota:E_\infty \rightarrow A_\infty$. A similar universal property argument shows that we obtain a natural group homomorphism $A_\infty \rightarrow E/M^{1/p^\infty}$. The composition of these two maps is the morphism $E_\infty\rightarrow E/M^{1/p^\infty}$, which is the limit of the maps $E\rightarrow E/M^{1/p^n}$ in the above diagram.
	We now want to look at these morphisms more closely one after the other.
	
	We start with the case of $E_\infty\rightarrow E/M^{1/p^\infty}$:
	\begin{proposition}\label{the morphism E->E/M^{1/p^n} in the limit}
		Denote by $M_\infty\cong M$ the perfectoid tilde-limit of the tower
		\begin{center}
			\begin{tikzcd}[column sep =0.7cm]
				%this is a tikzcd diagram because that's an easy way to have a tilde over and a [p] under the arrow.
				\dots \arrow[r, "{[p]}"',"\sim"] & M^{1/p^2} \arrow[r, "{[p]}"',"\sim"] & M^{1/p} \arrow[r, "{[p]}"',"\sim"] & M.
			\end{tikzcd}
		\end{center}
		There is a natural map $M_\infty \rightarrow E_\infty$ with respect to which we can interpret $M_\infty$ as a lattice of rank $r$ in $E_\infty$. The map fits into a short exact sequence of perfectoid groups
		\[0\rightarrow M_\infty\rightarrow E_\infty \rightarrow E/M^{1/p^\infty} \rightarrow 0.\]
		that is locally split. In particular, we can view $E_\infty$ as an $M_\infty$-torsor over  $E/M^{1/p^\infty}$.
	\end{proposition}
	\begin{proof}
		The map $M_\infty\rightarrow E_\infty$ is induced by the universal property of the perfectoid tilde-limit as usual. 
		In order to see that the sequence is exact, we need to see that the first map is a kernel of the second, and the second map is a categorical quotient of the first. To this end, we first analyse the morphism locally: The projections to the inverse system fit into a commutative diagram
		\begin{center}
			\begin{tikzcd}[row sep = {0.75cm,between origins}]
				M_\infty \arrow[r] \arrow[d,no head] & E_\infty \arrow[r] \arrow[d,no head] & E/M^{1/p^\infty} \arrow[d,no head] \\
				\vdots \arrow[d] & \vdots \arrow[d] & \vdots \arrow[d] \\
				M^{1/p^n} \arrow[dd, "{[p^n]}"] \arrow[r] & E \arrow[dd, "{[p^n]}"] \arrow[r] & E/M^{1/p^n} \arrow[dd, "{[p^n]}"] \\
				&  &  \\
				M \arrow[r] & E \arrow[r] & E/M
			\end{tikzcd}
		\end{center}
		
		Let us consider the preimages of $E_i \subseteq E/M$ under these morphisms: By Lemma~\ref{an admissible cover of A by cuboids of E} we see that the pullback to $E$ is $\bigsqcup_{q\in M} qE_i$. We can also see this as the isomorphic image of $M\times E_i$ under the multiplication map $E\times E\rightarrow E$. 
		The pullback of $E_i$ along $[p^n]:E/M^{1/p^n}\rightarrow E/M$ is $E_i^{1/p^n}$ as we have seen in Lemma~\ref{compatible cuboid charts for the tower over E/M}. The same Lemma shows that the pullback of this along $E\rightarrow E/M^{1/p^n}$ is $\bigsqcup_{q\in M^{1/p^n}}qE_i^{1/p^n}=M^{1/p^n}\times E_i^{1/p^n}$.
		We conclude that the pullback to $E_\infty$ is $M_\infty\times E_i^{1/p^\infty}$. 
		By construction of $E/M^{1/p^\infty}$ in the proof of Proposition~\ref{explicit construction of vertical tilde-limit}, the pullback of $E_i$ to $E/M^{1/p^\infty}$ is $E_i^{1/p^\infty}$. All in all, we obtain a pullback diagram
		\begin{center}
		\begin{tikzcd}[column sep={1.5cm,between origins},row sep={0.8cm,between origins}]
			& E_\infty \arrow[dd] \arrow[rr] &  & E/M^{1/p^\infty} \arrow[dd] \\
			M_\infty\times E_i^{1/p^\infty} \arrow[dd] \arrow[rr] \arrow[ru, hook] &  & E_i^{1/p^\infty} \arrow[dd] \arrow[ru, hook] &  \\
			& E \arrow[rr] &  & E/M \\
			M\times E_i \arrow[rr] \arrow[ru, hook] &  & E_i \arrow[ru, hook] & 
		\end{tikzcd}
		\end{center}
		We conclude that $E_\infty \rightarrow E/M^{1/p^\infty}$ is a principal $M_\infty$-torsor of perfectoid groups. It is then clear that $M_\infty$ is the preimage of $0\in E/M^{1/p^\infty}$, from which one easily verifies that $M_\infty\hookrightarrow E_\infty$ has the universal property of the kernel.
		It remains to see that $E_\infty \rightarrow E/M^{1/p^\infty}$ has the universal property of the cokernel: Given any perfectoid group $H$ and a group homomorphism $E_\infty\rightarrow H$ that is trivial on $M_\infty$, the restriction $M_\infty\times E_i^{1/p^\infty}\rightarrow H$ gives a natural map $E_i^{1/p^\infty}\rightarrow H$. Since by construction of $E/M^{1/p^\infty}$ the spaces $E_i^{1/p^\infty}$ and $E_j^{1/p^\infty}$ are glued on $E_{ij}^{1/p^\infty}$ using translation by $M_\infty$, these glue together to the desired morphism of $E/M^{1/p^\infty}$.
	\end{proof}
	
	
	The case of $\iota:A_\infty \rightarrow E/M^{1/p^\infty}$ is similar:
	\begin{proposition}\label{the morphism A->E/M^{1/p^n} in the limit}
		The subgroup $D_\infty \subseteq A_\infty$ gives rise to a short exact sequence of perfectoid groups
		\[0\rightarrow D_\infty \rightarrow A_\infty\rightarrow E/M^{1/p^\infty}\rightarrow 0.\]
		that is locally split. In particular, we can view $A_\infty$ as a $D_\infty$-torsor over  $E/M^{1/p^\infty}$.
	\end{proposition}
	\begin{proof}
		By Proposition~\ref{horizontal limit of vertical limit} the pullback of $E_i^{1/p^\infty}$ under $A_\infty \rightarrow E/M^{1/p^\infty}$ is
		\[D_\infty \times E_i^{1/p^\infty}\rightarrow E_i^{1/p^\infty} \]
		which shows that $A_\infty\rightarrow E/M^{1/p^\infty}$ is a $D_\infty$-torsor. As in the last proof, this implies that the sequence in the Proposition is a short exact sequence.
	\end{proof}
	
	Finally, we consider the case of $\iota:E\rightarrow A=E/M$. While the limits of the last two towers were fibre bundles again, the map $\iota$ shows quite a different behaviour and on the opposite is an injective group homomorphism. This may seem strange at first, but it is actually what one might expect following the intuition of the following example:
	\begin{remark}
		Consider the following inverse system of abstract groups:
	\begin{center}
	\begin{tikzcd}[row sep = {0.55cm,between origins}]
		& \arrow[dd,dotted] & \arrow[dd,dotted] & \arrow[dd,dotted] &  \\
		&&\\
		0 \arrow[r] & \mathbb Z \arrow[r] \arrow[dd, "{[p]}"] & \mathbb R \arrow[r] \arrow[dd, "{[p]}"] & \mathbb R/\mathbb Z \arrow[dd, "{[p]}"] \arrow[r] & 0 \\
		&&\\
		0 \arrow[r] & \mathbb Z \arrow[r] & \mathbb R \arrow[r] & \mathbb R/\mathbb Z \arrow[r] & 0
	\end{tikzcd}
	\end{center}
	While at finite level the maps on the right are all covering maps, in the inverse limit the homological algebra of $\varprojlim$ produces a long exact sequence
	\begin{center}
	\begin{tikzcd}
		0 \arrow[r] & 0 \arrow[r] & \mathbb R \arrow[r] & \varprojlim_{[p]}\mathbb R/\mathbb Z \arrow[r] & \varprojlim^1_{[p]}\mathbb Z = \mathbb Z_p/\mathbb Z\arrow[r] & 0.
	\end{tikzcd}
	\end{center}
	So in the limit the covering map becomes the kernel of a map to $\mathbb Z_p/\mathbb Z$.
	\end{remark}
	
	For perfectoid groups the homological algebra argument of course doesn't apply. Nevertheless, we can again use the explicit covers of the last section to show that the situation is very similar as in the remark. In the following, we use the notion of injective morphism from \cite{etale_cohomology_of_diamonds}, Definition 5.1.
	\begin{theorem}\label{the morphism E->A in the limit}
		The map $\iota:E_\infty \rightarrow A_\infty$ is an injective group homomorphism. It fits into the following commutative diagram of locally split short exact sequences of perfectoid groups:
		\begin{center}
			\begin{tikzcd}
				0 \arrow[r] & M_{\infty} \arrow[r] \arrow[d, hook] & E_\infty \arrow[d, hook] \arrow[r] & E/M^{1/p^\infty} \arrow[d,equal] \arrow[r] & 0 \\
				0 \arrow[r] & D_\infty \arrow[r] & A_\infty \arrow[r] & E/M^{1/p^\infty} \arrow[r] & 0
			\end{tikzcd}
		\end{center}
		The morphism $\iota$ is compatible with the splittings: Locally on open subspaces $U\subseteq E/M^{1/p^\infty}$ the morphism $E_\infty\hookrightarrow A_\infty$ is of the form $M_\infty\times U\rightarrow D_\infty \times U$.	In particular, one can describe $A_\infty$ as the associated fibre bundle
		\[A_\infty = D_\infty\times^{M_\infty}E_\infty.\]
	\end{theorem}
	\begin{proof}
		Recall that the map $\iota:E_\infty \rightarrow A_\infty$ arises by a universal property from the inverse system
		\begin{center}
			\begin{tikzcd}[row sep = {0.65cm,between origins}, column sep = {2cm,between origins}]
				\vdots \arrow[d] & \vdots \arrow[d] & \vdots \arrow[d] \\
				E \arrow[dd, "{[p]}"'] \arrow[r] & A \arrow[dd, "{[p]}"'] \arrow[r, "v"] & E/M^{1/p} \arrow[dd, "f"'] \\
				&  &  \\
				E \arrow[r] & A \arrow[r,equal] & E/M.
			\end{tikzcd}
		\end{center}
		Using Lemmas~\ref{compatible cuboid charts for the tower over E/M} and \ref{horizontal map is covering map} we see that the pullback of this diagram to $E_i\subseteq E/M$ is
		\begin{center}
			\begin{tikzcd}[row sep = {0.75cm,between origins}, column sep = {3cm,between origins}]
				\vdots \arrow[d] & \vdots \arrow[d] & \vdots \arrow[d] \\
				\bigsqcup\limits_{q\in M^{1/p}} q\cdot E_i^{1/p^n} \arrow[dd] \arrow[r] & \bigsqcup\limits_{q\in D_n} E_i^{1/p^n} \arrow[dd] \arrow[r] & E_i^{1/p^n} \arrow[dd] \\
				&  &  \\
				\bigsqcup\limits_{q\in M} q\cdot E_i \arrow[r] & E_i \arrow[r,equal] & E_i
			\end{tikzcd}
		\end{center}
	We see from this description and from the last part of Proposition~\ref{tilde-limit of tilde-limits of partial towers is tilde-limit of whole tower} that the pullback to infinite level is the sequence 
	\begin{equation}\label{local description of E_infty to A_infty to E/M_infty}
	M_\infty \times E_i^{1/p^\infty}\rightarrow D_\infty \times E_i^{1/p^\infty}\rightarrow E_i^{1/p^\infty}.
	\end{equation}
	This shows that the diagram of short exact sequences commutes. Since the $E_i^{1/p^\infty}$ cover $E/M^{1/p^\infty}$ by construction, it also shows the description of $A_\infty$ in terms of the associated fibre bundle.
	 
	It remains to prove that $\iota$ is injective: Choose a basis of $M$, and thus a trivialisation of all $D_n$. We then see that the map $M_\infty\rightarrow D_\infty$ on the level of the underlying topological spaces can be described as the inclusion $\mathbb Z^r\hookrightarrow \mathbb Z_p^r$. This shows that $M_\infty\hookrightarrow D_\infty$ is an injective group homomorphism of perfectoid spaces. Thus by equation~(\ref{local description of E_infty to A_infty to E/M_infty}), the morphism $E_\infty\rightarrow A_\infty$ is injective as well.
	\end{proof}
	
	\begin{corollary}\label{A_infty as a quotient of D_infty times E_infty}
		The injection $E_\infty \rightarrow A_\infty$ induces a short exact sequence of perfectoid groups
		\[0\rightarrow M_\infty\rightarrow D_\infty \times E_\infty \rightarrow A_\infty\rightarrow 0\]
		where the map on the left is the diagonal embedding of $M_\infty$ into $D_\infty\times E_\infty$. In particular, we can describe $A_\infty$ as the quotient $(D_\infty\times E_\infty)/M_\infty$ in the category of perfectoid groups.
	\end{corollary}
	\begin{proof}
		The map $D_\infty\times E_\infty \rightarrow A_\infty$ is just the composition of $D_\infty\times E_\infty \hookrightarrow A_\infty\times A_\infty$ with the multiplication of $A_\infty$. It is then clear from the short exact sequences of Theorem~\ref{the morphism E->A in the limit} that $M_\infty$ is the kernel of this map. That $A_\infty$ has the universal property of the cokernel is a consequence of the universal property of the associated fibre bundle construction: Explicitly, this follows from the fact that locally over $U\subseteq E/M^{1/p^\infty}$, the map on the right is the projection
		\[D_\infty \times M_\infty \times U \rightarrow D_\infty\times U.\]
		This gives a local splitting, and thus the necessary map in the universal property of the cokernel.
	\end{proof}
	We can see the short exact sequence of \ref{A_infty as a quotient of D_infty times E_infty} as the analogue at infinity of the short exact sequence 	of rigid groups
		\[0\rightarrow M\rightarrow E\rightarrow A\rightarrow 0.\]
	In particular, while as a rigid analytic space $A$ is locally isomorphic to open subspaces of $E$, the perfectoid space $A_\infty$ is locally isomorphic to open subspaces of $D_\infty \times E_\infty$.
	
	\appendix
	\section{Fibre bundles of formal and rigid spaces}
	In this chapter we review the theory of fibre bundles with structure group $T$ and in particular of principal $T$-bundles in the setting of formal and rigid geometry.
		
	
	
	In this chapter we denote by $T$ a commutative formal group scheme over $\mathcal O_K$. We denote the multiplication map by $m:T\times T\rightarrow T$. By a $T$-action on a formal scheme $X$ we mean a morphism $m_X:T\times X\rightarrow X$ such that the usual associativity diagram commutes. 
	\begin{definition}
		By a \textbf{$T$-linear map} of schemes $X$ and $Y$ with $T$-actions we mean a morphism $\phi:X\rightarrow Y$ such that the following diagram commutes
		\begin{center}
			\begin{tikzcd}
				T\times X \arrow[d, "m_X"] \arrow[r, "\operatorname{id}_T\times \phi"] & T\times Y \arrow[d, "m_Y"] \\
				X \arrow[r, "\phi"] & Y
			\end{tikzcd}
		\end{center}
		We denote by $\mathbf{FormAct}_T$ the category of formal schemes with action by $T$.
	\end{definition}
	
	
	The definition of a principal $T$-bundle is just what we get when we take the definition of a principal $G$-bundle and replace the category of topological spaces by the category of formal schemes.
	\begin{notation}
		In the following, if $\pi:E\rightarrow B$ is a morphism of formal schemes, then for a formal open subscheme $U\subseteq B$ we denote $E|_U:=\pi^{-1}(U)\subseteq E$.
	\end{notation}
	\begin{definition}\label{definition principal T-bundle}
		Let $T$ be a formal group scheme. Let $F$ be a formal scheme with an action $m:T\times F\rightarrow F$.
		A morphism $\pi:E\rightarrow B$ of formal schemes is called a \textbf{fibre bundle with fibre $F$ and structure group $T$} if there is a cover $\mathfrak U$ of $B$ of open formal subschemes $U_i\subseteq B$ with isomorphisms $\varphi_i:F\times U_i \xrightarrow{\sim} E|_{U_i}$ which satisfy the following conditions:
		\begin{enumerate}[label=(\alph*)]
			\item For every $U_i\in \mathfrak U$, the following diagram commutes:
			\begin{center}
				\begin{tikzcd}
					F\times U_{i} \arrow[r, "\varphi_i"] \arrow[rd, "p_2"] & E|_{U_{i}} \arrow[d, "\pi"] & \phantom{T\times U_{ij}} \\
					& U_{i} & 
				\end{tikzcd}
			\end{center}
			\item For every two $U_i,U_j\in \mathfrak U$ with intersection $U_{ij}$, the commutative diagram
			\begin{center}
				\begin{tikzcd}
					F\times U_{ij} \arrow[r, "\varphi_i"] \arrow[rd, "p_2"] & E|_{U_{ij}} \arrow[d, "\pi"] & F\times U_{ij} \arrow[ld, "p_2"] \arrow[l, "\varphi_j"'] \\
					& U_{ij} & 
				\end{tikzcd}
			\end{center}
			produces an isomorphism $\phi_{ij}:=\varphi_j^{-1}\circ\varphi_i: F\times U_{ij}\rightarrow F\times U_{ij}$ with the following property: There exists a morphism $\psi_{ij}:U_{ij}\rightarrow T$ such that
			\[\phi_{ij}=F\times U_{ij} \xrightarrow{\psi_{ij}\times \operatorname{id}\times\operatorname{id}} T\times F\times U_{ij}\xrightarrow{m\times \operatorname{id}} F\times U_{ij}\]
		\end{enumerate}
	\end{definition}
	\begin{definition}
		When we take $F$ equal to the formal scheme $T$ with the action on itself by left multiplication, then a fibre bundle $\pi:E\rightarrow B$ with fibre $T$ and structure group $T$ is called a \textbf{principal $T$-bundle}. This is also called a $T$-torsor.
	\end{definition}
	
	\begin{example}
		For the short exact sequence $0\rightarrow \overline{T}\rightarrow \overline{E}\xrightarrow{\pi} \overline{B}\rightarrow 0$ from Section~\ref{Raynaud extensions as principal bundles of formal and rigid spaces}, $\overline{E}\xrightarrow{\pi} \overline{B}$ defines a principal $\overline{T}$-bundle by Lemma~\ref{formal Raynaud sequence is locally split}. Moreover, for any formal open subscheme $U\subseteq \overline{B}$, the map $E|_U\rightarrow U$ is still a principal $\overline{T}$-bundle. This is what we mean when we say that the notion of principal $\overline{T}$-bundles is better suited for studying $\overline{E}$ locally on $\overline{B}$ than the notion of short exact sequences.
	\end{example}
	
	The morphism $\phi_{ij}$ from condition (b) is fully determined by the morphism $\psi_{ij}:U_{ij}\rightarrow T$. By a glueing argument, one shows:
	\begin{lemma}\label{equivalent characterisation of principal $T$-bundle}
		Suppose we are given formal schemes $F$ and $B$ and a formal group scheme $T$ with an action on $F$. Then fibre bundles $\pi:E\rightarrow B$ with fibre $F$ and structure group $T$ are equivalent to the data (up to refinement) of a cover $\mathfrak U$ of $B$ by formal open subschemes and morphisms $\psi_{ij}:U_{ij}\rightarrow T$ for all $U_i,U_j\in \mathfrak U$ that satisfy the cocycle condition $\psi_{jk}\cdot \psi_{ij}=\psi_{ik}$, by which we mean that the following diagram commutes:
		
		\begin{center}\begin{equation}		\label{cocycle condition of fibre bundle}
			\begin{tikzcd}
			U_{ijk} \arrow[r, "\psi_{ij}\times\psi_{jk}"] \arrow[d,equal] & T\times T \arrow[d, "m"] \\
			U_{ijk} \arrow[r, "\psi_{ik}"] & T.
			\end{tikzcd}
			\end{equation}
		\end{center}
	\end{lemma}
	
	In order to define the category of  fibre bundles, we also need the following:
	\begin{lemma}
		Let $E\rightarrow B$ be a fibre bundle with fibre $F$ and structure group $T$. With notations like in Definition~\ref{definition principal T-bundle} we have a natural $T$-action on $F\times U_{i}$ for each $i$ when we let $T$ act trivially on $U_{i}$. These actions glue together to a $T$-action on $E$.
	\end{lemma}
	\begin{proof}
		This is immediate from condition (b).
	\end{proof}
	\begin{definition}
		Let $\pi:E\rightarrow B$ be a fibre bundle with fibre $F$ and structure group $T$ and let $\pi':E'\rightarrow B'$ be a fibre bundle with fibre $F'$ and structure group $T$. Then a \textbf{morphism of fibre bundles} $f:(E',B',\pi')\rightarrow (E,B,\pi)$ is a commutative diagram of formal schemes
		\begin{center}
			\begin{tikzcd}
				E' \arrow[d] \arrow[d, "f_E"] \arrow[r, "\pi'"] & B' \arrow[d, "f_B"] \\
				E \arrow[r, "\pi"] & B
			\end{tikzcd}
		\end{center}
		in which the morphism $f_E$ is also $T$-linear (we often abbreviate this by writing $f:E'\rightarrow E$). We thus obtain the category of fibre bundles over $T$ that we denote by $\mathbf{FormFibBun}_T$ and the full subcategory of principal $T$-bundles, that we denote by $\mathbf{FormPrinBun}_T$.
	\end{definition}
	
	In the case of principal $T$-bundles, this data can be given as follows: Let $\mathfrak U$ be of $B$ a cover over which $E$ is trivialised. Then we can always refine $U$ in such a way that for all $U\in \mathfrak{U}$ the fibre bundle $E'$ is trivial over $U':=f_B^{-1}(U)$. The induced map $f_E:T\times U'\rightarrow T\times U$ is then $T$-linear and thus can be reconstructed from the induced map
	\[\theta:U'=1\times U'\hookrightarrow T\times U'\xrightarrow{f_E} T\times U\xrightarrow{p_1} T.\]
	
	\begin{lemma}\label{equivalent characterisation of morphisms of principal T-bundles}
		Fix a morphism $f_B:B'\rightarrow B$ and fix notation as above. Then the data of a morphism $f=(f_E,f_B)$ of principal $T$-bundles is equivalent to the data of morphisms $\theta_i: U'_i=f_B^{-1}(U_i)\rightarrow T$ 
		for some cover $\mathfrak{U}$ of $B$ by formal open subschemes $U_i$ %it might be a bit misleading to say that the cover is part of the data, since there's this whole thing about categories of "trivialised fibre bundles" where one actually adds such covers to the data, which is different to what we do, but that's what I might think of when I hear "the data of a cover + ..."
		such that each $\theta_i$ induces a map $f_{E,i}$ such that for all $i,j$ the following diagram commutes:
		\begin{center}
			\begin{tikzcd}
				T\times U'_{ij} \arrow[d, "f_{E,i}"] \arrow[r, "\phi'_{ij}"] & T\times U'_{ij} \arrow[d, "f_{E,j}"] \\
				T\times U_{ij} \arrow[r, "\phi_{ij}"] & T\times U_{ij}.
			\end{tikzcd}
		\end{center}
		Moreover, commutativity of the above diagram is equivalent to commutativity of 
		\begin{center}
			\begin{tikzcd}
				U'_{ij} \arrow[r, "\psi'_{ij}\times\theta_j"] \arrow[d, "(\psi_{ij}\circ f)\times\theta_i"'] & T\times T \arrow[d] \arrow[d, "m"] \\
				T\times T \arrow[r] \arrow[r, "m"] & T.
			\end{tikzcd}
		\end{center}
	\end{lemma}
	Or in short hand notation,
	\[\psi'_{ij}(u)\theta_j(u)=\psi_{ij}(f(u))\cdot \theta_i(u)\]
	\begin{proof}
		One direction is clear. For the other, the first part follows from glueing. The second part is a consequence of all maps in the first diagram being $T$-linear.
	\end{proof}
	
	\begin{definition}\label{definition of Borel construction}
		Let $\pi:E\rightarrow B$ be a principial $T$-bundle. Let $F$ be a formal scheme with an action by $T$. Since the data in the equivalent characterisation of Lemma~\ref{equivalent characterisation of principal $T$-bundle} is completely independent of the fibre, the morphisms $\psi_{ij}:U_{ij}\rightarrow T$ by Lemma~\ref{equivalent characterisation of principal $T$-bundle} define a fibre bundle with fibre $F$ and structure group $T$ that we denote by $F\times^T E$. This is called the \textbf{associated bundle} or Borel-Weil construction.
	\end{definition}
	
	Note that in many authors in differential geometry and topology denote the associated bundle by "$F\times_T E$" instead of $F\times^T E$. In our setting, however, this is slightly confusing since we often have natural maps from $T$ to $F$ and $E$, but $F\times^T E$ is usually \textit{not} their fibre product. In fact it behaves more like a pushout, for instance in the case that $E$ comes from a short exact sequence.
	
	\begin{proposition}\label{associated bundle construction is bifunctor}
		The associated bundle construction is a bifunctor \[-\times^T-:\mathbf{FormAct}_T\times \mathbf{FormPrinBun}_T\rightarrow \mathbf{FormFibBun}_T\] 
		from the categories of formal schemes with $T$-action $\times$ the category of principal $T$-bundles to the category of fibre bundles with structure group $T$.
	\end{proposition}
	\begin{proof}
		Let $E$ and $E'$ be principal $T$-bundles and let $f:E'\rightarrow E$ be a morphism of $T$-bundles. Let $F$ and $F'$ be formal schemes with $T$-action and let $h:F'\rightarrow F$ be a $T$-equivariant morphism. Then we can find compatible covers $\mathfrak U'$ of $B'$ and $\mathfrak U$ of $B$ such that locally we have diagrams like in Lemma~\ref{equivalent characterisation of morphisms of principal T-bundles}. Then locally $F\times^T E$ and $F'\times^T E'$ are of the form $F\times U_i$ and $F'\times U'_i$ such that we obtain a natural map
		\[F'\times U'_i\xrightarrow{h\times^T f} F\times U_i, \quad (x,u)\mapsto (h(x)\theta_i(u),f_B(u))\]
		(of course this description is just a short hand for a diagram of maps, and not a description in terms of "points"). These maps glue together over the cover, since on intersections Lemma~\ref{equivalent characterisation of morphisms of principal T-bundles} implies that we have a commutative diagram
		\begin{center}
			\begin{tikzcd}
				F'\times U'_{ij} \arrow[r, "h\times^T f"] & F\times U_{ij} \\
				F'\times U'_{ij} \arrow[u, "\psi'_{ij}\times \operatorname{id}"] \arrow[r, "h\times^T f"] & F\times U_{ij} \arrow[u, "\psi_{ij}\times \operatorname{id}"'].
			\end{tikzcd}
		\end{center}
		One easily checks that this is functorial in both components.
	\end{proof}
	
	\begin{lemma}
		Let $S$ be another formal group scheme that receives an action of $T$ from a group homomorphism $g:T\rightarrow S$. Then for any principal $T$-bundle $E$, the Borel construction $S\times^T E$ is a principal $S$-bundle.
	\end{lemma}
	\begin{proof}
		This follows from Lemma~\ref{equivalent characterisation of principal $T$-bundle}. The only thing we need to check is that the cocycle condition from diagram (\ref{cocycle condition of fibre bundle}) also holds with respect to $S$. But $g$ is a homomorphism and therefore the following diagram commutes:
		\begin{center}
			\begin{tikzcd}
				T\times T \arrow[d, "m"] \arrow[r, "g\times g"] & S\times S \arrow[d, "m"] \\
				T \arrow[r, "g"] & S.
			\end{tikzcd}
		\end{center}
		
		
	\end{proof}
	
	
	\begin{lemma}\label{change of fibre is functorial}
		The Borel construction is a functor $S\times^T -$ from principal $T$-bundles to principal $S$-bundles.
	\end{lemma}
	\begin{proof}
		This is a consequence of Lemma~\ref{equivalent characterisation of morphisms of principal T-bundles}. One obtains the necessary data by composing the morphisms $\theta':U_i'\rightarrow T$ with the morphism $T\rightarrow S$. These morphisms glue together because the second diagram of Lemma~\ref{equivalent characterisation of morphisms of principal T-bundles} commutes, as one easily sees from the fact that $T\rightarrow S$ is a morphism of formal groups. 
	\end{proof}
	
	The Borel construction satisfies the following universal property:
	\begin{lemma}\label{universal property of associated fibre construction for principal bundles}
		Let $g:T\rightarrow S$ be a homomorphism of formal group schemes and let $E\rightarrow B$ be a principal $T$-bundle. Let $X$ be any principal $S$-bundle. Note that $X$ receives a $T$-action from $g$. Then there is a functorial one-to-one correspondence between $T$-linear morphisms $E\rightarrow X$ and morphisms of principal $S$-bundles $S\times^T E\rightarrow X$.
	\end{lemma}
	
	\subsection{The semi-linear case}
	We later want to consider morphisms of fibre bundles that are induced from morphisms of short exact sequences. In this situation, in order to describe the morphism of the kernels, we need to incorporate morphisms of the structure group into the notion of morphisms of fibre bundles. For this we need semi-linear group actions.
	\begin{definition}
		Let $T$ and $S$ be formal group schemes and let $g:T\rightarrow S$ be a homomorphism. Let $X$ and $Y$ be formal schemes with actions $m:T\times X\rightarrow X$ and $m:S\times Y\rightarrow Y$ respectively. Then by a $g$-linear morphism $f:X\rightarrow Y$ we mean a morphism of formal schemes such that the following diagram commutes
		\begin{center}
			\begin{tikzcd}
				T\times X \arrow[r, "g\times f"] \arrow[d, "m"'] & S\times Y \arrow[d, "m"] \\
				X \arrow[r, "f"] & Y.
			\end{tikzcd}
		\end{center}
	\end{definition}
	
	\begin{definition}
		We denote by $\mathbf{FormAct}$ the category of pairs $(T,X)$ where $T$ is a formal group scheme and $X$ is a formal scheme with $T$-action, and morphisms are pairs of $(g,f)$ where $g$ is a group homomorphism and $f$ is a $g$-linear morphism. It has a natural forgetful functor to $
		\mathbf{FormGrp}$, the category of formal group schemes.
	\end{definition}
	
	\begin{definition}
		Let $g:T'\rightarrow T$ be a homomorphism of formal group schemes. Let $\pi:E\rightarrow B$ be a fibre bundle with fibre $F$ and structure group $T$ and let $\pi':E'\rightarrow B'$ be a fibre bundle with fibre $F'$ and structure group $T'$ . Then a $g$-linear morphism of principal bundles is a diagram
		\begin{center}
			\begin{tikzcd}
				E' \arrow[d] \arrow[d, "f_E"] \arrow[r, "\pi'"] & B' \arrow[d, "f_B"] \\
				E \arrow[r, "\pi"] & B
			\end{tikzcd}
		\end{center}		
		such that $f_E$ is $g$-linear. We denote by $\mathbf{ FormFibBun}$ the category of fibre bundles $(E,B,\pi,T,F)$ with arrows being the morphisms of principal bundles. It has a natural forgetful functor \[(E,B,\pi,T,F) \mapsto T\]
		to the category $\mathbf{FormGrp}$. We denote by $\mathbf{FormPrinBun}$ the full subcategory of principal bundles.
	\end{definition}
	We get the natural analogue of Lemma~\ref{equivalent characterisation of morphisms of principal T-bundles}:
	
	\begin{lemma}
		With the notations from Lemma~\ref{equivalent characterisation of morphisms of principal T-bundles}, a $g$-linear morphism of a principal $T'$-bundle to a principal $T$-bundle is equivalent to the data of morphisms $\theta: U_i'\rightarrow T$ such that the following diagram commutes on intersections:
		\begin{center}
			\begin{tikzcd}
				U'_{ij} \arrow[r, "\psi'_{ij}\times\theta_j"] \arrow[d, "(\psi_{ij}\circ f) \times \theta_i"'] & T'\times T \arrow[r, "g\times \operatorname{id}"] & T\times T  \arrow[d, "m"] \\
				T\times T \arrow[rr, "m"] &  & T
			\end{tikzcd}
		\end{center}
	\end{lemma}
	Or in short hand notation,
	\begin{equation}\label{shorthand for description of semi-linear morphism of fibre  bundles}
	g(\psi'_{ij}(u))\cdot\theta_j(u)=\psi_{ij}(f(u))\cdot \theta_i(u).
	\end{equation}
	
	
	Similarly as in Proposition~\ref{associated bundle construction is bifunctor} one can conclude from this that change of fibre is functorial in the following sense:
	
	\begin{proposition}\label{associated bundle construction in the semi-linear case is a sort of fibered bifunctor}
		
		Given any homomorphism of group schemes $g:T'\rightarrow T$ and a $g$-linear homomorphism $h:F'\rightarrow F$ of formal schemes with $T'$ and $T$-actions respectively, and a homomorphism $f:E'\rightarrow E$ of principal $T'$ and $T$-bundles over $g$, one obtains a morphism
		\[h\times^g f : F'\times^{T'}E'\rightarrow F\times^T E\]
		of fibre bundles over $g$, in a way that is functorial in $h,g,f$. 
		More precisely, the associated bundle construction is a fibered bifunctor
		\[-\times^{-}-: \mathbf{FormAct} \times_{\mathbf{FormGrp}} \mathbf{FormPrinBun}\rightarrow \mathbf{FormFibBun}. \]
	\end{proposition}
	\begin{proof}
		Let ($E$,$B$,$\pi$,$T$) and ($E'$,$B'$,$\pi'$,$T'$) be principal bundles. Let $F$ and $F'$ be formal schemes with $T$-action and $T'$-action respectively. Let $g:T\rightarrow T'$ be a group homomorphism and let $h:F'\rightarrow F$ be a $g$-equivariant morphism.
		Let $f:E'\rightarrow E$ be a morphism of principal fibre bundles over $g$.
		Then we can find compatible covers $\mathfrak U'$ of $B'$ and $\mathfrak U$ of $B$ such that locally we have diagrams like in Lemma~\ref{equivalent characterisation of morphisms of principal T-bundles}. Then locally $F\times^T E$ and $F'\times^T E'$ are of the form $F\times U_i$ and $F'\times U'_i$ such that we obtain a natural map
		\[F'\times U'_i\xrightarrow{(h\times^g f)} F\times U_i, \quad (x,u)\mapsto (h(x)\theta_i(u),f_B(u))\]
		(as before this description is just a short hand for a diagram of maps, and not a description in terms of "points"). These maps glue together over the cover, since on intersection Lemma~\ref{equivalent characterisation of morphisms of principal T-bundles} implies that we have a commutative diagram
		\begin{center}
			\begin{tikzcd}
				F'\times U'_{ij} \arrow[r, "h\times^g f"] & F\times U_{ij} \\
				F'\times U'_{ij} \arrow[u, "\psi'_{ij}\times \operatorname{id}"] \arrow[r, "h\times^g f"] & F\times U_{ij} \arrow[u, "\psi_{ij}\times \operatorname{id}"'].
			\end{tikzcd}
		\end{center}
		More precisely, by $g$-linearity of $h$ one has
		\[h(x\cdot\psi_{ij}'(u))\cdot \theta_j(u)  =  h(x)\cdot g(\psi_{ij}'(u))\cdot \theta_j(u)  \stackrel{(\ref{shorthand for description of semi-linear morphism of fibre  bundles})}{=} h(x)\cdot \psi_{ij}(f(u))\cdot \theta_i(u).\]
		This shows that the maps glue to a morphism $h\times^g f$ as desired.
		
		By refining covers, one easily checks that this is functorial in both components.
	\end{proof}
	We obtain a variant of Lemma~\ref{universal property of associated fibre construction for principal bundles} in the semilinear case:
	
	\begin{lemma}\label{universal property of associated fibre construction in the semilinear case}
		Let $E'$ be a principal $T'$ bundle and let $E$ be a principal $T$-bundle. Let $H'$ and $H$ be formal group schemes and assume there is a commutative diagram of group homomorphisms
		\begin{center}
			\begin{tikzcd}
					H' \arrow[r, "h"] & H \\
					T' \arrow[r, "g"] \arrow[u] & T \arrow[u].
			\end{tikzcd}
		\end{center}
		Let moreover $f:E'\rightarrow E$ be a $g$-linear morphism of fibre bundles.
		Then the map $h\times^g f$ from Proposition~\ref{associated bundle construction in the semi-linear case is a sort of fibered bifunctor} is the unique $h$-linear morphism of fibre bundles making the following diagram commute:
		\begin{center}
			\begin{tikzcd}
				H'\times^{T'}E' \arrow[r, "h\times^{f}g"] & H\times^{T}E \\
				E' \arrow[r, "f"] \arrow[u] & E \arrow[u].
			\end{tikzcd}
		\end{center}
	\end{lemma}
	\begin{proof}
		The morphism exists by Proposition~\ref{associated bundle construction in the semi-linear case is a sort of fibered bifunctor}. The vertical maps in the commutative diagram exist by functoriality via $E=T\times^{T}E\rightarrow H\times^{T}E$. 
		
		On the other hand, on any compatible trivialisation $T'\times U'\rightarrow T\times U$ of $f:E'\rightarrow E$ there is clearly only one way to extend this to $H'\times U'\rightarrow H\times U$ in a $h$-linear way.
	\end{proof}
	
	\begin{remark}\label{appendix in the case of rigid spaces and schemes}
	All that we have done in this chapter can be done in completely the same way with formal schemes replaced by rigid spaces (covers being replaced by admissible covers) and also for schemes, or in fact for any site. We have preferred to use formal schemes to make things more explicit. 
	The different categories of fibre bundles are well-behaved with respect to the usual functors between the different categories: For instance, by functoriality of fibre products there are natural rigidification and reduction functors from formal principal $T$-bundles over $\mathcal O_K$ to rigid principal $T_\eta$-bundles over $K$ on the generic fibre, and to principal $\overline{T}$-bundles on the reduction $\mathcal O_K/\pi$. Moreover, these generic fibre and reduction functors commute with the associated fibre construction:
	\end{remark}
	\begin{lemma}\label{associated bundle commutes with generic fibre}
		Let $T$ be a formal group scheme and let $\pi:E\rightarrow B$ be a principial $T$-bundle. Let $F$ be a formal scheme with an action by $T$. Then
		\[(F\times^T E)_\eta = F_\eta\times^{T_\eta} E_\eta \]
	\end{lemma}
	\begin{proof}
		This can be checked locally on any trivialising cover, where it is clear.
	\end{proof}
  


\addtocontents{toc}{\protect\setcounter{tocdepth}{0}} %some hack to hide the acknowledgements in the toc
\section*{Acknowledgements}
\addtocontents{toc}{\protect\setcounter{tocdepth}{2}} % end hack
This work started as a group project at the 2017 Arizona Winter School. We would like to thank Bhargav Bhatt for proposing the project, for his guidance and for his constant encouragement. In addition we would like to thank the organizers of the Arizona Winter School for setting up a great environment for us to participate in this project. 

During this work Dami\'an Gvirtz and Ben Heuer were supported by the Engineering and Physical Sciences Research Council [EP/L015234/1], the EPSRC Centre for Doctoral Training in Geometry and Number Theory (The London School of Geometry and Number Theory), University College London. 
Peter Wear was supported by NSF grant DMS-1502651 and UCSD and would like to thank Kiran Kedlaya for helpful discussions.
{\color{red} Other grants?}
 {\color{green} Can acknowledgements go back to front as soon as edits in \S1 and \S2 are done? } {\color{red} I have one concern -- namely the list of grants etc. is too long, so I thought maybe keeping it in the end, before the Appendix should be better -- in fact, Appendix should appear after references, and have its own references, no?} 
 
 
\begin{thebibliography}{99}
	
	\bibitem{Bosch lectures} 
	Siegfried Bosch,
	\textit{Lectures on Formal and Rigid Geometry}, Lecture Notes in Mathematics, vol 2015. Springer, Berlin/Heidelberg/New York (2014).
	
	\bibitem{BL} 
	Siegfried Bosch, Werner L\"utkebohmert,
	\textit{Degenerating abelian varieties}, Topology {\bf 30} (1991), 653-698.
	
	\bibitem{FvdP}
	Jean Fresnel and Marius van der Put,
	\textit{Rigid Analytic Geometry and its Applications}, Progress in Mathematics, vol. 218. Birkh\"auser Boston, Inc., Boston (2004).
	
	\bibitem{rigid geometry of curves} 
	Werner L\"utkebohmert,
	\textit{Rigid Geometry of Curves and Their Jacobians}, Springer, Ergeb. 3. Folge, vol. 61. Springer International Publishing (2016). 
	
	\bibitem{perfectoid} 
	Peter Scholze,
	\textit{Perfectoid spaces}, Publ. Math. de l'IHES {\bf 116} (2012), no. 1, 245-313.
	
	\bibitem{p-adic Hodge} 
	Peter Scholze,
	\textit{p-adic Hodge theory for rigid analytic varieties}, Forum Math. Pi {\bf 1} (2013), el, 77.
	
	\bibitem{torsion} 
	Peter Scholze,
	\textit{On torsion in the cohomology of locally symmetric varieties}, Ann. of Math. (2) {\bf 182} (2015), no. 3, 945-1066.
	
	\bibitem{SW} 
	Peter Scholze and Jared Weinstein,
	\textit{Moduli of p-divisible groups}, Camb. J. Math. {\bf 1} (2013), no. 2, 145-237.
	
\end{thebibliography}

	
	
\end{document}
