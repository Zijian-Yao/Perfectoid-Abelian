\documentclass[10pt,oneside]{amsart}
\usepackage{amsmath}
\usepackage{amsthm}
\usepackage{amsfonts}
\usepackage{amssymb,amscd,epsf,verbatim}
\usepackage{mathrsfs}
\usepackage{graphicx}
\usepackage{latexsym}
\usepackage{standalone}
\usepackage{lscape}
\usepackage[colorlinks=true]{hyperref}
\hypersetup{colorlinks, citecolor=blue, filecolor=black, linkcolor=red, urlcolor=green}
\usepackage{epstopdf}
\usepackage{tikz}
\usetikzlibrary{calc}
\usetikzlibrary{matrix,arrows,decorations.pathmorphing}
\usepackage{tikz-cd}
\usepackage{color}
\usepackage{geometry}
\usepackage{multirow}
\usepackage{enumitem}
\usepackage{framed}

\newcommand{\dstyle}{\displaystyle}

%%%%%%% please do NOT add any new command 
%%%%%%(unless it is absolutely necessary, in which case please send everyone an email about it.
%%%%%%%

%\theoremstyle{theorem}
\newtheorem{theorem}{Theorem}[section]
\newtheorem{lemma}[theorem]{Lemma}
\newtheorem{step}[theorem]{step}
\newtheorem{proposition}[theorem]{Proposition}
\newtheorem{corollary}[theorem]{Corollary}
\newtheorem{claim}[theorem]{Claim}
\newtheorem{conjecture}[theorem]{Conjecture}
\newtheorem*{outline}{Outline of Proof}

\theoremstyle{definition}
\newtheorem{definition}[theorem]{Definition}
\newtheorem{assumption}[theorem]{Assumption}

\theoremstyle{remark}
\newtheorem*{theorem*}{Theorem}
\newtheorem*{lemma*}{Lemma}
\newtheorem*{example}{Example}
\newtheorem*{question}{Question}
\newtheorem*{notation}{Notation}
\newtheorem*{exercise}{Exercise}
\newtheorem*{warning}{Warning}
\newtheorem{remark}[theorem]{Remark}



\title[perfectoid limits of rigid groups via formal models]{perfectoid limits of rigid groups via formal models} 
%\date{August 2017}
\author{
	Clifford Blakestad \and
	Ben Heuer \and 
	Damian Gvirtz \and
	Koji Shimizu \and 
	Peter Wear \and
	Daria Schedrina \and
	Zijian Yao}

\begin{document}
	
	\maketitle
	
	\begin{abstract}
For an abelian variety $A$ over an algebraically closed non-archimedean field, we show that there is a perfectoid space $A_\infty$ such that $A_{\infty}\sim\varprojlim_{[p]}A$. 
	\end{abstract}
	
	
	\tableofcontents
	
\newpage 
	
%%%%%%%%%%%%
%%      Section 1
%%%%%%%%%%%%
	\section{Introduction}
{\color{red} To do: Rewrite Introduction, summarize main results, outline, remarks about $GL_n$, etc.
	
Instead of an actual introduction, for now we have the following}

{\color{blue} Outline:} 

For an abelian variety $A$ over an algebraically closed non-archimedean field we show that there is a perfectoid space $A_\infty$ such that $A\sim\varprojlim_{[p]}A$. 

We first more generally consider a rigid group  $G$ over a non-archimedean field $K$. While inverse limits usually don't exist in the rigid analytic category, limits are much better behaved in formal schemes over the ring of integers $\mathcal O_K$ of $K$. One can therefore give a simple criterion in terms of formal models that guarantees that a tilde-limit $G_\infty \sim\varprojlim_{[p]} G$ exists, namely that there is a well-behaved formal model of the $[p]$-multiplication tower.
If $K$ is perfectoid, we give a stronger criterion involving a Frobenius factorisation condition, which implies that $G_\infty$ is perfectoid.

In the case of a rigid analytic split torus $T$, one can use a family of explicit covers by affinoids to construct formal models for which both of these conditions are satisfied. 

Next we consider the case of the Raynaud extension $E$ associated to a semistable abelian variety $A$ over a perfectoid field $K$. One can construct $E$ by extending the rigid fibre of a formal group scheme $\overline{E}$ by a rigid torus $T$. In order to construct a formal model of $E$ one therefore just needs to extend $\overline{E}$ by a formal model of $T$. While this can be done explicitly using affinoid covers, the language of formal and rigid fibre bundles allows for a more conceptual treatment. Using the associated fibre construction we then show that there is a formal model of the $[p]$-multiplication tower of $E$ which satisfies all the necessary criteria to show that $E_\infty$ exists and is perfectoid.  

We then construct a tilde-limit of $\varprojlim_{[p]} A$ from $E_\infty$: By Raynaud uniformisation, $A$ is naturally isomorphic to the rigid analytic quotient of $E$ by a lattice $M$. After a choice of $\Gamma_0(p^\infty)$-structure, the $[p]$-multiplication tower of $E/M$ factors in a ``ramified'' and an ``\'etale'' part. By a careful choice of charts of $E/M$ in terms of subspaces of $E$ that behave well under $[p^n]:E\rightarrow E$, one can explicitly construct first a perfectoid tilde-limit of the ``ramified'' tower, and then in a second step the space $A_\infty$. This space is independent of the choice of $\Gamma_0(p^\infty)$-structure but remembers it as a pro-\'etale subgroup $D_\infty \subseteq A_\infty$. The construction shows that the perfectoid tilde-limit $A_\infty$ still exists under the weaker assumption that $K$ is a perfectoid field over which there exist lattices $M^{1/p^n}$ for all $n$ whose $p^n$-th multiple is $M$.

The approach via explicit covers finally gives an explicit description of $A_\infty$ in terms of open subspaces of $E_\infty$ which we use in the last section to study the induced map $E_\infty\rightarrow A_\infty$. We show that $E_\infty$ is in fact an open subspace of $A_\infty$ and thus obtain a second description of $A_\infty$ as a quotient group $(D_\infty\times E_\infty)/M_\infty$.


{\color{blue} Notation:} 

	Throughout we will study rigid analytic spaces over $K$. If such a space is obtained from a $K$-scheme $X$ via rigid-analytification $X\mapsto X^{\operatorname{an}}$, we will often denote both by the same symbol $X$. Also, we will make no distinction between rigid analytic spaces and their corresponding adic spaces.

	
	Throughout we denote by $\mathfrak X$ a topologically finite type formal scheme over $\operatorname{Spf} R$ with the $\pi$-adic topology. Let $\mathfrak X_\eta \rightarrow \operatorname {Sp} K$ be its rigid generic fibre, and let $\tilde{X}=\mathfrak X\times_{\operatorname{Spf}\mathcal O_K}\operatorname{Spf}\mathcal O_K/\pi$ be its special, considered as a scheme over $\operatorname{Spec}\mathcal O_K/\pi$.
{\color{red} Add definition of tilde limits}

{\color{red} Add acknowledgements}  	
	

%%%%%%%%%%%%
%%      Section 2
%%%%%%%%%%%%	
	\section{Tilde-limits of rigid groups}
	
	

	Let $p$ be a prime. Let $K$ be a complete non-archimedean field that is either an extension of $\mathbb Q_p$ or of characteristic $p$. Denote by $\mathcal O_K$ the ring of integers and let $\pi$ denote a pseudo-uniformiser. 
	
	
	Let $G$ be a rigid group, that is a group object in the category of rigid spaces. One way that rigid groups arise is by analytification of finite type group schemes over $K$: We will be most interested in the analytification of an abelian variety $A$, but other important cases are the analytifications  $\mathbb G_a^{\operatorname{an}}$ of $\mathbb G_a$ and $\mathbb G_m^{\operatorname{an}}$ of $\mathbb G_m$, or more generally of tori $T$ over $K$. A second source of rigid groups are generic fibres of topologically finite type formal group schemes over $\mathcal O_K$. A third important example is the covering space $E$ in the sense of Raynaud of an abelian variety with semi-stable reduction.
	
	This note is concerned with the following question:
	
	\begin{framed}
	\begin{question}
		Given a rigid group $G$, when is there an adic space $G_\infty$ such that \[G_\infty \sim  \varprojlim_{[p]} G\] in the sense of \cite{SW}? If it exists, and $K$ is perfectoid, when is $G_\infty$ perfectoid?
	\end{question}
	\end{framed}
	Note that if a perfectoid tilde-limit exists, it is unique with this property by Proposition~2.4.5 in \cite{SW}.
	Our main goal will be to show that this is true for $A$ an abelian variety over a perfectoid field.
	But before we give proves for examples of rigid groups $G$ for which a perfectoid tilde-limit exists, we first note that the second question certainly doesn't have an affirmative answer for all rigid group varieties:
	\begin{example}
		For the additive group $\mathbb G_a^{\operatorname{an}}$, we know that $[p]$ is an isomorphism and therefore $\varprojlim_{[p]} \mathbb G_a=\mathbb G_a$ exists (even as an actual limit in the category of adic spaces) but is certainly not perfectoid.
	\end{example}
	\subsection{A condition ensuring that the tilde-limit exists}
  
	Inverse limits often don't exist in the category of adic spaces, and neither do they in rigid spaces. They do, however, often exist in the category of formal schemes:
	\begin{lemma}
		Let $(\mathfrak X_i,\phi_{ij})_{i\in I}$ be an inverse system of formal schemes $\mathfrak X_i$ over $\mathcal O_K$ with affine transition maps $\phi_{ij}:\mathfrak X_j\rightarrow \mathfrak X_i$. Then the inverse limit $\mathfrak X=\varprojlim \mathfrak X_i$ exists in the category of formal schemes over $\mathcal O_K$. If all the $\mathfrak X_i$ are flat formal schemes, so is $\mathfrak X$.
	\end{lemma}
	\begin{proof}
	In the affine case, if the inverse system is $\operatorname{Spf} A_i$, take $A$ to be the $p$-adic completion of $\varinjlim A_i$, then  $\operatorname{Spf} A$ is the inverse limit of the $\operatorname{Spf}A_i$. In general, we can use the fact that the transition maps are affine to reduce to the affine case. 
	\end{proof}
	In the situation of the lemma,  the transition maps are affine and hence quasi-compact and quasi-separated, so after passing to adic spaces, $\mathfrak X$ is also the tilde-limit  $\mathfrak X\sim \varprojlim \mathfrak X_i$ in the sense of \cite{SW}. Even better, this remains true after passing to the generic fibre $\operatorname{Spa}(K,\mathcal O_K)\rightarrow \operatorname{Spa}(\mathcal O_K,\mathcal O_K)$.
	\begin{lemma}\label{tilde-limit from adic generic fibre of formal schemes}
		Let $(\mathfrak X_i,\phi_{ij})_{i\in I}$ be an inverse system of formal schemes $\mathfrak X_i$ over $\mathcal O_K$ with affine transition maps $\phi_{ij}$ and let $\mathfrak X=\varprojlim_{\phi_j} \mathfrak X_i$ be the limit. Let $\mathcal X_i =(\mathfrak X_i)_\eta$ and  $\mathcal X = (\mathfrak X)_\eta$ be the adic generic fibres. Then
		\[\mathcal X \sim \varprojlim \mathcal X_i.\]
	\end{lemma}
	\begin{proof}
		This is a consequence of \cite{SW}, Proposition 2.4.2: The transition maps in the system are affine, hence quasi-separated quasi-compact. In order to prove the Lemma, we can restrict to an affine open subset $\operatorname{Spf}(A)$ of $\mathfrak X$ that arises as the inverse limit of affine open subsets $\operatorname{Spf}(A_i)\subseteq \mathfrak X_i$. Here all formal schemes are considered with the $\pi$-adic topology and $A$ is the $\pi$-adic completion of $\varinjlim A_i$. 
		On the generic fibre, $A_i$ with ideal of definition $I_i=\pi A_i$ is an open subring of definition of $A_i[1/\pi]$. We then clearly have $I_iA_j = A_j$ for any $j\geq i$. The inverse system therefore satisfies the conditions of \cite{SW}, Proposition 2.4.2, and we conclude that $\operatorname{Spf}(A)_\eta \sim \varprojlim \operatorname{Spf}(A_i)_\eta$ as desired.
	\end{proof}
	
	\begin{remark}
	This means that one can always construct the limit of an inverse system of rigid spaces $\mathcal X_i$ if it arises from an inverse system of formal schemes $\mathfrak X_i$ with affine transition maps. This is precisely what Scholze uses in \cite{torsion} in order to construct the space $\mathcal X_{\Gamma_0(p^\infty)}(\epsilon)_a$ (see the proof of Corollary III.2.19 in \cite{torsion}).
	\end{remark}
	
	If one starts with an inverse system of rigid spaces $\mathcal X_i$, a straightforward strategy to construct ``the" tilde limit $\varprojlim \mathcal X_i$ is thus to look for formal models $\mathfrak X_i$, that is formal schemes over $\operatorname{Spf} \mathcal O_K$ such that $\mathcal X_i = (\mathfrak X_i)_\eta$, as well as affine formal models $\phi_{ji}:\mathfrak X_j\rightarrow \mathfrak X_i$ of the transition maps. If such data exists, Lemma~\ref{tilde-limit from adic generic fibre of formal schemes} produces a tilde-limit $\mathcal X \sim \varprojlim \mathcal X_i$. Here we follow the following standard terminology:
	\begin{definition}
		\begin{enumerate}
			\item Let $\mathcal X$ be a rigid space over $K$. Then a \textbf{formal model} of $\mathcal X$ is an admissible topologically finite type formal scheme $\mathfrak X$ over $\mathcal O_K$ together with an isomorphism of its generic fibre $\mathfrak X_\eta \xrightarrow{\sim} \mathcal X$ (which is often suppressed from notation).
			\item Let $\phi:\mathcal X\rightarrow \mathcal Y$ be a morphism of rigid spaces over $K$. Let $\mathfrak X,\mathfrak Y$ be formal models of $\mathcal X,\mathcal Y$ respectively. Then a morphism of formal schemes $\Phi:\mathfrak X \rightarrow \mathfrak Y$ is a \textbf{formal model} of $\phi$ if the following diagram commutes:
			\begin{center}
				\begin{tikzcd}
					\mathcal X \arrow[r, "\phi"] & \mathcal Y \\
					\mathfrak X_\eta \arrow[u, "\cong"] \arrow[r, "\Phi"] & \mathfrak Y_\eta \arrow[u, "\cong"]
				\end{tikzcd}
			\end{center}
			
		\end{enumerate}
		
	\end{definition}	
 
	The theory of Raynaud's formal models explains under which circumstances formal models of rigid spaces and their maps exist. We need the following definition:
	
	\begin{definition}[\cite{Bosch lectures}, Def 8.2.12]
	A topological (resp G-topological) space $X$ is called \textbf{quasi-paracompact} if there exists an open (resp admissible open) cover $\mathfrak U$ of $X$ such that
	\begin{itemize}
		\item each $U\in \mathfrak U$ is quasi-compact and
		\item the cover $\mathfrak U$ is of finite type, that is for each $U_i \in \mathfrak U$ there are only finitely many $U_j \in \mathfrak U$ such that $U_i\cap U_j\ne\emptyset$.
	\end{itemize}
	For instance, the spaces $\mathbb G_a^{\operatorname{an}}$ and $\mathbb G_m^{\operatorname{an}}$ are not quasi-compact, but they are quasi-paracompact since they can be covered using families of annuli that are admissible covers of finite type. Similarly, one should be able to show (replacing annuli by intersections of Laurent domains and Weierstrass domains) that if $X$ is any quasi-compact space and $S\subseteq X$ is a Zariski-closed subset, then $X\setminus S$ is quasi-paracompact.
	\end{definition}
		
	The main result of Raynaud's theory of formal models is then:
	
	\begin{theorem}[\cite{Bosch lectures}, section 8.4]\label{Raynaud theory main theorem}
		\leavevmode
		\begin{enumerate}
			
			\item Let $X$ be a quasi-separated quasi-paracompact rigid space over $K$. Then there exist an admissible quasi-paracompact formal scheme $\mathfrak X$ over $\mathcal O_K$ such that $X=\mathfrak X_\eta$.
			\item If $\mathfrak X'\rightarrow \mathfrak X$ is an admissible blow-up of admissible formal schemes, then its generic fibre is an isomorphism $\mathfrak X'_\eta \xrightarrow{\sim} \mathfrak X_\eta$.
			\item Let $\mathfrak X$ and $\mathfrak Y$ be admissible quasi-paracompact formal schemes over $\mathcal O_K$ and let $f:\mathfrak X_\eta \rightarrow \mathfrak Y_\eta$ be a morphism of their associated rigid spaces. Then there exist an admissible blow-up $\pi:\mathfrak X'\rightarrow \mathfrak X$ and a map $\mathfrak f:\mathfrak X'\rightarrow \mathfrak Y$ such that $\mathfrak f_\eta = f\circ \pi_\eta$.
			\begin{center}
				
			\begin{tikzcd}
				\mathfrak X' \arrow[d,"\pi"] \arrow[rrd, "\mathfrak f"] &  &  &  & \mathfrak X'_\eta \arrow[d, "\cong","\pi_\eta"'] \arrow[rrd, "\mathfrak f_\eta"] &  &  \\
				\mathfrak X &  & \mathfrak Y &  & \mathfrak X_\eta \arrow[rr, "f"] &  & \mathfrak Y_\eta
			\end{tikzcd}
			\end{center}
		\end{enumerate}
	\end{theorem}
	
	The theorem implies that given an inverse system ($\mathcal X_i,\phi_{ij})$ of rigid spaces, one can always choose formal models $\mathfrak X_i$ and by successive admissible blow-ups while going along the inverse system one can also find models for the $\phi_{ij}$. If it is possible to do this in such a way that transition maps are affine, this way one always obtains a construction of $G_\infty$. More precisely, we can formalise this as follows:
	
	\begin{definition}
		For a rigid analytic group $G$, we call \textbf{$[p]$-model tower} the data of:
		\begin{enumerate}
			\item a family of formal models $\mathfrak G_n$ of $G$ for $n\in \mathbb N$,
			\item morphisms of formal schemes $[p]:\mathfrak G_{n+1}\rightarrow \mathfrak G_{n}$ satisfying the following conditions:
			\begin{enumerate}
				\item $[p]:\mathfrak G_n\rightarrow \mathfrak G_{n+1}$ is a formal model of $[p]:G\rightarrow G$. 
				\item $[p]$ is an affine morphism
			\end{enumerate}
		\end{enumerate}
	\end{definition}
	More generally, by a $[p]$-model tower of an admissible open subset $U\subseteq G$ we mean the same formal model data for the tower of pull-backs over $U$
	
	\begin{center}
		\begin{tikzcd}
			\dots \arrow[r] & \left [p^2\right]^{-1}(U) \arrow[r, "{[p]}"] \arrow[d, hook'] & \left [p\right]^{-1}(U) \arrow[r, "{[p]}"] \arrow[d, hook'] & U \arrow[d, hook'] \\
			\dots \arrow[r, "{[p]}"] & G \arrow[r, "{[p]}"] & G \arrow[r, "{[p]}"] & G
		\end{tikzcd}
	\end{center}
	
	We can summarise our discussion in this chapter by the following Proposition:
	\begin{proposition}
		Let $G$ be a rigid analytic group. Then if $G$ has a $[p]$-model tower, there exists a space $G_\infty$ such that $G_\infty \sim \varprojlim_{[p]}G$.
	\end{proposition}
	
	\subsection{A condition ensuring that the tilde-limit is perfectoid}
	
	First we consider the case where $G=A$ is an abelian variety with good reduction. Slightly more generally, we have
	\begin{proposition}\label{limit exists and is perfectoid: commutative formal group case}
		Assume that $K$ is perfectoid. Let $\mathfrak G$ be a flat commutative formal group scheme of topologically finite type over $\mathcal O_K$ for which the p-multiplication map $[p]:\mathfrak G\rightarrow \mathfrak G$ is affine. Let $G = \mathfrak G_\eta$ be the rigid group obtained on the generic fibre. Then $G_\infty$ exists and is perfectoid.
	\end{proposition}
	\begin{proof}
		Following notation from the previous subsection, the map $[p]:\mathfrak G\rightarrow \mathfrak G$ is a formal model of the map $[p]$ on its rigid generic fiber. By Lemma~\ref{tilde-limit from adic generic fibre of formal schemes} we therefore have 
		
		\[G_\infty = (\varprojlim_{[p]}\mathfrak G)_\eta \sim \varprojlim_{[p]}G. \] 
		
		To see that $G_\infty$ is perfectoid, we proceed exactly like in the proof of \cite{torsion}, Corollary III.2.19. It suffices to prove that $\mathfrak G_\infty = \varprojlim_{[p]} \mathfrak G$ can be covered by formal schemes of the form $\operatorname{Spf}(S)$ where $S$ is a flat $\mathcal O_K$-algebra such that the Frobenius map \[S/p^{1/p} \rightarrow  S/p\] is an isomorphism. Lemma~5.6 of \cite{perfectoid} then shows that $S[1/p]$ is perfectoid.
		
		The key observation here is that upon reduction mod $p$, the $p$-multiplication factors through relative Frobenius. More precisely, denote by $\tilde{G}$ the reduction of $\mathfrak G$ mod $p$. Then $[p]:\tilde{G}\rightarrow \tilde{G}$ factors as 
		\begin{center}
			\begin{tikzcd}[row sep = small]
				& \tilde{G} \arrow[rd, dashed] &  \\
				\tilde{G}^{(p)} \arrow[ru, "F_{rel}"] \arrow[rr, "{[p]}"] &  & \tilde{G}
			\end{tikzcd}
		\end{center}
		
		This has the following consequence: Let $\operatorname{Spf}(S_1)$ be any affine open subspace of $\mathfrak G$ and let $\operatorname {Spf}S_n$ be the pullback via $[p^n]:\mathfrak G\rightarrow \mathfrak G$. Then we have a commutative diagram:
		\begin{center}
			\begin{tikzcd}[row sep = small]
				&  & \tilde{S}_{n}^{(p)} \arrow[rd, "F_{rel}"] &  & \tilde{S}_{n+1}^{(p)} \arrow[rd, "F_{rel}"] &  &  \\
				\dots \arrow[r] & \tilde{S}_{n-1} \arrow[rr] \arrow[ru, "V", dashed] &  & \tilde{S}_n \arrow[ru, "V", dashed] \arrow[rr, "{[p]}"] &  & \tilde{S}_{n+1} \arrow[r] & \dots
			\end{tikzcd}
		\end{center}
		From this we can check on elements that relative Frobenius is an isomorphism on $\tilde{S}_\infty := \varinjlim_n \tilde{S}_n$. Since $K$ is perfectoid, we moreover have an isomorphism $\mathcal O_K/p^{1/p}\rightarrow \mathcal O_K/p$ from the absolute Frobenius on $\mathcal O_K/p$. Therefore absolute Frobenius on $S_\infty/p$ induces an isomorphism
		\[S_\infty/p^{1/p}\xrightarrow{\sim} S_\infty/p.\]
		Since $\mathfrak G$ is flat, so are the $S_n$ and thus so is $S_\infty$. Thus $S_\infty[1/p]$ is a perfectoid $K$-algebra.
		Since $G_\infty$ is covered by affinoids of the form $\operatorname{Spf}(S_\infty)_\eta$, this shows that $G_\infty$ is perfectoid.
	\end{proof}
	
	
	\begin{corollary}\label{tilde limit exists in the good reduction case}
		Let $A$ be an abelian variety of good reduction over a perfectoid field $K$. Then $A_\infty$ exists and is perfectoid.
	\end{corollary}
	 	Before we proceed, we would like to mention two illustrative examples:
	\begin{example}
		Let $\mathfrak G$ be the $p$-adic completion of the affine group scheme $\mathbb G_m$ over $\mathcal O_K$, that is the formal scheme of $\mathfrak G$ is $\operatorname {Spf} S$ where $S=\mathcal O_K\langle X^{\pm 1} \rangle$.  It is clear that $\mathbb G$ satisfies the conditions of Proposition~\ref{limit exists and is perfectoid: commutative formal group case}, so for the generic fibre $G=\mathfrak G_\eta$ we obtain a perfectoid tilde-limit $G_\infty = \varprojlim_{[p]} G$. More precisely, the $[p]$-multiplication map corresponds to the homomorphism
		\[[p]:\mathcal O_K\langle X^{\pm 1} \rangle\rightarrow  \mathcal O_K\langle X^{\pm 1} \rangle, \quad X\rightarrow X^{p}.\]
		In the direct limit, we obtain the algebra $S_\infty = (\varinjlim_{[p]} S)^{\wedge} = \mathcal O_K\langle  X^{\pm 1/p^\infty} \rangle$. On the generic fibre we thus obtain
		\[G_\infty = \operatorname{Spa}(K\langle X^{\pm 1/p^\infty} \rangle,\mathcal O_K\langle X^{\pm 1/p^\infty} \rangle)\]
		and one can verify by hand that we indeed have $G_\infty \sim \varprojlim_{[p]} G$.
	\end{example}
	\begin{example}
		An example of a very different flavour is $\mathfrak G$ the $p$-adic completion of the affine group scheme $\mathbb G_a$ over $\mathcal O_K$. Note that $G=\mathfrak G_\eta$ is not equal to $\mathbb G_a^{an}$, but is the closed unit disc in the latter.
		
		While the underlying formal scheme of $\mathfrak G$ is $\operatorname {Spf} S$ where $S=\mathcal O_K\langle X \rangle$ as before, the $[p]$-multiplication is now given by
		\[[p]:\mathcal O_K\langle X \rangle\rightarrow  \mathcal O_K\langle X \rangle, \quad X\rightarrow pX.\]
		In the direct limit, we first obtain the algebra $S'_\infty = \varinjlim_{[p]} S = \mathcal O_K\langle \frac{1}{p^\infty}X \rangle$ of power series $f=\sum_{n=0}^\infty  a_nX^n\in \mathcal O_K[[X]]$ for which there is $m\in \mathbb Z_{\geq 0}$ such that $|p^{nm}a_n|\to 0$. In order to form $S_\infty$, we need to $p$-adically complete $S'_\infty$. But we have 
		\[p^n\mathcal O_K\langle \frac{1}{p^\infty}X \rangle = p^n\mathcal O_K + X \mathcal O_K\langle \frac{1}{p^\infty}X \rangle\]
		and therefore $S'_\infty/p^n=\mathcal O_K/p^n\mathcal O_K$. Consequently, the completion is $S_\infty = \mathcal O_K$ and thus $G_\infty = \operatorname{Spa}(K,\mathcal O_K)$ is perfectoid, but just one point!
		
		Geometrically, this makes sense: On the level of $K$-points, the formal scheme $G$ is the closed unit disc and $[p]$ is scaling points by $p$. A $K$-point in $\varprojlim_{[p]} G(K)$ therefore corresponds to a sequence of $K$-points of the closed unit disc of the form $x,\frac{1}{p}x,\frac{1}{p^2}x,\dots$. But for this to be contained in the closed unit disc, we must have $x=0$. Thus $\varprojlim_{[p]} G(K)=0$.
		
	\end{example}
	
	Looking closely at  Proposition \ref{limit exists and is perfectoid: commutative formal group case}, all we need is in fact a model for the $p$-multiplication morphism $[p]:G\rightarrow G$, and we never use that all the $\mathfrak G$ in the tower are copies of the same formal scheme. Weakening these two conditions, we arrive at the following definition:
	
	\begin{definition}
		For a rigid analytic group $G$, we call \textbf{$[p]$-$F$-model tower} the data of:
		\begin{enumerate}
			\item a family of flat formal models $\mathfrak G_n$ of $G$ for $n\in \mathbb N$,
			\item morphisms of formal schemes $[\mathfrak p]:\mathfrak G_{n+1}\rightarrow \mathfrak G_{n}$ satisfying the following conditions:
			\begin{enumerate}
				\item the generic fibre of $[\mathfrak p]:\mathfrak G_n\rightarrow \mathfrak G_{n+1}$ coincides with $[p]:G\rightarrow G$. 
				\item $[\mathfrak p]$ is an affine morphism
				\item Denote by $\tilde{G}_n$ the reduction of $\mathfrak G_n$ mod $p$. Then $[\mathfrak p]$ factors through the relative Frobenius morphism:
				\begin{center}
					\begin{tikzcd}
						& \tilde G_{n}^{(p)} \arrow[rd, dashed] &  \\
						\tilde G_{n} \arrow[rr, "{[p]}"] \arrow[ru, "F_{rel}"] &  & \tilde G_{n-1}
					\end{tikzcd}
				\end{center}
				
			\end{enumerate}
			
			 
		\end{enumerate}
	\end{definition}
	More generally, as before, we can consider $[p]$-$F$-model towers of admissible open subset $U\subseteq G$ by which we mean the same formal model data for the tower of pull-backs over $U$.
	
	\begin{example}
		If $\mathfrak G$ is a flat commutative formal group scheme such that $p$-multiplication is affine, then setting $\mathfrak G_n = \mathfrak G$ and taking for $[\mathfrak p]$ the actual $p$-multiplication maps $[p]$ defines a $[p]$-$F$-model tower for the rigid analytic group $G=\mathfrak G_\eta$.
	\end{example}
	
	 The same proof for Proposition~\ref{limit exists and is perfectoid: commutative formal group case} works with the weakened conditions: 
	\begin{proposition}\label{existence of p-F-model tower implies perfectoid}
		Let $G$ be a rigid analytic group over a perfectoid field $K$. If $G$ admits a $[p]$-$F$-model, then $G_\infty$ exists and is perfectoid.
	\end{proposition}
 
	
	What we aim to prove in the rest of this write-up is that for a Raynaud extension $0\rightarrow T\rightarrow E\rightarrow B\rightarrow 0$, there is a $[p]$-$F$-model for $T$ which induces a $[p]$-$F$-model for $E$. This will prove that tilde-limits $T_\infty$ and $E_\infty$ exist and are perfectoid if $K$ is perfectoid.
	
	\subsection{A few consequences}
	
	One reason why perfectoid limits along group morphisms are particularly interesting is that the perfectoidness ensures that the limit has again a group structure:
	
	\begin{definition}
		By a \textbf{perfectoid group} we mean a group object in the category of perfectoid spaces.
		Note that the category of perfectoid spaces over $K$ has finite products, so the notion of a group object makes sense. 
	\end{definition}
	
	\begin{proposition}\label{perfectoid tilde limit is perfectoid group in a functorial way}
		Let $G$ be a rigid group and assume that there is a perfectoid space $G_\infty$ such that $G_\infty \sim \varprojlim_{[p]} G$. Then
		\begin{enumerate}
		\item  there is a unique way to endow $G_\infty$ with the structure of a perfectoid group in such a way that all projections $G_\infty\rightarrow G$ are group homomorphisms
		\item given a rigid group $H$ with perfectoid tilde-limit $H_\infty\sim\varprojlim_{[p]}H$ and a group homomorphism $H\rightarrow G$, there is a unique group homomorphism $H_\infty\rightarrow G_\infty$
		commuting with all projection maps. In particular, formation of $\varprojlim_{[p]}$ is functorial.
	\end{enumerate}
	\end{proposition}
	\begin{proof}
		These are all consequences of the universal property of the perfectoid tilde-limit, cf Proposition~2.4.5 of \cite{SW}, which shows that one can argue like in the case of usual limits.
	\end{proof}
	
%%%%%%%%%%%%
%%      Section 3
%%%%%%%%%%%%	
	\section{Formal models for tori}
	
	Let $K$ be perfectoid. In this section we want to show that for a split rigid torus $T$ over $K$, a tilde-limit $T_\infty$ exists and is perfectoid. We do this by exhibiting a $[p]$-$F$-model of $T$.
	
	As a preparation, we consider the torus $\mathbb G_m^{\operatorname{an}}$ over $K$. Recall that it arises from rigid analytification of the affine torus $\mathbb G_m$ over $K$. Note however that $\mathbb G_m^{\operatorname{an}}$ is not affinoid (and not even quasi-compact). It contains the generic fibre of the $p$-adic completion of $\mathbb G_m$ as an open subspace. If we see $\mathbb G_m^{\operatorname{an}}$ as the rigid affine line with origin removed, this subspace $\widehat{\mathbb G}_m$ can be identified with the open annulus of radius 1. In other words, on the level of points it corresponds to $\mathcal O_K^\times \subseteq K^\times$.
	
	Finally, recall that for every $x\in K^\times$ we have a translation map
	\[\mathbb G_m^{\operatorname{an}}\xrightarrow{x\cdot} \mathbb G_m^{\operatorname{an}}\]
	that is an isomorphism of rigid spaces sending the point $1$ to $x$.
	
	\subsection{A family of explicit covers}
	We briefly recall how $\mathbb G_m^{\operatorname{an}}$ is constructed: The following is inspired by \cite{Bosch lectures}, \S 9.2, although we choose slightly different constructions. Throughout we use the following shorthand notation: for any $a\in K$ we write
	\[K\langle X,a/X\rangle = K\langle X,Z\rangle/(X\cdot Z - a). \]
	
	Let $q \in K^\times$ with $|q|\leq 1$. Consider the annulus $\mathcal B(q,1)$ of radii $|q|$ and $1$ inside $\mathbb A_K^{\operatorname{an}}$:
	\[\mathcal B(q,1) = \operatorname{Sp}(L_q),\quad \text{ where } L_q = K\langle X,q/X\rangle/a. \]
	Similarly, for $q\in K^\times$ with $|q|\geq 1$ one constructs the annulus $\mathcal B(1,q)$ by
	\[\mathcal B(1,q) = \operatorname{Sp}(L_{q}),\quad \text{ where } L_{q} = K\langle X/q,1/X\rangle\]
	where $K\langle X/q\rangle$ denotes the ring of those power series $f=\sum c_nX^m\in K[[X]]$ for which $|c_n/q|\to 0$ for $n\to \infty$. In particular, we have isomorphisms
	 \[K\langle X',q^{-1}/X'\rangle\cong K\langle X/q,1/X\rangle,\quad X'\mapsto q^{-1}X.\]
	One can now construct $\mathbb G_m$ as follows: Choose sequences $a_n,b_n\in K^\times$ with $a_0=1=b_0$ such that $|a_n|<|a_{n-1}|<...<1$ and $|a_n| \to 0$ and similarly $|b_n|>|b_{n-1}|>...>1$ and $|b_n| \to \infty$. Then one can glue the annuli $\mathcal B(a_n,1)$ and $\mathcal B(1,b_n)$ using the following maps:
	\begin{equation}\label{torus explicit cover glue map 1}
	\begin{alignedat}{2}
	\mathcal B(a_{n},1)&\hookleftarrow&& \mathcal B(a_{n-1},1)\\
	L_{a_n}=K\langle X,a_n/X\rangle&\rightarrow &&K\langle X,a_{n-1}/X\rangle=L_{a_{n-1}}\\
	X,a_{n}/X&\mapsto&& X, \frac{a_{n}}{a_{n-1}}a_{n-1}/X
	\end{alignedat}
	\end{equation}
	and similarly 
	\begin{equation}\label{torus explicit cover glue map 2}
	\begin{alignedat}{2}
	\mathcal B(1,b_n)&\hookleftarrow&& \mathcal B(1,b_{n-1})\\
	L_{b_n}=K\langle  X/b_{n},1/X\rangle&\rightarrow &&K\langle X/b_{n-1},1/X\rangle=L_{b_{n-1}}\\
	X/b_{n},1/X&\mapsto&& \frac{b_{n-1}}{b_{n}} X/b_{n-1}, 1/X.
	\end{alignedat}
	\end{equation}
	
	Also, via the above maps, the annuli $\mathcal B(a_n,1)$ and $\mathcal B(1,b_m)$ are glued along $\mathcal B(a_0,1)=\mathcal B(1,1)=\mathcal B(1,b_0).$ This gives the desired space $\mathbb{G}_m^{\operatorname{an}}$.
	
	Since we are mainly interested in the $p$-multiplication map, we will more precisely use the following cover on which $[p]$ can be seen directly: Choose $q\in K^\times$ with $|q|<1$. Then for the sequences $a_n$ and $b_n$ from above we take $a_n = q^n$, $b_n = q^{-n}$. 
	We call this cover $\mathfrak U_q$.
	
	Assume now that $q$ has a $p$-th root $q^{1/p}$ in $K$. The above then gives a finer cover $\mathfrak U_{q^{1/p}}$ of $\mathbb G_m^{\operatorname{an}}$. Using both covers $\mathfrak U_q$ and $\mathfrak U_{q^{1/p}}$, we can easily see the $[p]$-multiplication $[p]:\mathbb G_m^{\operatorname{an}}\rightarrow \mathbb G_m^{\operatorname{an}}$ as follows: Consider the affinoid open subsets $\mathcal B(q^{1/p},1)$ of the source and  $\mathcal B(q^{1/p},1)$ of the target. Then $[p]$ restricts to
	\begin{equation}
	\begin{alignedat}{2} \label{torus explicit [p] map 1}
	\mathcal B(q,1)&\xleftarrow{[p]}&& \mathcal B(q^{1/p},1)\\
	K\langle X,q/X\rangle&\rightarrow &&K\langle X,q^{1/p}/X\rangle\\
	X,q/X&\mapsto&& X^p, (q^{1/p}/X)^p
	\end{alignedat}
	\end{equation}
	and similarly, on $\mathcal B(1,q^{-1/p})$ and $\mathcal B(1,q^{-1})$ the map is 
	\begin{equation}
	\begin{alignedat}{2} \label{torus explicit [p] map 2}
	\mathcal B(1,q^{-1})&\xleftarrow{[p]}&& \mathcal B(1,q^{-1/p})\\
	K\langle X/q,1/X\rangle&\rightarrow &&K\langle X/q^{1/p},1/X\rangle\\
	X/q,1/X&\mapsto&& (X/q^{1/p})^p, (1/X)^p.
	\end{alignedat}
	\end{equation}
	The same works for the other affinoid open subspaces $\mathcal B(q^{n},1)\xleftarrow{[p]} \mathcal B(q^{n/p},1)$ and for $\mathcal B(1,q^{-n})\xleftarrow{[p]} \mathcal B(1,q^{-n/p})$.
	One can then show that the maps~(\ref{torus explicit [p] map 1}) and (\ref{torus explicit [p] map 2}) are compatible with the glue maps~(\ref{torus explicit cover glue map 1}) and~(\ref{torus explicit cover glue map 2}). In the case of~(\ref{torus explicit [p] map 1}) this is basically because $a_n/a_{n-1} = q$ or $a_n/a_{n-1}=q^{1/p}$ depending on whether we work with $\mathfrak U_q$ or $\mathfrak U_{q^{1/p}}$ respectively, and the only thing to check is that the following diagram commutes:
	\begin{center}
		\begin{equation}\label{diagram showing that [p]-multiplication on torus glues together}
		\begin{tikzcd}
			\mathcal B(q^{n},1) & \mathcal B(q^{n-1},1) \arrow[l, hook'] & q^n/X \arrow[r, maps to] \arrow[d, maps to] & q\cdot q^{n-1}/X \arrow[d, maps to] \\
			\mathcal B(q^{n/p},1) \arrow[u, "{[p]}"] & \mathcal B(q^{(n-1)/p},1) \arrow[u, "{[p]}"'] \arrow[l, hook'] & (q^{n/p}/X)^p \arrow[r, maps to] & q\cdot(q^{(n-1)/p}/X)^p. 
		\end{tikzcd}
		\end{equation}
	\end{center} 
	The case of~(\ref{torus explicit [p] map 2}) is very similar.
	
	\subsection{A family of formal models}
	Recall that we have constructed a cover $\mathfrak U_q$ of $\mathbb G_m^{\operatorname{an}}$ depending on a choice of $q\in K^\times$ with $|q|<1$. The affinoid subspaces $\mathcal B(q^n,1)$ that we have used for this admit natural formal models: Namely, consider the $\mathcal O_K$-algebra
	\[L_q^\circ := \mathcal O_K\langle X,Z\rangle/(XZ-q).\]
	This is clearly of topologically finite type over $\mathcal O_K$. It is moreover flat as an $\mathcal O_K$-algebra (this should follow from Lemma 8.2.1 in \cite{Bosch lectures}). For the same reason (or by $L_{q^{-1}} \cong L_q$) we see that \[L_{q^{-1}}^\circ := \mathcal O_K\langle X',Z\rangle/(X'Z-q)\] is a flat topologically finite type $\mathcal O_K$-algebra. Consequently, we have flat formal models 
	\begin{alignat*}{4}
		\mathfrak B(q,1)&:=&& \operatorname{Spf}(L_{q}^\circ), &\quad&\mathfrak B(q,1)_\eta &=& \mathcal B(q,1)\\ 
		\mathfrak B(1,q)&:=&& \operatorname{Spf}(L_{q^{-1}}^\circ), &\quad&\mathfrak  B(1,q)_\eta &=& \mathcal B(1,q)
	\end{alignat*}
	For the glueing maps~(\ref{torus explicit cover glue map 1}) and (\ref{torus explicit cover glue map 2}) it is clear from $a_n/a_{n-1} = b_{n-1}/b_n = q$ that these extend to glueing maps $\mathfrak B(q^n,1)\hookleftarrow \mathfrak B(q^{n-1},1)$ and $\mathfrak B(1,q^{-n})\hookleftarrow \mathfrak B(1,q^{-(n-1)})$. We conclude:

	\begin{lemma}\label{formal model of torus}
		The affine formal schemes $\mathfrak B(q^n,1)$ and $\mathfrak B(1,q^n)$ glue together to a flat formal scheme $\mathfrak G_q$ such that $(\mathfrak G_q)_\eta = \mathbb G_m^{\operatorname{an}}$. In other words, $\mathfrak G_q$ is a formal model for $\mathbb G_m^{\operatorname{an}}$.
	\end{lemma}
	\subsection{A family of formal models for $p$-multiplication}
	As before choose $q\in K^\times$ such that $|q|<1$ and such that there exists a $p$-th root $q^{1/p} \in K$. A closer look at the maps~(\ref{torus explicit [p] map 1}) and~(\ref{torus explicit [p] map 2}) shows that the $[p]$-multiplication extends to a morphism of formal schemes
	\[\mathfrak B(q,1)\xleftarrow{[p]} \mathfrak B(q^{1/p},1):[\mathfrak p]\]
	and similarly for $\mathfrak B(1,q^{-1})$. The diagram~(\ref{diagram showing that [p]-multiplication on torus glues together}) shows that these maps glue to a morphism
	\[[\mathfrak p]: \mathfrak G_{q^{1/p}}\rightarrow  \mathfrak G_q.\]
	By construction, after tensoring $-\otimes_{\mathcal O_K} K$ all morphisms on algebras coincide with those defined in ~(\ref{torus explicit cover glue map 1}),~(\ref{torus explicit cover glue map 2}),~(\ref{torus explicit [p] map 1}), (\ref{torus explicit [p] map 2}) respectively. We conclude:
	\begin{proposition}
		The map $[\mathfrak p]: \mathfrak G_{q^{1/p}}\rightarrow  \mathfrak G_q$ is a formal model of $[p]:\mathbb G_m^{\operatorname{an}}\rightarrow \mathbb G_m^{\operatorname{an}}$.
	\end{proposition}
	We moreover see directly from the construction:
	\begin{proposition}
		The map $[\mathfrak p]: \mathfrak G_{q^{1/p}}\rightarrow  \mathfrak G_q$ reduces mod $p$ to the relative Frobenius map.
	\end{proposition}
	We now have everything together to finish our proof that $(\mathbb G_m^{\operatorname{an}})_\infty$ is perfectoid:
	\begin{proposition}
		The space $\mathbb G_m^{\operatorname{an}}$ has a $[p]$-$F$-model tower. In particular, there exists a perfectoid space $(\mathbb G_m^{\operatorname{an}})_\infty$ such that $(\mathbb G_m^{\operatorname{an}})_\infty \sim \varprojlim_{[p]} \mathbb G_m^{\operatorname{an}}$.
	\end{proposition}
	\begin{proof}
		Since $K$ is perfectoid, we can find $q\in K^\times$ such that $|q|<1$ for which there exist arbitrary $p^n$-th roots. We choose such a $q$ and roots $q^{1/p^n}$ for all $n$. Then the two Propositions above combine to show that 
		\[\dots \xrightarrow{[\mathfrak p]} \mathfrak G_{q^{1/p^2}}\xrightarrow{[\mathfrak p]} \mathfrak G_{q^{1/p}}\xrightarrow{[\mathfrak p]} \mathfrak G_q\]
		is a $[p]$-$F$-model tower. Proposition~\ref{existence of p-F-model tower implies perfectoid} then gives the desired space $(\mathbb G_m^{\operatorname{an}})_\infty$ and shows that it is perfectoid.
	\end{proof}
	
	\subsection{The action of $\overline{T}$}
	The multiplication $\mathbb G_m^{\operatorname{an}}\times \mathbb G_m^{\operatorname{an}}\rightarrow \mathbb G_m^{\operatorname{an}}$ can locally be described in terms of the rigid analytic cover that we have defined above as follows  : Let $a,b \in K^\times$ such that $|a|,|b|\leq 1$, then the multiplication map restricts to
	\begin{equation}
	\begin{alignedat}{2} \label{torus multiplication map}
	\mathcal B(a,1)\times \mathcal B(b,1)&\xrightarrow{m}&& \mathcal B(ab,1)\\
	K\langle X,ab/X\rangle&\rightarrow &&K\langle X,a/X\rangle\widehat{\otimes} K\langle X,b/X\rangle\\
	X&\mapsto&& X\otimes X\\
	ab/X&\mapsto&& b/X\otimes a/X
	\end{alignedat}
	\end{equation}
	and similarly on the $\mathcal B(1,a)\times \mathcal B(1,b)$ for $|a|,|b|\geq 1$. Multiplication on the $\mathcal B(a,1)\times \mathcal B(1,b)$ for $|a|< 1 < |b|$ is more difficult to see on the cover that we have chosen.
	
	The same arguments as in the last section show that the map described in~(\ref{torus multiplication map}) has a flat formal model
	\[\mathfrak B(a,1)\times \mathfrak B(b,1)\rightarrow \mathfrak B(ab,1).\]
	This does \textit{not} mean that multiplication has a formal model $\mathfrak G\times \mathfrak G\rightarrow \mathfrak G$. Indeed, the chosen description has different covers on source and target which in the formal case give rise to different formal schemes (the inversion map $i:\mathbb G_m^{\operatorname{an}}\rightarrow \mathbb G_m^{\operatorname{an}}$ on the other hand does have a formal model). Nevertheless, if we take $a=1$ in the above , we see that we do have an action of the torus $\overline{T}:=\mathfrak B(1,1)$ on each of $\mathfrak B(b,1)$ and $\mathfrak B(1,b)$. Using the formal models from the last section, we conclude:
	
	\begin{proposition}\label{action on formal model of torus}
		For any $q\in K^\times$ with $|q|<1$, the formal torus $\overline{T}:=\mathfrak B(1,1)$ has a natural action on $\mathfrak G_q$ via a map
		\[\mathfrak m:\overline{T}\times \mathfrak G_q\rightarrow \mathfrak G_q.\]
		This map is a formal model of the action of the annulus $\mathcal B(1,1)$ on $\mathbb G_m^{\operatorname{an}}$. Furthermore, this action is compatible with the models for $[p]$ in the sense that if there is a $p$-th root $q^{1/p}\in K$, then the following diagram commutes.
		\begin{center}
		\begin{tikzcd}
			\overline{T}\times \mathfrak G_{q^{1/p}} \arrow[d, "{[p]\times [\mathfrak p]}"'] \arrow[r, "\mathfrak m"] & \mathfrak G_{q^{1/p}} \arrow[d, "{[\mathfrak p]}"] \\
			\overline{T}\times \mathfrak G_{q} \arrow[r, "\mathfrak m"] & \mathfrak G_{q}.
		\end{tikzcd}
		\end{center}
	\end{proposition} 
	\begin{proof}
		The existence of $\mathfrak m$ follows from the above consideration concerning the map~(\ref{torus multiplication map}). The rest is clear from the construction: All adic rings we have used in the construction are $\mathcal O_K$-subalgebras of the affinoid $K$-algebras used to define $\mathbb G_m^{\operatorname{an}}$, so the equalities hold because they hold for $\mathbb G_m^{\operatorname{an}}$.
	\end{proof}
	
	\subsection{The case of general tori}
	By taking products everywhere, all of the statements in this section immediately generalises to split tori:
	\begin{corollary}\label{torus has formal models}
		Let $T$ be a split torus over $K$ of the form $T=(\mathbb G_m^{\operatorname{an}})^d$. Then for any $q\in K^\times$ with $|q|<1$ the formal scheme $\mathfrak T_q := (\mathfrak G_q)^d$ is a formal model of $T$. If there is a $p$-th root $q^{1/p}\in K$, the $p$-multiplication map has a formal model $[\mathfrak p]:\mathfrak T_{q^{1/p}}\rightarrow \mathfrak T_{q}$ that locally on polyannuli is of the form $[\mathfrak p]:\mathfrak B(q^{1/p},1)^d\rightarrow \mathfrak B(q,1)^d$. Moreover this map reduces mod p to the relative Frobenius morphism.
	\end{corollary}
	\begin{corollary}\label{torus has p-F-model tower and has perfectoid tilde-limit}
		Let $T$ be a split torus over $K$, considered as a rigid space. Then $T$ has a $[p]$-$F$-model tower. In particular, there exists a perfectoid space $T_\infty$ such that $T_\infty \sim \varprojlim_{[p]} T$. 
	\end{corollary}
	
	\begin{corollary}\label{action on formal model of torus}
		Let $T$ be any split torus over $K$. For any $q\in K^\times$ with $|q|<1$, the formal completion $\overline{T}$ has a natural action on $\mathfrak T_q$ via a map
		\[\mathfrak m:\overline{T}\times \mathfrak T_q\rightarrow \mathfrak T_q.\]
		This map is a formal model of the action of the annulus $\overline{T}$ on $T$. Furthermore, this action is compatible with the models for $[p]$ in the sense that if there is a $p$-th root $q^{1/p}\in K$, then the map $[\mathfrak p]:\mathfrak T_q^{1/p}\rightarrow \mathfrak T_q$ is semi-linear with respect to $[p]:\overline{T}\rightarrow \overline{T}$.
	\end{corollary} 
	
%%%%%%%%%%%%
%%      Section 4
%%%%%%%%%%%%
	\section{Raynaud extensions as principal bundles of formal and rigid spaces}
	In the following discussion let $A$ be an abelian variety over $K$ of semi-stable reduction. We denote by $N$ the identity component of the N\'eron-model and by $\overline E$ its completion along the special fibre. Then by the theory of Raynaud, $\overline E$ is a formal group that fits into a short exact sequence of formal group schemes
	\begin{equation}\label{formal Raynaud extension}
	0\rightarrow \overline T \rightarrow \overline E \xrightarrow{\pi} \overline{B}\rightarrow 0
	\end{equation}
	where $\overline{B}$ is the completion of an abelian variety $B$ over $K$ of good reduction (we also denote by $B$ the rigid space associated to it), and $\overline{T}$ is the completion of a torus of rank $r$ over $K$.
	After passing to a finite extension of $K$, we can always assume that the torus is split. The rigid generic fibre $\overline{T}_\eta$ of the torus $\overline{T}$ canonically embeds into the torus $T^{\operatorname{an}}$ which again we simply denote by $T$. One can show that this induces a pushout exact sequence in the category of rigid groups, see \S 1 of \cite{BL}. More precisely, there exists a rigid group variety $E$ such that the following diagram commutes and the left square is a pushout.
	\begin{center}
		\begin{equation}
		\begin{tikzcd}
			0 \arrow[r] & \overline{T}_\eta \arrow[d, hook] \arrow[r] & \overline{E}_\eta \arrow[d, hook] \arrow[r] & \overline{B}_\eta \arrow[d,equal] \arrow[r] & 0 \\
			0 \arrow[r] & T \arrow[r] & E \arrow[r] & B \arrow[r] & 0
		\end{tikzcd}
		\end{equation}
	\end{center}
	
	The abelian variety $A$ we started with can then be uniformized in terms of $E$ as follows:
	
	\begin{definition}
		A subset $M$ of a rigid space $G$ is called \textbf{discrete} if the intersection of $M$ with any affinoid open subset of $G$ is a finite set of points.
		Let $G$ be a rigid group, then a \textbf{lattice} in $G$ of rank $r$ is a discrete torsion-free subgroup $M$ of $G$ which is isomorphic to the constant rigid group $\underline{\mathbb Z^r}$. 
	\end{definition}
	
	\begin{proposition}\label{Raynaud uniformisation}
		There exists a lattice $M \subseteq E$ of rank equal to the rank $r$ of the torus for which $E/M$ exists as a rigid space, has a group structure such that $E\rightarrow E/M$ is a rigid group homomorphism, and for which there is a natural isomorphism
		\[A=E/M.\]
	\end{proposition}
	
	Since $M$ is discrete, the local geometry of $A$ is thus determined by the local geometry of $E$. More precisely, we will first study the $[p]$-multiplication tower of $E$ and deduce properties of the $[p]$-multiplication tower of $A$ later.
	
	In order to do so, we would like to study the local geometry of $E$ and $\overline{E}$ via $T$ and $B$. An obstacle in doing this is that the categories of formal or rigid groups are not abelian, which makes working with short exact sequences a subtle issue. Another issue is that one cannot direcly study short exact sequences locally on $T$, $E$ or $B$. An important tool is therefore the following Lemma:

	\begin{lemma}[\cite{BL}, \S 1]\label{formal Raynaud sequence is locally split}
		The short exact sequence (\ref{formal Raynaud extension}) admits local sections, that is there is a cover of $B$ by formal open subschemes $U_i$ such that there exist sections $s:U_i\rightarrow \overline{E}$ of $\pi$. In particular, one can cover $\overline{E}$ by formal open subschemes of the form $\overline{T}\times U_i\hookrightarrow E$.
	\end{lemma}
	\begin{proof}
		This is proved in Proposition A.2.5 in~\cite{rigid geometry of curves}, in terms of the group $\operatorname{Ext}(B,T)$.
	\end{proof}
	
	The last Lemma suggest that instead of considering Raynaud extensions from the abelian category viewpoint, one should consider them as fibre bundles of formal schemes with structure group $T$, or more precisely as principal $T$-bundles of formal schemes, which are also called torsors. This is the language we want to use in the following: We will work with fibre bundles of formal schemes, rigid spaces and schemes. The main technical tool we will need is the associated fibre construction in these settings. For a rigorous  treatment of these we refer to the Appendix {\color{red} which should be replaced by a link to the relevant literature}.
	
	First of all, we note that the sequence~(\ref{formal Raynaud extension}) from the last section gives rise to a principal $\overline{T}$-bundle
	$\overline{E}\rightarrow \overline{B}$. The fact that $E$ is obtained from $\overline{E}_\eta$ via push-out from $\overline{T}_\eta\rightarrow T$ can now conveniently be expressed in terms of the associated fibre bundle by saying that $E_\eta = T\times^{\overline{T}_\eta}\overline{E}_\eta$ in the sense of Definition~\ref{definition of Borel construction}. We have the following description of $[p]$:
	\begin{lemma}\label{p-multiplication is induced from Borel construction}
		The map $[p]:E\rightarrow E$ coincides with the morphism 
		\[[p]\times^{[p]}[p]: T\times^{\overline{T}_\eta}\overline{E}_\eta\rightarrow T\times^{\overline{T}_\eta}\overline{E}_\eta\]
		induced by the different $[p]$-multiplication maps by Proposition~\ref{associated bundle construction in the semi-linear case is a sort of fibered bifunctor}.
	\end{lemma}
	\begin{proof}
		Lemma~\ref{universal property of associated fibre construction in the semilinear case} in light of Remark~\ref{appendix in the case of rigid spaces and schemes} applied to the maps $g=[p]:\overline{T}_\eta\rightarrow \overline{T}_\eta$, $h=[p]:T\rightarrow T$ and $f=[p]:\overline{E}_\eta\rightarrow \overline{E}_\eta$ says that there is a unique morphism of fibre bundles $E\rightarrow E$ making the following diagram commute:
		\begin{center}
			\begin{equation}\label{rigid p-multiplication cube}
			\begin{tikzcd}[row sep = small, column sep = small]
				& T \arrow[rr] &  & E \\
				T \arrow[ru, "{[p]}"] \arrow[rr,crossing over] &  & E \arrow[ru, "\exists!", dotted] &  \\
				& \overline{T}_\eta \arrow[uu, hook] \arrow[rr,crossing over] &  & \overline{E}_\eta \arrow[uu] \\
				\overline{T}_\eta \arrow[ru, "{[p]}"] \arrow[uu, hook] \arrow[rr] &  & \overline{E}_\eta \arrow[ru, "{[p]}"] \arrow[uu] & 
			\end{tikzcd}
			\end{equation}
		\end{center}
		Since $[p]:E\rightarrow E$ is such a map, the Lemma follows.
	\end{proof}

%%%%%%%%%%%%
%%      Section 5
%%%%%%%%%%%%	
	
	\section{Formal models for $E$}
	In this subsection we prove that $E$ admits a $[p]$-$F$-tower model. The first step is to construct a family of formal models for $E$. We do this by using the formal models $\mathfrak T_q$.
	\begin{proposition}
	Let $q\in K^\times$ with $|q|<1$. Let $\mathfrak T_q$ be the formal model from Corollary~\ref{torus has formal models}. Then there is a formal scheme $\mathfrak E_q :=\mathfrak T_q \times^{\overline{T}}\overline{E}$ that is a formal model of the rigid space $E$. Furthermore, there exists a morphism
	\[\mathfrak E_q :=\mathfrak T_q \times^{\overline{T}} \overline{E} \rightarrow \overline{B} \]
	which is a fibre bundle and a formal model of $E\rightarrow B$.
	\end{proposition}
	\begin{proof}
		Recall from Proposition~\ref{action on formal model of torus} that $\mathfrak T_q$ has a $\overline{T}$-action that is a model of the $\overline{T}_\eta$-action on $T$. In particular, the associated fibre construction for the principal $\overline{T}$-bundle $\overline{E}$ gives a fibre bundle $\mathfrak E_q :=\mathfrak T_q \times^{\overline{T}} \overline{E} \rightarrow \overline{B}$. Since $\mathfrak T_q$ is a formal model of $T$, one sees by Lemma~\ref{associated bundle commutes with generic fibre} that this is a formal model of $T\times^{\overline{T}_\eta}\overline{E}_\eta$ which by definition is equal to $E$.
	\end{proof}
	Next we want to construct a model for the $[p]$-multiplication map. Here we can use again that $[p]$ exists on $\overline{E}$ and on $\mathfrak T_{q^{1/p}}$.
	\begin{proposition}\label{formal model of p-multiplication on E}
		Let $q\in K^\times$ be such that $|q|<1$ and assume there exists a $p$-th root $q^{1/p}\in K$. Then there is an affine morphism
		\[[\mathfrak p]:\mathfrak E_{q^{1/p}} \rightarrow  \mathfrak E_{q}\]
		which is a formal model of $[p]:E\rightarrow E$.
	\end{proposition}
	\begin{proof}
		Recall that the multiplication map $[p]:T\rightarrow T$ has a formal model $[\mathfrak p]:\mathfrak T_{q^{1/p}}\rightarrow \mathfrak T_q$ by Corollary~\ref{torus has formal models}. This fits into a commutative diagram
		\begin{center}
			\begin{tikzcd}
				\mathfrak T_{q^{1/p}} \arrow[r, "{[\mathfrak p]}"] & \mathfrak T_q \\
				\overline{T} \arrow[u, hook] \arrow[r, "{[p]}"] & \overline{T} \arrow[u, hook].
			\end{tikzcd}
		\end{center}
		
		Functoriality of the associated fibre construction in the general case, Proposition~\ref{associated bundle construction in the semi-linear case is a sort of fibered bifunctor}, applied to the diagram below then gives a natural map $\mathfrak E_{q^{1/p}}\rightarrow \mathfrak E$ making the diagram commute:
		\begin{center}
			\begin{equation}\label{formal model of p-multiplication cube}
			\begin{tikzcd}[column sep={1.3cm,between origins},row sep={1.3cm,between origins}]
				& \mathfrak T_{q} \arrow[rr] &  & \mathfrak E_q \\
				\mathfrak T_{q^{1/p}} \arrow[ru, "{[\mathfrak p]}"] \arrow[rr] &  & \mathfrak T_{q^{1/p}}\times^{\overline{T}}\overline{E} \arrow[ru, "\exists", dotted] &  \\
				& \overline{T} \arrow[uu] \arrow[rr] &  & \overline{E} \arrow[uu] \\
				\overline{T} \arrow[uu] \arrow[rr] \arrow[ru, "{[p]}"] &  & \overline{E} \arrow[uu] \arrow[ru, "{[p]}"] & 
			\end{tikzcd}
			\end{equation}
		\end{center}
		By Lemma~\ref{p-multiplication is induced from Borel construction} this diagram equals diagram~(\ref{rigid p-multiplication cube}) on the generic fibre. 
	
		To see that the morphism $[\mathfrak p]:\mathfrak E_{q^{1/p}} \rightarrow  \mathfrak E_{q}$ is affine, first note that $[p]:\overline{B}\rightarrow \overline{B}$ is an affine morphism. The map $[\mathfrak p]:\mathfrak T_{q^{1/p}}\rightarrow \mathfrak T_{q}$ is affine by construction, namely by Corollary~\ref{torus has formal models} it is locally on $\mathfrak T_{q}$ of the form $[\mathfrak p]:\mathfrak B(q^{1/p},1)^d\rightarrow \mathfrak B(q,1)^d$. Note that both of these affine open subsets are fixed by the action of $\overline{T}$.
		We conclude from the construction in the proof of Proposition~\ref{associated bundle construction in the semi-linear case is a sort of fibered bifunctor} that the morphism  $[\mathfrak p]:\mathfrak E_{q^{1/p}} \rightarrow  \mathfrak E_{q}$ locally on the target is of the form
		\[[\mathfrak p]:\mathfrak B(q^{1/p},1)^d \times U' \rightarrow \mathfrak B(q,1)^d \times U\]
		for an affine open formal subscheme $U\subseteq \overline{B}$ with affine preimage $U'$ under $[p]:\overline{B}\rightarrow \overline{B}$. This shows that the morphism is affine locally on the target, and therefore is affine.
	\end{proof}
	
	We have thus proved the first part of what we want to show about tilde-limits of $E$:
	\begin{proposition}\label{p-model tower exists for E}
		Let $K$ be perfectoid. Then $E$ has a $[p]$-model tower of the form
		\[\dots \xrightarrow{[\mathfrak p]} \mathfrak E_{q^{1/p^2}}\xrightarrow{[\mathfrak p]} \mathfrak E_{q^{1/p}}\xrightarrow{[\mathfrak p]} \mathfrak E_q\]
		for some $q\in K^\times$. In particular, there exists a space $E_\infty$ such that $E_\infty\sim \varprojlim_{[p]}E$.
	\end{proposition}
	\begin{proof}
		By Proposition~\ref{formal model of p-multiplication on E}, any choice of $q\in K^\times$ with $|q|<1$ for which there exists a compatible system of $p^n$-th roots $q^{1/p^n}\in K^\times$ gives a tower
		\[\dots \xrightarrow{[\mathfrak p]} \mathfrak E_{q^{1/p^2}}\xrightarrow{[\mathfrak p]} \mathfrak E_{q^{1/p}}\xrightarrow{[\mathfrak p]} \mathfrak E_q\]
		that on the generic fibre equals $\dots\xrightarrow{[p]} E\xrightarrow{[p]} E$. This is the desired $[p]$-model tower.
	\end{proof}
	
	We are now ready to prove the main result of this note, namely that $E_\infty$ is perfectoid.
	\begin{framed}
	\begin{theorem}\label{p-F-model tower exists for E}
		Let $K$ be perfectoid. Then the $[p]$-model tower from Proposition~\ref{p-model tower exists for E}
		\[\dots \xrightarrow{[\mathfrak p]} \mathfrak E_{q^{1/p^2}}\xrightarrow{[\mathfrak p]} \mathfrak E_{q^{1/p}}\xrightarrow{[\mathfrak p]} \mathfrak E_q\]
		 is already a $[p]$-$F$-model tower.
		In particular, the corresponding space $E_\infty$ is perfectoid.
	\end{theorem}
	\end{framed}
	\begin{proof}
	
	It suffices to prove that for any $q\in K^\times$ with $|q|<1$ and a $p$-th root $q^{1/p}$, the map $[\mathfrak p]:\mathfrak E_{q^{1/p}}\xrightarrow{} \mathfrak E_q$ upon reduction mod $p$ factors through relative Frobenius.
	
	In the following we denote reduction of a formal scheme by a $\sim$ over the formal scheme, for example the reductions of $\overline{T}$, $\overline{E}$ and $\mathfrak T$ are denoted by $\tilde{T}$, $\tilde{E}$ and $\tilde{\mathfrak{T}}$.
	
		
	Recall that $[\mathfrak p]:\mathfrak E_{q^{1/p}}\xrightarrow{} \mathfrak E_q$ was constructed using the $[p]$-multiplication cube in diagram~(\ref{formal model of p-multiplication cube}) and functoriality of the associated bundle. 	
	Also recall that all statements we have used about fibre bundles also hold when we replace formal schemes over $\mathcal O_K$ by schemes over $\mathcal O_K/p$, and formation of the associated bundle commutes with this reduction. In particular,
	\[\tilde{\mathfrak{E}}_q = \tilde{\mathfrak T}_q\times^{\tilde{T}}\tilde E.\]
	By  Corollary~\ref{torus has formal models}, the model of the multiplication map $[\mathfrak p]:\mathfrak T_{q^{1/p}} \rightarrow \mathfrak T_{q}$ reduces to relative Frobenius over $p$. In particular, we have a natural isomorphism
	\[\tilde{\mathfrak T}_{q^{1/p}}^{(p)} \cong \tilde{\mathfrak T}_{q}\]
	and we can identify $\tilde{\mathfrak T}_{q^{1/p}}^{(p)} = \tilde{\mathfrak T}_{q}$ in the following. The same is true for $\tilde{T}^{(p)} = \tilde{T}$.
	
	Since $\tilde{E}$ and $\tilde{T}$ are group schemes, the reduction of $[p]$ on them factors through the relative Frobenius maps $F_E$ and $F_E$ respectively. By functoriality of relative Frobenius ("Frobenius commutes with any map") we have a commutative diagram
	\begin{center}
		\begin{tikzcd}
			\tilde{E} \arrow[r, "F_{\tilde E}"] &  \tilde{E}^{(p)} \\
			\tilde{T} \arrow[r, "F_{\tilde T}"] \arrow[u, hook] &  \tilde{T}^{(p)} \arrow[u, hook].
		\end{tikzcd}
	\end{center}
	In other words, $F_{\tilde E}$ is a $F_{\tilde T}$-linear morphism of fibre bundles. Again by functoriality of Frobenius we also have a commutative diagram
	\begin{center}
	\begin{tikzcd}
	\tilde{\mathfrak T}_{q^{1/p}} \arrow[r, "F_{\tilde { \mathfrak T}}"]  & \tilde{\mathfrak T}_{q^{1/p}}^{(p)} \\
	\tilde T \arrow[r, "F_{\tilde T}"] \arrow[u,hook] & \tilde{T}^{(p)} \arrow[u,hook].
	\end{tikzcd}
	\end{center}
	
	By Proposition~\ref{associated bundle construction in the semi-linear case is a sort of fibered bifunctor}, we thus obtain a natural morphism
	\[F_{\tilde{\mathfrak{T}}}\times^{F_{\tilde{T}}} F_{\tilde
		E}:\tilde{\mathfrak T}_{q^{1/p}}\times^{\tilde T}\tilde E \rightarrow \tilde{\mathfrak T}_{q^{1/p}}^{(p)}\times^{\tilde T^{(p)}}\tilde E^{(p)}. \]
	Using the explicit description of $F_{\tilde{\mathfrak{T}}}\times^{F_{\tilde{T}}} F_{\tilde
		E}$ in the proof of Proposition~\ref{associated bundle construction in the semi-linear case is a sort of fibered bifunctor}, we easily check that this morphism is just the relative Frobenius of $\tilde{\mathfrak{E}}_{q^{1/p}}$: This is a consequence of the fact that relative Frobenius on the fibre product $\tilde{T}\times \tilde{U}$ for any $\tilde{U}\subseteq \tilde{B}$ is just the product of the relative Frobenius morphisms of $\tilde T$ and $\tilde U$, and thus the morphisms $\theta_i$ from Lemma~\ref{equivalent characterisation of morphisms of principal T-bundles} are all trivial. 
	
	But this means that again by Proposition~\ref{associated bundle construction in the semi-linear case is a sort of fibered bifunctor}, the reduction of the formal model of the $p$-multiplication cube in diagram~\ref{formal model of p-multiplication cube} admits the following factorisation:
	
	\begin{center}
		\begin{tikzcd}[column sep={1cm,between origins},row sep={1cm,between origins}]
			&  &  &  & \tilde{\mathfrak T}_{q} \arrow[rrr] &  &  & \tilde{\mathfrak E}_{q} \\
			&  & \tilde{\mathfrak T}_{q}^{(p)} \arrow[rru,equal] \arrow[rrr] &  &  & \tilde{\mathfrak E}^{(p)}_{q^{1/p}} \arrow[rru, dotted] &  &  \\
			\tilde{\mathfrak T}_{q^{1/p}} \arrow[rru, "\, F"'] \arrow[rrr] &  &  & \tilde{\mathfrak E}_{q^{1/p}} \arrow[rru, "\, F"'] &  &  &  &  \\
			&  &  &  & \tilde{T} \arrow[rrr] \arrow[uuu] &  &  & \tilde{E} \arrow[uuu] \\
			&  & \tilde{T}^{(p)} \arrow[rrr] \arrow[rru,equal] \arrow[uuu] &  &  & \tilde{E}^{(p)} \arrow[rru] \arrow[uuu] &  &  \\
			\tilde{T} \arrow[rrr] \arrow[uuu] \arrow[rru, "\, F"'] &  &  & \tilde{E} \arrow[rru, "\, F"'] \arrow[uuu] &  &  &  & 
				\end{tikzcd}
	\end{center}
	Since the composed maps $\tilde{E}\rightarrow \tilde{E}$ on the bottom right, $\tilde{T}\rightarrow \tilde{T}$ on the bottom left and  $\tilde{\mathfrak T}_{q^{1/p}}\rightarrow \tilde{\mathfrak T}_{q}$ on the upper left by construction are the reductions of the respective $p$-multiplication maps $[p]$, the functoriality of the associated bundle construction in Proposition~\ref{associated bundle construction in the semi-linear case is a sort of fibered bifunctor} implies that the two maps on the upper right compose to the reduction of $[\mathfrak p]\times^{[p]}[p]$. But $[\mathfrak p]\times^{[p]}[p]$ is equal to $[\mathfrak p]:\mathfrak E_{q^{1/p}}\xrightarrow{} \mathfrak E_q$ by definition of the latter. 	This completes the proof that the reduction of $[\mathfrak p]:\mathfrak E_{q^{1/p}}\xrightarrow{} \mathfrak E_q$ factors through the relative Frobenius on $\tilde{\mathfrak E}_{q^{1/p}}$.
	
	The conclusion that $E_\infty$ exists and is perfectoid then follows from Proposition~\ref{existence of p-F-model tower implies perfectoid}.
	\end{proof}
	
	\section{The case of abelian varieties with semi-stable reduction}
	Let $K$ be a complete non-archimedean field with ring of integers $\mathcal O_K$. Let $A$ be an abelian variety over $K$ of dimension $g$. Recall from Proposition~\ref{Raynaud uniformisation} that associated to $A$ we have a Raynaud extension $E$ which is an extension of a rigid torus $T$ of rank $r$ which we assume to be split by an abelian variety $B$ of good reduction. Moreover there is a lattice $M\subseteq E$ of rank $r$ such that $A=E/M$.
	In this chapter, we want to prove the main result of this write-up:
	
	\begin{theorem}\label{main theorem}
		Assume that $A$ admits a constant partial anticanonical $\Gamma_0(p^\infty)$-structure (this will be defined later, but we remark already that this condition is always satisfied if $K$ is algebraically closed). Then there is a perfectoid space $A_\infty$ such that
				\[A_\infty \sim \varprojlim_{[p]} A.\]
	\end{theorem}
	
	The proof of Theorem~\ref{main theorem} will be completed in several steps over the following sections. Our strategy is to describe the $[p]$-multiplication tower on $E/M$ locally in terms of the $[p]$-multiplication tower of $E$.
	As a first step towards this goal, in the following section, we want to give a cover of $E/M$ by subspaces of $E$ that behaves well under $[p]$-multiplication.
	
	\subsection{Covering $A$ by subspaces of $E$}
	As a first step we recall how to relate the lattice $M$ to a Euclidean lattice in $\mathbb R^r$, cf \S2.7 and \S6.2 in  \cite{rigid geometry of curves}. On the level of points, $\mathbb{G}_m^{\operatorname{an}}$ has an absolute value map
	\[|-|:\mathbb{G}_m^{\operatorname{an}}(K)=K^\times\rightarrow \mathbb R^\times, \quad x\mapsto |x|\]
	which induces the following group homomorphism from the torus $T$:
	\[|-|:T(K)=(K^\times)^r\rightarrow (\mathbb R^\times)^r, \quad (x_1,\dots,x_n)\mapsto (|x_1|,\dots,|x_n|)\]
	Since when working with lattices we prefer additive notation, we also consider the map
	\[\ell:T(K)=(K^\times)^r\rightarrow \mathbb R^r, \quad x_1,\dots,x_n\mapsto (-\log |x_1|,\dots,-\log |x_n|).\]
	Note that this map has dense image by our assumption that $K$ is perfectoid.
	
		The formal torus $\overline{T}$ on $K$-points corresponds to $\overline{T}_\eta(K) = (\mathcal O_K^\times)^r$ and is thus in the kernel of $-\log$. We can therefore extend $\ell$ to $E(K)$ as follows: Locally over an open subspace $U\subseteq B$ we have $E|_U = T\times^{\overline{T}_\eta}\overline{E}_\eta|_{U}$ and we define $\ell$ by projection from the first factor. The different $E|_U$ are then glued on intersections using the $\overline{T}_\eta$-action on $T$. But since $\ell$ on $T$ is invariant under the $\overline{T}_\eta$-action, the maps glue together to a group homomorphism 
	\[\ell:E(K)\rightarrow \mathbb R^r.\]
	
	Since $A=E/M$ is proper, the lattice $M$ is sent by $-\log$ to an Euclidean lattice $\Lambda \subset \mathbb R^r$ of full rank $r$ (see Proposition 6.1.4 in \cite{rigid geometry of curves}). In particular, this induces an isomorphism of discrete torsionfree groups
	\[\ell:M\xrightarrow{\sim} \Lambda\subseteq\mathbb R^r.\]
	
	The idea is now that one can understand the quotient $E/M$ in terms of the quotient $\mathbb R^r/\Lambda$. We are going to make this precise in the following:
	
	In the space $\mathbb R^r$, we can now choose a $d\in \mathbb R_{> 0}^r$ and a cuboid with center at the origin
	\[S(d) = \{(x_1,\dots,x_r)\in \mathbb R^r | |x_i|\leq d_i \}\]
	that intersects $\Lambda$ only in $0\in \Lambda$ and
	 $q_1,\dots,q_r\in K$ such that $|q_i|=\exp(-d_i)$. We denote by $\mathcal B(q,q^{-1})$ the affinoid open multi-annulus $T$ centered at $1$ of radii $|q_i|< 1 < |q_i|^{-1}$ in every direction.
	
	\begin{lemma}\label{cube around origin gives local chart for E/M}
		The inverse image $\ell^{-1}(S(d))\subseteq E(K)$ is the underlying set of the admissible open subset $E(q):=\mathcal B(q,q^{-1})\times^{\overline{T}_\eta}\overline{E}_\eta$.
	\end{lemma}
	\begin{proof}
		One shows this first for the map $T\rightarrow \mathbb R^d$, where it is clear that the preimage is $\mathcal B(q,q^{-1})$. This is also described in \S 6.4 of \cite{FvdP}. The statement for $\mathcal B(q,q^{-1})\times^{\overline{T}_\eta}\overline{E}_\eta$ follows by direct inspection on local trivialisations $\mathcal B(q,q^{-1})\times U$ for $U\subseteq B$.
	\end{proof}
	
	
	Note that the map $\mathbb R^d\rightarrow \mathbb R^d/\Lambda$ maps $S(d)$ bijectively onto its image. 
	Lemma~\ref{cube around origin gives local chart for E/M} says that we can use $\mathcal B(q,q^{-1})\times^{\overline{T}_\eta}\overline{E}_\eta$ as a chart for $E/M$ around the origin.
	
	In order to obtain charts around other points of $E/M$, we simply need to consider translations: Recall that for every $c\in T(K)$, the translation map
	\[T\xrightarrow{\cdot c}T\]
	is an isomorphism of rigid spaces that sends the unit $1$ to $c$. We denote the image of any admissible open set $U$ under translation by $c\cdot U$.
	
	\begin{lemma}\label{cube around point gives local chart for E/M}
		With notation as before, let $c \in T(K)$ be any point such that $l(c)=s$. Then the inverse image $\ell^{-1}(s+S(d))\subseteq E(K)$ of the translation of $S(d)$ by $s$ is the underlying set of the admissible open subset $E(c,q) := (c\cdot \mathcal B(q,q^{-1}))\times^{\overline{T}_\eta}\overline{E}_\eta \subseteq E$. We can choose $c\in T(K)$ and $q\in T(K)$ in such a way that they admit arbitrary $p^n$-th roots.
	\end{lemma}
	
	\begin{proof}
		Since $\ell$ commutes with the translations
		\begin{center}
			\begin{tikzcd}
				T(K) \arrow[r, "\ell"] \arrow[d, "\cdot c"'] & \mathbb R^r \arrow[d, "+s"] \\
				T(K) \arrow[r, "\ell"] & \mathbb R^r,
			\end{tikzcd}
		\end{center}
		the first part is an immediate consequence of Lemma~\ref{cube around origin gives local chart for E/M}. For the second, note that for any $c'\in T(K)$ with $l(c')=l(c)$ and $q\in T(K)$ with $l(q')=l(q)$ we have $c\cdot \mathcal B(q,q^{-1}) = c'\cdot \mathcal B(q,q^{-1})$, and thus $E(c,q)=E(c',q')$. The statement therefore follows from $K$ being perfectoid, for instance using $\mathcal O_K^\flat = \varprojlim_{x\mapsto x^p}\mathcal O_K$.
	\end{proof}
    
    \begin{definition}\label{defininition of cuboid}
		We call spaces of the form $E(c,q)\subseteq E$ \textbf{cuboids} centered at $c$. More precisely they are locally a cuboid   $c\cdot\mathcal B(q,q^{-1})\subseteq T$ times an admissible open subset of the abelian variety $B$. 
	\end{definition}
    
	\begin{lemma}\label{an admissible cover of A by cuboids of E}
	There exist finitely many admissible open cuboids $E_1,\dots,E_k\subseteq E$ which map isomorphically to $A=E/M$ and which cover $A$ admissibly. 
	
	One can reconstruct $A$ from any such cover by glueing $E_1,\dots,E_k$ as follows: By construction, for any $E_i$ the translates $q\cdot E_i$ by $q\in M$ are pairwise disjoint and we thus have a canonical projection $\pi$ from the union $\cup_{m\in M} (m\cdot E_i)\subseteq E$ to $E_i$.	Let $E_{ij}:=(\bigcup_{m\in M} m\cdot E_i)\cap E_j \subseteq E$. Then we glue $E_j$ to $E_i$ via the map
	\[E_{ij}\rightarrow \bigcup_{m\in M} q\cdot E_i \xrightarrow{\pi} E_i.\]
	
	\begin{figure}
		\includestandalone[]{glue-cover-tikzpicture}%
		% or use \input{mytikz}
		\caption{Given two charts $E_i$ and $E_j$, the chart $E_j$ is glued to $E_i$ along intersections with all translates of $E_i$ by $q\in M$.}
		\label{glue-cover-tikzpicture}
	\end{figure}
	
	\end{lemma} 
	\begin{proof}
	Since $\mathbb R^r/\Lambda$ is compact, we can find finitely many $s_1,\dots,s_k$ and $d_1,\dots,d_k \in \mathbb{R}^r_{>0}$ such that $\mathbb R^r/\Lambda$ is covered by the $s_i+S(d_i)$. When we choose corresponding $c_1,\dots,c_k \in T(K)$ and $q_1,\dots,q_k$ as in Lemma~\ref{cube around point gives local chart for E/M}, then the corresponding $E_i:=E(c_i,q_i)$ are an atlas of $A=E/M$ by admissible open subsets of $E$.
	
	In order to reconstruct $A$, note that $\cup_{m\in M} q\cdot E_i$ is precisely the preimage of $E_i$ under the projection $E\rightarrow E/M$. In particular, the subspace $E_{ij}$ is precisely the preimage of $E_i$ under the composition $E_j\hookrightarrow E \rightarrow E/M$. In other words, the subspace $E_{ij}\subseteq E_j$ is the intersection of $E_i$ and $E_j$ when considered as subspaces of $A$. This shows that as charts of $A$, the spaces $E_i$ and $E_j$ are glued via $E_{ij}$ as described.
	\end{proof}
	Finally, we need some control about what happens to the cubes under $[p]$-multiplication. Recall from Lemma~\ref{cube around point gives local chart for E/M} that we can always assume that $c$ admits $p$-th roots.
	\begin{lemma}\label{pullback of cuboid is cuboid}
		Let $c^{1/p}$ be a $p$-th root of $c$ and let $q^{1/p}$ be a $p$-th root of $q$ in $(K^\times)^r$. Then under $[p]:E\rightarrow E$, the admissible open $E(c_i,q)$ pulls back to the admissible open $E(c_i^{1/p},q^{1/p})$.
	\end{lemma}
	\begin{proof}
		It is clear that under $[p]:T\rightarrow T$, the admissible open cuboid $c\cdot \mathcal B(q,q^{-1})$ centered at $c$ pulls back to $c^{1/p}\cdot \mathcal B(q^{1/p},q^{-1/p})$. Note that this is independent of the choices of $c^{1/p}$ and $q^{1/p}$. Now recall that in terms of fibre bundles, multiplication $[p]:E\rightarrow E$ is
		\[[p]\times^{[p]}[p]: T\times^{\overline{T}_\eta}\overline{E}_\eta\rightarrow T\times^{\overline{T}_\eta}\overline{E}_\eta \]
		by Lemma~\ref{p-multiplication is induced from Borel construction}. Thus $(c\cdot \mathcal B(q,q^{-1}))\times^{\overline{T}_\eta}\overline{E}_\eta$ pulls back to $(c^{1/p}\cdot \mathcal B(q^{1/p},q^{-1/p}))\times^{\overline{T}_\eta}\overline{E_\eta}$.
	\end{proof}

	\subsection{The two towers}
	In this section we want to separate the $[p]$-multiplication of $A$ into two different towers, which we think of as being a ``ramified'' tower and an ``\'etale'' tower. Of course in characteristic $0$ both towers will actually be \'etale, and these words are only meant to describe the behaviour compared to $[p]:E\rightarrow E$.
	
	The ``ramified'' tower only exists under an additional condition, which we want to briefly discuss now:
	\begin{definition}
		By a \textbf{partial anticanonical $\Gamma_0(p^\infty)$-structure} on $A$ we mean a choice of subgroups $D_n\subseteq A[p^n]$ of rank $p^{rn}$ for all $n$ such that $pD_{n+1}=D_n$ and $D_n+E[p^n]=A[p^n]$. 
		
		Note that the conditions imply that $D_n$ is necessarily finite \'etale, and after a finite extension of $K$ is isomorphic to the constant group $\underline{(\mathbb Z/p^n\mathbb Z)^{r}}$. We say that a partial anticanonical $\Gamma_0(p^\infty)$-structure is \textbf{constant} if such an isomorphism exists over $K$ for all $n$.
	\end{definition}
	The name is chosen because if $B$ admits a canonical subgroup (that is, satisfies a condition on its Hasse invariant), the choice of a (full) anticanonical $\Gamma_0(p^\infty)$-structure on $A$ is equivalent to the choice of a partial anticanonical $\Gamma_0(p^\infty)$-structure on $A$ and an anticanonical $\Gamma_0(p^\infty)$-structure on $B$. Note however that $A$ may have a partial anticanonical subgroup even if $B$ does not have a canonical subgroup. For instance, it is clear that $A$ always has a constant partial anticanonical subgroup if $K$ is algebraically closed.
	
	We will use this definition only via the following equivalent description:
	\begin{lemma}
		Existence and choice of a constant partial anticanonical $\Gamma_0(p^\infty)$-structure on $A$ are equivalent to existence and choice of lattices $M^{1/p^n}\subseteq E$ defined over $K$ such that $[p]:E\rightarrow E$ restricts to isomorphisms $M^{1/p^{n+1}}\rightarrow M^{1/p^n}$ for all $n$.
	\end{lemma}
	\begin{proof}
		Given lattices $M^{1/p^{n+1}}$ as in the lemma, we obtain a split partial anticanonical $\Gamma_0(p^\infty)$ structure by setting $D_n:=M^{1/p^{n+1}}/M$. This is because any such lattice gives a splitting of the short exact sequence $0\rightarrow E[p^n]\rightarrow A[p^n]\rightarrow M/M^p \rightarrow 0$.
		
		Conversely, given subgroups $D_n\subseteq A[p^n]$ that form a partial anticanonical $\Gamma_0(p^\infty)$ structure, we recover $M^{1/p^n}$ as the kernel of $E\rightarrow A\rightarrow A/D_n$. The finite group $M^{1/p^n}/M$ is then constant if and only if $M^{1/p^n}$ is a lattice.
	\end{proof}
	
		
	\begin{assumption}\label{assumption that points of D_n are defined over K}
		From now on we will assume that $A$ admits a constant partial anticanonical $\Gamma_0(p^\infty)$-structure. Let us fix a choice of such a structure, that is we choose
		 lattices $M^{1/p^n}$ defined over $K$ that map isomorphically to $M$ under $[p^n]$.
	\end{assumption}
		
		The quotient $A/D_n = E/M^{1/p^n}$ is then another abelian variety over $K$ and the quotient map $E/M\rightarrow E/M^{1/p^n}$ is an isogeny of degree $p^{2gn}$  through which  $[p^n]:A\rightarrow A$ factors: 
		\begin{center}
			\begin{equation}\label{factorisation of [p] over E/M}
			\begin{tikzcd}
				& E/M^{1/p} \arrow[rd, "{[p]}"] &  \\
				E/M \arrow[rr, "{[p]}"] \arrow[ru,"v"] &  & E/M.
			\end{tikzcd}
			\end{equation}
		\end{center}
		We think of these maps as being an analogue of Frobenius and Verschiebung, which is why we denote the left map by $v$.
		Putting everything together, the $[p]$-multiplication tower splits into two towers
		\begin{center}
		\begin{equation}\label{p-multiplication tower of E/M splits into vertical and horizontal tower}
		\begin{tikzcd}[column sep={1.2cm,between origins},row sep={0.8cm,between origins}]
			\ddots \arrow[rd] &  &  & \vdots \arrow[d] &  & \vdots \arrow[d] \\
			& E/M \arrow[rr,"v"] \arrow[rrdd, "{[p]}"'] &  & E/M^{1/p} \arrow[rr,"v"] \arrow[dd,"{[p]}"] &  & E/M^{1/p^2} \arrow[dd,"{[p]}"] \\
			&  &  &  &  &  \\
			&  &  & E/M \arrow[rrdd, "{[p]}"'] \arrow[rr,"v"] &  & E/M^{1/p} \arrow[dd,"{[p]}"] \\
			&  &  &  &  &  \\.
			&  &  &  &  & E/M
		\end{tikzcd}
		\end{equation}
		\end{center}
		Since each quotient $M^{1/p^n}/M$ is a finite \'etale group scheme, all horizontal maps are finite \'etale. The vertical tower on the other hand fits into a second commutative diagram of rigid groups which compares it to the $[p]$-tower of $E$:
		
		\begin{center}
		\begin{equation}\label{F-tower for E/M}
		\begin{tikzcd}
			& \vdots \arrow[d] & \vdots \arrow[d] & \vdots \arrow[d] &  \\
			0 \arrow[r] & M^{1/p^2} \arrow[d, "\cong"] \arrow[r] & E \arrow[d, "{[p]}"] \arrow[r] & E/M^{1/p^2} \arrow[d, "{[p]}"] \arrow[r] & 0 \\
			0 \arrow[r] & M^{1/p} \arrow[d, "\cong"] \arrow[r] & E \arrow[d, "{[p]}"] \arrow[r] & E/M^{1/p} \arrow[d, "{[p]}"] \arrow[r] & 0 \\
			0 \arrow[r] & M \arrow[r] & E \arrow[r] & E/M \arrow[r] & 0
		\end{tikzcd}
		\end{equation}
		\end{center}
		
		\subsection{Constructing a limit of the vertical tower}
		
		Our first step is to show that the tower on the right has a perfectoid tilde-limit.
		Recall from Lemma~\ref{an admissible cover of A by cuboids of E} that $E/M$ can be covered by admissible open subspaces $E_1,\dots,E_k\subseteq E$ which map isomorphically onto an admissible open via $E\to E/M$. Denote by $E_i^{1/p^n}\subseteq E$ the pullback along $[p^n]:E\rightarrow E$. Also denote by $E_{ij}^{1/p^n}\subseteq E$ the pullback of $E_{ij}$. Because of our assumption, we can then reconstruct the space $E/M^{1/p^n}$ from the $E_i^{1/p^n}$ as follows:
		\begin{lemma}\label{compatible cuboid charts for the tower over E/M}
			\leavevmode
			\begin{enumerate}
		\item The restriction to $E_{i}^{1/p^n}\subseteq E$ of $E\rightarrow E/M^{1/p^n}$ is an isomorphism onto its image. In particular, we can view $E_{i}^{1/p^n}$ as a chart of $E/M^{1/p^n}$, and this is the preimage of $E_i$ under $E/M^{1/p^n}\rightarrow E/M$.
		\item The collection of  $E_{i}^{1/p^n}$ is an atlas for  $E/M^{1/p^n}$. 
		\item We can reconstruct $E/M^{1/p^n}$ from glueing the $E_{i}^{1/p^n}$ along the $E_{ij}^{1/p^n}$.
		\item The map $[p^n]:E/M^{1/p^n}\rightarrow E/M$ can be glued from the restrictions of $[p^n]:E\rightarrow E$ to $E_{i}^{1/p^n}\rightarrow E_{i}$, that is these maps commute with the glueing maps on $E_{ij}^{1/p^n}$.
		\end{enumerate}
		The situation is thus like in Figure~\ref{transform-glue-cover-tikzpicture}.
		\end{lemma}
			\begin{figure}
				\includestandalone{transform-glue-cover-tikzpicture}
				\caption{Illustration of how $[p]:E/M^{1/p^n}\rightarrow E/M$ can be glued from the maps $E_j^{1/p^n}\rightarrow E_j$. Here $E_j$ on bottom and $E_j^{1/p^n}$ on top are represented by the grey cuboids in the middle. On the left they are embedded into $E$ whereas on the right they are considered as charts for $E/M$ and $E/M^{1/p}$.}
				\label{transform-glue-cover-tikzpicture}
			\end{figure}
		\begin{proof}
		The first part follows because the map on the left of diagram~\ref{F-tower for E/M} is an isomorphism. The second follows from pulling back the $E_i$ along $E/M^{1/p^n}\rightarrow E/M$, using that the diagram commutes.	We thus obtain an admissible cover by cuboids $E_1^{1/p^n},\dots,E_k^{1/p^n}$ of $E/M^{1/p^n}$. The glueing maps that can be used to reconstruct $E/M^{1/p^n}$ by glueing along subspaces $E_{ij}^{1/p^n}$ then exist by the second part of Lemma~\ref{an admissible cover of A by cuboids of E}: Note that we can apply Lemma~\ref{an admissible cover of A by cuboids of E} also to $E/M^{1/p^n}$ because of Assumption~\ref{assumption that points of D_n are defined over K}.
		
		Finally, in order to see that one can glue together the map $[p]:E/M^{1/p^n}\rightarrow E/M$ from the $E_i^{1/p^n}$, use that intersection of cuboids are again cuboids, and so $E_{ij}^{1/p^n}$ is a disjoint union of cuboids. It then follows from Lemma~\ref{pullback of cuboid is cuboid} that $E_{ij}$ pulls back under $[p]$ to the intersection $E_{ij}^{1/p^n}\subseteq E/M^{1/p^n}$. That $[p]$ commutes with the glueing maps is clear because we know from diagram (\ref{factorisation of [p] over E/M}) that $[p]:E\rightarrow E$ descends to a morphism $[p]:E/M^{1/p^n}\rightarrow E/M$.
		\end{proof}
		We are now ready to prove:
		\begin{proposition}\label{explicit construction of vertical tilde-limit}
			There is a perfectoid space $E/M^{1/p^\infty}$ such that\[E/M^{1/p^\infty}\sim \varprojlim_{n}E/M^{1/p^n}. \]
		\end{proposition}
		\begin{proof}
		 Denote by $E_i^{1/p^\infty}$ the pullback of $E_i\subseteq E$ to $E_\infty$. This is an open subspace of a perfectoid space and hence perfectoid. Moreover, by Proposition 2.4.3 in \cite{SW} we have  \[ E_i^{1/p^\infty}\sim \varprojlim E_i^{1/p^n}.\] 
		Given two different $E_i$, $E_j$, we know by Lemma~\ref{compatible cuboid charts for the tower over E/M} that at every step in the tower, the pullbacks $E_i^{1/p^n}$ and $E_j^{1/p^n}$ to $E/M^{1/p^n}$ intersect in  $E_{ij}^{1/p^n}$.
		We can thus glue the $E_i^{1/p^\infty}$ along pullbacks $E_{ij}^{1/p^\infty}$ of the intersections $E_{ij}=E_i\cap E_j$ to $E_\infty$ and thus obtain a perfectoid space $E/M^{1/p^\infty}$. This is a tilde-limit for $\varprojlim_{[p]}E/M^{1/p^n}$ because by construction it is so locally, and the definition of tilde-limits in Definition 2.4.1 of \cite{SW} is local on the source.
		\end{proof}
	 
	\subsection{Constructing a limit of the horizontal tower}
	In order to construct a tilde-limit for $\varprojlim A$, we can now use that the horizontal maps in diagram~(\ref{p-multiplication tower of E/M splits into vertical and horizontal tower}) are all finite \'etale.
	\begin{lemma}\label{horizontal map is covering map}
		For any $0\leq m\leq n$, the preimage of $E_i^{1/p^n}$ from Lemma~\ref{compatible cuboid charts for the tower over E/M} under the horizontal map $v^m:E/M^{1/p^{m}}\rightarrow E/M^{1/p^n}$ is isomorphic to $p^{r(n-m)}$ disjoint copies of $E_i^{1/p^n}$. More canonically, it can be described as the isomorphic image of $D_n/D_{m}\times E_i^{1/p^n}$ under the multiplication on the abelian variety $E/M^{1/p^n}$.
	\end{lemma}
	\begin{proof}
		By the first part of Lemma~\ref{compatible cuboid charts for the tower over E/M}, we know that the preimage of $E_i^{1/p^n}$ under the projection $E\rightarrow E/M^{1/p^n}$ is a disjoint union of translates of $E_i^{1/p^n}$ by $M^{1/p^{n}}$. The result then follows because $M^{1/p^{n}}/M^{1/p^{m}} = D_n/D_m =  \underline{(\mathbb Z/p^{n-m}\mathbb Z)^r}$.
	\end{proof}
	We also record the following immediate consequence:
	
	\begin{lemma}\label{preimage of E_i under p^n is disjoint copies}
		The preimage of $E_i$ under $[p^n]:A\rightarrow A$ is isomorphic to $p^{rn}$ disjoint copies of $E_i^{1/p^n}$. More canonically, we can describe the preimage as the isomorphic image of  $D_n \times E_i^{1/p^n}$ under the multiplication $A\times A\rightarrow A$. The situation is thus as in figure~\ref{local-glueing-[p]-tikzpicture}.
	\end{lemma}
	\begin{figure}
		\includestandalone{local-glueing-[p]-tikzpicture}
		\caption{Illustration of how $[p]:E/M\rightarrow E/M$ factors in a part that is ``ramified'' (the vertical tower) and a part that is ``\'etale'' (the horizontal tower) with respect to our cover.}
		\label{local-glueing-[p]-tikzpicture}
	\end{figure}
	\begin{proof}
		This follows from the first part of Lemma~\ref{compatible cuboid charts for the tower over E/M} combined with Lemma~\ref{horizontal map is covering map} in the case of $m=0$.
	\end{proof}
	
	
	\begin{lemma}\label{squares in double tower are pullbacks}
		The squares in diagram~(\ref{p-multiplication tower of E/M splits into vertical and horizontal tower}) are all pullback diagrams.
		\begin{center}
			\begin{tikzcd}
				E/M^{1/p^n} \arrow[d, "{[p]}"] \arrow[r,"v"] & E/M^{1/p^{n+1}} \arrow[d, "{[p]}"] \\
				E/M^{1/p^{n-1}} \arrow[r,"v"] & E/M^{1/p^n}
			\end{tikzcd}
		\end{center}
	\end{lemma}
	\begin{proof}
		This can for instance be checked locally: The admissible open subset $E_i^{1/p^n}\subseteq E/M^{1/p^n}$ from Lemma~\ref{compatible cuboid charts for the tower over E/M} is pulled back to $E_i^{1/p^{n+1}}$ under the vertical map $[p]:E/M^{1/p^{n+1}}\rightarrow E/M^{1/p^n}$. The preimage of $E_i^{1/p^n}$ under the horizontal map $E/M^{1/p^{n-1}}\rightarrow E/M^{1/p^n}$ is $p^r$ disjoint copies of $E_i^{1/p^n}$ by Lemma~\ref{horizontal map is covering map}. The pullback of $E_i^{1/p^n}$ to the upper right is thus $p^r$ disjoint copies of $E_i^{1/p^{n+1}}$, which is clearly the fibre product.
	\end{proof}
	\begin{lemma}\label{horizontal etale map pulls back to vertical limit}
	The horizontal maps in diagram~(\ref{p-multiplication tower of E/M splits into vertical and horizontal tower}) induce natural finite \'etale morphisms $v:E/M^{1/p^\infty}\rightarrow E/M^{1/p^\infty}$ that fit into Cartesian diagrams
	\begin{center}
		\begin{tikzcd}
			E/M^{1/p^\infty} \arrow[d, "{}"] \arrow[r,"v^m"] & E/M^{1/p^\infty} \arrow[d, "{}"] \\
			E/M^{1/p^{n-m}} \arrow[r,"v^m"] & E/M^{1/p^{n}}
		\end{tikzcd}
	\end{center}
	In particular, the preimage of $E_i^{1/p^\infty}$ under $v^m$ is isomorphic to $p^{rm}$ copies of $E_i^{1/p^\infty}$.
	\end{lemma}
	\begin{proof}
		Since $E/M \rightarrow E/M^{1/p}$ is finite \'etale, the fibre product with $E/M^{1/p^\infty} \rightarrow E/M^{1/p}$ exists and is perfectoid by Proposition~7.10 of \cite{perfectoid IHES}.
		
		The universal property of the fibre product then gives a unique map 
		\[E/M^{1/p^\infty}\rightarrow E/M\times_ {E/M^{1/p}}E/M^{1/p^\infty}\]
		making the natural diagrams commute.
		On the other hand, using Lemma~\ref{squares in double tower are pullbacks} we see that the fibre product has compatible maps into the vertical inverse system over $E/M$. Since by Proposition~2.4.5 of \cite{SW} the perfectoid tilde-limit $E/M^{1/p^\infty}$ is universal for maps from perfectoid spaces to the inverse system, we obtain a unique map into the other direction.
	\end{proof}
	We thus obtain a pro-\'etale tower
	\begin{equation}\label{proetale tower in the vertical limit}
	\dots \xrightarrow{v}E/M^{1/p^\infty}\xrightarrow{v} E/M^{1/p^\infty}\xrightarrow{v} E/M^{1/p^\infty}
	\end{equation}
	which we think of as being a kind of vertical ``limit'' of diagram~\ref{p-multiplication tower of E/M splits into vertical and horizontal tower}. One can always take the limit of such a tower, as the following Lemma asserts:
	\begin{lemma}\label{proetale over perfectoid has perfectoid tilde limit}
		Let $X$ be a perfectoid space and $(X_i)_{i\in I}$ be a pro-\'etale cover of $X$. Then there is a perfectoid space $X_\infty$ such that
		\[X_\infty \sim \varprojlim_{i \in I}X_i\]
	\end{lemma}
	\begin{proof}
		This follows from Lemma~4.6 of \cite{p-adic Hodge} and the discussion in the paragraph before it.
	\end{proof}
	
	We conclude from this that the tower in equation~\ref{proetale tower in the vertical limit} has a perfectoid limit $(E/M^{1/p^\infty})_\infty$. 
	
	\subsection{The diagonal tower: proof of the main theorem}
	We want to show that this space is in fact a tilde-limit of the $[p]$-multiplication tower. In other words, this says that the horizontal tilde-limit of the vertical tilde-limits in diagram~\ref{p-multiplication tower of E/M splits into vertical and horizontal tower} is also a diagonal tilde-limit.
	However, while it follows immediately from Proposition~2.4.5 of \cite{SW} that $(E/M^{1/p^\infty})_\infty$ is universal for maps of perfectoid spaces $Y$ into the inverse system $\varprojlim_{[p]}E/M$, it does not follow immediately from the definition that it is also the tilde-limit of this inverse system. The problem is that for a space $X$ to be a tilde-limit $X\sim \varprojlim_{[p]}X_i$, it suffices for $X$ to satisfy a property locally on some affinoid cover. But it is not clear a priori how such affinoid covers of $(E/M^{1/p^\infty})_\infty$ and  $(E/M^{1/p^\infty})$ are related.
	In our situation, on the other hand, we can use that we have a good understanding of the local behaviour of the maps in the tower in terms of the cuboids $E_i$ to solve this problem:
	
	\begin{proposition}\label{tilde limit of tilde limits of partial towers is tilde limit of whole tower}
		The perfectoid space  $A_\infty:=(E/M^{1/p^\infty})_\infty$ is a tilde-limit of $\varprojlim_{[p]}A$.	It is independent up to unique isomorphism of the choice of partial anticanonical $\Gamma_0(p^\infty)$-structure, but it remembers the choice as a pro-finite \'etale closed subgroup $D_\infty \subseteq A_\infty$. 
	\end{proposition}
	\begin{proof}
	It is clear from $(E/M^{1/p^\infty})_\infty \sim \varprojlim E/M^{1/p^\infty}$ and $(E/M^{1/p^\infty})\sim \varprojlim E/M^{1/p^n}$ that the underlying topological space is indeed isomorphic to $\varprojlim_{[p]}|E/M|$.

	In order to show that it is a tilde-limit of $\varprojlim_{[p]}E/M$, it thus suffices to give an explicit cover of $(E/M^{1/p^\infty})_\infty$ by open affinoids satisfying the tilde-limit property. To this end,
	first note that the pro-\'etale system $D_n=M^{1/p^n}/M$ over $M/M=\operatorname{Sp}K$ under multiplication maps $[p]$ has a perfectoid tilde-limit $D_\infty$ by Lemma~\ref{proetale over perfectoid has perfectoid tilde limit}. Since the $D_n$ are all constant perfectoid groups $\cong \mathbb \underline{(\mathbb Z/p^r\mathbb Z)^n}$, it is clear that $D_\infty$ is isomorphic to the constant perfectoid group $\underline{\mathbb Z_p^r}$.
	
	Recall that by construction of $(E/M^{1/p^\infty})$ we have a cover of $E/M$ by open subsets $E_i$ that pull back to perfectoid open subspaces $E_i^{1/p^\infty}$ for which $E_i^{1/p^\infty}\sim \varprojlim E_i^{1/p^n}$. Moreover, by the second part of Proposition~\label{horizontal etale map pulls back to vertical limit} we know that the pullback of $E_i^{1/p^\infty}$ to $(E/M^{1/p^\infty})_\infty$ is the disjoint union of $|\mathbb Z_p^r|$ copies of $E_i^{1/p^\infty}$, which more precisely can be described as $D_\infty \times E_i^{1/p^\infty}$. But it is clear from Lemma~\ref{preimage of E_i under p^n is disjoint copies} that the image of any such isomorphic copy of $E_i^{1/p^\infty}\hookrightarrow   (E/M^{1/p^\infty})_\infty$ in the inverse system $\dots\xrightarrow{[p]}A\xrightarrow{[p]} A$ is isomorphic to the inverse system $\dots\rightarrow E_i^{1/p}\rightarrow E_i$. Since $E_i^{1/p^\infty}\sim \varprojlim E_i^{1/p^n}$, this shows that $(E/M^{1/p^\infty})_\infty$ is locally the tilde-limit of the $[p]$-tower on $A$ as desired.
	
	That $A_\infty$ is independent of the $\Gamma_0(p^\infty)$-structure up to unique isomorphism is a consequence of the universal property of the perfectoid tilde-limit. That $D_\infty$ can be described as a closed subgroup of $A_\infty$ is then clear from the local description of $A_\infty$ as $D_\infty \times E_i^{1/p^\infty}$.
	\end{proof}
	
	This finishes the proof of Theorem~\ref{main theorem}.
	
	Note that while the approach via cuboids $E_i$ may look a bit technical on first glance, it has the advantage of giving an explicit description of $(E/M)_\infty$ as being glued from pieces of $E_\infty$ by glueing data that is controlled by the lattices $M^{1/p^n}$. This might be interesting for applications, and in particular for computing the tilt. 
	
	
	\section{Limits of the covering maps}
	
	Over the course of the proof, we have used three different towers: The tower $\dots \rightarrow E\xrightarrow{[p]} E$, the tower $\dots \rightarrow E/M \xrightarrow{[p]} E/M$ and the tower $\dots \rightarrow E/M^{1/p} \xrightarrow{[p]} E/M$ (for the latter we also needed the additional Assumption~\ref{assumption that points of D_n are defined over K}). The three are related by covering maps which fit into a commutative diagram of towers
	\begin{center}
		\begin{tikzcd}
			E \arrow[d, "{[p]}"] \arrow[r] & E/M \arrow[d, "{[p]}"] \arrow[r] & E/M^{1/p^{n+1}} \arrow[d, "{[p]}"] \\
			E \arrow[r] & E/M \arrow[r] & E/M^{1/p^n}
		\end{tikzcd}
	\end{center}
	As we have seen in the last sections, all three towers have perfectoid tilde-limits, that we have denoted by $E_\infty$, $A_\infty$ and $E/M^{1/p^\infty}$.
	
	By Proposition~\ref{perfectoid tilde limit is perfectoid group in a functorial way} the map $\pi:E\rightarrow A=E/M$ in the limit induces a natural group homomorphism $\iota:E_\infty \rightarrow A_\infty$. A similar universal property argument shows that we obtain a natural group homomorphism $A_\infty \rightarrow E/M^{1/p^\infty}$. In this section we want to study these morphisms more closely. Throughout we are going to retain Assumption~\ref{assumption that points of D_n are defined over K}.
	
	We start with the case of $E\rightarrow E/M^{1/p^\infty}$:
	\begin{proposition}\label{the morphism E->E/M^{1/p^n} in the limit}
		Denote by $M_\infty\cong M$ the perfectoid tilde-limit of the tower
		\begin{center}
			\begin{tikzcd}
				\arrow[r] & M^{1/p^2} \arrow[r, "{[p]}"',"\sim"] & M^{1/p} \arrow[r, "{[p]}"',"\sim"] & M.
			\end{tikzcd}
		\end{center}
		There is a natural map $M_\infty \rightarrow E_\infty$ with respect to which we can interpret $M_\infty$ as a lattice of rank $r$ in $E_\infty$. The map fits into a short exact sequence of perfectoid groups
		\[0\rightarrow M_\infty\rightarrow E_\infty \rightarrow E/M^{1/p^\infty} \rightarrow 0.\]
	\end{proposition}
	\begin{proof}
		The map $M_\infty\rightarrow E_\infty$ is induced by the universal property of the perfectoid tilde-limit as usual. 
		In order to see that the sequence is exact, we need to see that the first map is a kernel of the second, and the second map is a categorical quotient of the first. To this end, we first analyse the morphism locally: The projections to the inverse system fit into a commutative diagram
		
		\begin{center}
			\begin{tikzcd}[row sep = {0.75cm,between origins}]
				M_\infty \arrow[r] \arrow[d,no head] & E_\infty \arrow[r] \arrow[d,no head] & E/M^{1/p^\infty} \arrow[d,no head] \\
				\vdots \arrow[d] & \vdots \arrow[d] & \vdots \arrow[d] \\
				M^{1/p} \arrow[dd, "{[p]}"] \arrow[r] & E \arrow[dd, "{[p]}"] \arrow[r] & E/M^{1/p} \arrow[dd, "{[p]}"] \\
				&  &  \\
				M \arrow[r] & E \arrow[r] & E/M
			\end{tikzcd}
		\end{center}
		
		Let us consider the preimages of $E_i \subseteq E/M$ under these morphisms: By Lemma~\ref{an admissible cover of A by cuboids of E} we see that the pullback to $E$ is $\bigsqcup_{q\in M} qE_i$. We can also see this as the isomorphic image of $M\times E_i$ under the multiplication map $E\times E\rightarrow E$. 
		
		The pullback of $E_i$ along $[p^n]:E/M^{1/p^n}\rightarrow E/M$ is $E_i^{1/p^n}$ as we have seen in Lemma~\ref{compatible cuboid charts for the tower over E/M}. The same Lemma shows that the pullback of this to $E$ is $\bigsqcup_{q\in M^{1/p^n}}qE_i^{1/p^n}=M^{1/p^n}\times E_i^{1/p^n}$.
		
		We then see that the pullback to $E_\infty$ is $M_\infty\times E_i^{1/p^\infty}$. 
		By construction of $E/M^{1/p^\infty}$ in the proof of Proposition~\ref{explicit construction of vertical tilde-limit}, the pullback of $E_i$ to $E/M^{1/p^\infty}$ is $E_i^{1/p^\infty}$. All in all, we obtain a pullback diagram
		\begin{center}
		\begin{equation}\label{pullback diagram covering map E inf to E/M^1/p^inf }
		\begin{tikzcd}[column sep={1.3cm,between origins},row sep={1.3cm,between origins}]
			& E_\infty \arrow[dd] \arrow[rr] &  & E/M^{1/p^\infty} \arrow[dd] \\
			M_\infty\times E_i^{1/p^\infty} \arrow[dd] \arrow[rr] \arrow[ru, hook] &  & E_i^{1/p^\infty} \arrow[dd] \arrow[ru, hook] &  \\
			& E \arrow[rr] &  & E/M \\
			M\times E_i \arrow[rr] \arrow[ru, hook] &  & E_i \arrow[ru, hook] & 
		\end{tikzcd}
		\end{equation}
		\end{center}
		We conclude that $E_\infty \rightarrow E/M^{1/p^\infty}$ is a principal $M_\infty$-torsor of perfectoid groups. It is then clear that $M_\infty$ is the preimage of $0\in E/M^{1/p^\infty}$, from which one easily verifies that $M_\infty\hookrightarrow E_\infty$ has the universal property of the kernel.
		Similarly, $E_\infty \rightarrow E/M^{1/p^\infty}$ has the universal property of the cokernel: Given any perfectoid group $H$ and a group homomorphism $E_\infty\rightarrow H$ that is trivial on $M_\infty$, the restriction $M_\infty\times E_i^{1/p^\infty}\rightarrow H$ gives a natural map $E_i^{1/p^\infty}\rightarrow H$. Since by construction of $E/M^{1/p^\infty}$ the spaces $E_i^{1/p^\infty}$ and $E_j^{1/p^\infty}$ are glued on $E_{ij}^{1/p^\infty}$ using translation by $M$, these glue together to a morphism of $E/M^{1/p^\infty}$.
	\end{proof}
	
	
	The case of $\iota:A_\infty \rightarrow E/M^{1/p^\infty}$ is similar:
	\begin{proposition}\label{the morphism A->E/M^{1/p^n} in the limit}
		The subgroups $D_n= M^{1/p^n}/M\subseteq A$ in the limit give rise to a perfectoid group $D_\infty\sim \varprojlim_{[p]}D_n$ which is a subgroup $D_\infty \subseteq A_\infty$ in a natural way.
		If we assume Assumption~\ref{assumption that points of D_n are defined over K}, this fits into a short exact sequence of perfectoid groups
		\[0\rightarrow D_\infty \rightarrow A_\infty\rightarrow E/M^{1/p^\infty}\rightarrow 0.\]
	
	\end{proposition}
	\begin{proof}
		The tilde-limit $D_\infty$ of the pro-\'etale tower $\varprojlim_{[p]}D_n$ exists by Lemma~\ref{proetale over perfectoid has perfectoid tilde limit}. The group structure and the map to $A_\infty$ follow from the universal property as usual. Under assumption~\ref{assumption that points of D_n are defined over K} is is easy to see that $D_\infty$ is in fact non-canonically isomorphic to the constant group $\underline{\mathbb Z_p^r}$.
		
		We can now argue similarly as in the proof of the last Proposition: at finite level we obtain short exact sequences 
		\begin{center}
			\begin{tikzcd}
				D_{n+1} \arrow[equal,r] \arrow[d] & M^{1/p^{n+1}}/M \arrow[r] \arrow[d, "{[p]}"] & E/M \arrow[r] \arrow[d, "{[p]}"] & E/M^{1/p^{n+1}} \arrow[d, "{[p]}"] \\
				D_n \arrow[r,equal] & M^{1/p^{n}}/M \arrow[r] & E/M \arrow[r] & E/M^{1/p^n}
			\end{tikzcd}
		\end{center}
		The preimages of $E_i$ under these maps are
		\[D_n\times E_i^{1/p^n}\rightarrow E_i^{1/p^n} \]
		by Lemma~\ref{preimage of E_i under p^n is disjoint copies}. In the limit the pullback is then
		\[D_\infty \times E_i^{1/p^\infty}\rightarrow E_i^{1/p^\infty} \]
		which shows that $A_\infty \rightarrow E/M^{1/p^\infty}$ is a $D_\infty$-torsor. As in the last proof, this implies that the sequence in the Proposition is a short exact sequence.
	\end{proof}
	
	Finally, we consider the case of $\iota:E\rightarrow A=E/M$. While the limits of the last two towers were fibre bundles again, the map $\iota$ shows quite a different behaviour and on the opposite is an injective group homomorphism. While this may seem strange at first, it is actually what one might expect following the intuition of the following example:
	\begin{remark}
		To illustrate why this phenomenon occurs, consider the following inverse system of abstract groups:
	\begin{center}
	\begin{tikzcd}[row sep = {0.65cm,between origins}]
		& \arrow[dd,dotted] & \arrow[dd,dotted] & \arrow[dd,dotted] &  \\
		&&\\
		0 \arrow[r] & \mathbb Z \arrow[r] \arrow[dd, "{[p]}"] & \mathbb R \arrow[r] \arrow[dd, "{[p]}"] & \mathbb R/\mathbb Z \arrow[dd, "{[p]}"] \arrow[r] & 0 \\
		&&\\
		0 \arrow[r] & \mathbb Z \arrow[r] & \mathbb R \arrow[r] & \mathbb R/\mathbb Z \arrow[r] & 0
	\end{tikzcd}
	\end{center}
	While at infinite level the maps on the right are all covering maps, in the inverse limit the homological algebra of $\varprojlim$ produces a long exact sequence
	\begin{center}
	\begin{tikzcd}
		0 \arrow[r] & 0 \arrow[r] & \mathbb R \arrow[r] & \varprojlim_{[p]}\mathbb R/\mathbb Z \arrow[r] & \varprojlim^1_{[p]}\mathbb Z = \mathbb Z_p/\mathbb Z\arrow[r] & 0.
	\end{tikzcd}
	\end{center}
	So in the limit the covering map becomes the kernel of a map to $\mathbb Z_p/\mathbb Z$.
	\end{remark}
	
	For perfectoid groups the homological algebra argument of course doesn't apply in the same way. Nevertheless, if we retain Assumption~\ref{assumption that points of D_n are defined over K}, we can again use the explicit covers of the last section to show that the situation is very similar as in the remark:
	\begin{proposition}\label{the morphism E->A in the limit}
		The homomorphism $\iota:E_\infty \rightarrow A_\infty$ is an open and closed immersion. Moreover, there is a natural map $\rho:A_\infty \rightarrow D_\infty/M_\infty$ which is a quotient of $\iota$ in the category of perfectoid groups. Here $D_\infty/M_\infty$ is a constant group which is uncanonically isomorphic to the constant perfectoid group $\underline{(\mathbb Z_p/\mathbb Z)^r}$. We thus obtain a short exact sequence of perfectoid groups
		\[0\rightarrow E_\infty \rightarrow A_\infty \rightarrow D_\infty/M_\infty\rightarrow 0\]
		which is uncanonically split on the level of the underlying perfectoid spaces.
	\end{proposition}
	\begin{proof}
		To see that $\iota$ is open and closed, it suffices to prove this locally. So it suffices to consider the induced map over the preimages of $E_i\subseteq A$ under the projection $A_\infty\rightarrow A$.
		
		Recall that the map $\iota:E_\infty \rightarrow A_\infty$ arises by a universal property from the inverse system
		\begin{center}
			\begin{tikzcd}[row sep = {0.75cm,between origins}, column sep = {2cm,between origins}]
				\vdots \arrow[d] & \vdots \arrow[d] & \vdots \arrow[d] \\
				E \arrow[dd, "{[p]}"'] \arrow[r] & A \arrow[dd, "{[p]}"'] \arrow[r, "v"] & E/M^{1/p} \arrow[dd, "f"'] \\
				&  &  \\
				E \arrow[r] & A \arrow[r,equal] & E/M.
			\end{tikzcd}
		\end{center}
		Using Lemmas~\ref{compatible cuboid charts for the tower over E/M} and \ref{horizontal map is covering map} we see that the pullback of this diagram to $E_i\subseteq E/M$ is
		\begin{center}
			\begin{tikzcd}[row sep = {0.85cm,between origins}, column sep = {3cm,between origins}]
				\vdots \arrow[d] & \vdots \arrow[d] & \vdots \arrow[d] \\
				\bigsqcup\limits_{q\in M^{1/p}} q\cdot E_i^{1/p^n} \arrow[dd] \arrow[r] & \bigsqcup\limits_{q\in D_n} E_i^{1/p^n} \arrow[dd] \arrow[r] & E_i^{1/p^n} \arrow[dd] \\
				&  &  \\
				\bigsqcup\limits_{q\in M} q\cdot E_i \arrow[r] & E_i \arrow[r,equal] & E_i
			\end{tikzcd}
		\end{center}
		where recall that we write $D_n$ for the group $M^{1/p^n}/M \subseteq A[p^n]$.
		In particular, at every level in the tower over $E_i$ the maps are trivial covering maps. This means that in order to understand them we only need to keep track of the maps on index sets, which are
		\begin{center}
		\begin{tikzcd}[row sep = {1.5cm,between origins}, column sep = {2cm,between origins}]
			 \arrow[d,"\cong"',dotted] &  \arrow[d, "{[p]}",dotted] & \arrow[d,dotted] \\
			M^{1/p} \arrow[d, "{[p]}","\cong"'] \arrow[r] & M^{1/p}/M \arrow[d, "{[p]}"] \arrow[r] & 1 \arrow[d] \\
			M \arrow[r] & 1 \arrow[r] & 1
		\end{tikzcd}
		\end{center}
		Recall that we have defined $D_\infty = \varprojlim_{[p]} D_n = \varprojlim_{[p]}M^{1/p^n}/M$ and $M_\infty \cong \varprojlim_{[p]}M^{1/p^n} \hookrightarrow E_\infty$. In the limit, the diagram gives rise to a natural map $M_\infty\hookrightarrow D_\infty$. Since $M \cong \underline{ \mathbb Z^r}$ and $D_\infty \cong \underline{\mathbb Z_p^r}$ uncanonically, the quotient $D_\infty/M_\infty$ exists and is uncanonically isomorphic to \[D_\infty/M_\infty\cong\underline{(\mathbb Z_p/\mathbb Z)^r}.\]

		Using the map $M_\infty\hookrightarrow D_\infty$ on indices, we see that the pullback to $E_i\subseteq E/M$ of the morphisms $\iota:E_\infty \rightarrow A_\infty\rightarrow E/M^{1/p^\infty}$ is given by the diagram
		\begin{center}
			\begin{tikzcd}[row sep = {1.4cm,between origins}, column sep = {1.4cm,between origins}]
			& E_\infty \arrow[rr,"\iota"] \arrow[dd] &  & A_\infty \arrow[rr] \arrow[dd] &  & E/M^{1/p^\infty} \arrow[dd] \\
			\bigsqcup\limits_{M_\infty} E_i^{1/p^\infty} \arrow[dd] \arrow[rr] \arrow[ru, hook] &  & \bigsqcup\limits_{D_\infty} E_i^{1/p^\infty} \arrow[dd] \arrow[rr] \arrow[ru, hook] &  & E_i^{1/p^\infty} \arrow[dd] \arrow[ru, hook] &  \\
			& E \arrow[rr] &  & A \arrow[rr] &  & E/M \\
			\bigsqcup_{M} E_i \arrow[rr] \arrow[ru, hook] &  & E_i \arrow[rr] \arrow[ru, hook] &  & E_i \arrow[ru, hook] & .
				\end{tikzcd}
		\end{center}
	But since $M_\infty\hookrightarrow D_\infty$ is injective, the map $\iota:\bigsqcup_{M_\infty} E_i^{1/p^\infty} \rightarrow \bigsqcup_{D_\infty} E_i^{1/p^\infty}$ is clearly open and closed. This shows that $\iota:E_\infty \rightarrow A_\infty$ is open and closed.
	
	In order to see that $\iota$ has a quotient, note that the last diagram implies that the morphism
	\[D_\infty\times E_\infty \rightarrow A_\infty\]
	induced from multiplication on $A_\infty$ is surjective. More precisely, for any point of $x\in D_\infty$, the translate $x\cdot E_\infty$ inside $A_\infty$ is \textit{equal} to $E_\infty$ if $x\in M_\infty \hookrightarrow E_\infty$ and is \textit{disjoint} from $E_\infty$ otherwise. This shows that any choice of a set of coset representatives $(x_i)$ of $D_\infty/M_\infty$ or equivalently a set-theoretic section  $x:D_\infty/M_\infty\rightarrow D_\infty$ gives a morphism of perfectoid spaces
	\begin{equation}\label{uncanonical splitting map of embedding of E_inf into A_inf}
	\bigsqcup_{i\in D_\infty/M_\infty} x_i\cdot E_\infty\xrightarrow{\sim} A_\infty
	\end{equation}
	which is an isomorphism because it is bijective and each $x_iE_\infty \hookrightarrow A_\infty$ is an open immersion. Moreover, since $\iota:E_\infty\rightarrow A_\infty$ is a group homomorphism, the multiplication $A_\infty\times A_\infty \rightarrow A_\infty$ locally on the left hand side of this isomorphism restricts to a morphism
	\begin{center}
	\begin{tikzcd}
		E_\infty\times E_\infty \arrow[d, "m"] \arrow[r,"{(\cdot x_i,\cdot x_j)}","\sim"'] & x_iE_\infty\times x_jE_\infty \arrow[d] \arrow[r] & A_\infty\times A_\infty \arrow[d, "m"] \\
		E_\infty \arrow[r,"{\cdot x_ix_j}","\sim"'] & x_ix_jE_\infty \arrow[r] & A_\infty
	\end{tikzcd}
	\end{center}
	commuting with the translation maps.
	
	Now since $D_\infty/M_\infty$ is a constant group, there is a natural locally constant morphism
	\[\bigsqcup_{i\in D_\infty/M_\infty} x_i\cdot E_\infty \rightarrow D_\infty/M_\infty\]
	which is independent of the choice of the $x_i$.
	Commutativity of the diagram shows that it moreover composes with equation~\ref{uncanonical splitting map of embedding of E_inf into A_inf} to a group homomorphism $\rho:A_\infty \rightarrow D_\infty/M_\infty$. It is then clear from~\ref{uncanonical splitting map of embedding of E_inf into A_inf} that this map has the universal property of a quotient for $\iota:E_\infty \rightarrow A_\infty$: Given any perfectoid group $H$ and a homomorphism $g:A_\infty\rightarrow H$ such that $g\circ\iota = 0$, the map $g$ must be constant on any $x_iE_\infty$ and thus factors uniquely through the morphism $D_\infty/M_\infty \rightarrow H$, $i\mapsto g(x_i)$.
	
	That $\iota$ is the kernel of $\rho$ follows from the fact that $E_\infty$ is the pullback of $0\in D_\infty/M_\infty$ under $\rho$. This shows that the sequence in the Proposition is exact.
	
	That the short exact sequence is split on the level of perfectoid spaces follows from the projection maps $\bigsqcup_{i\in D_\infty/M_\infty} x_i\cdot E_\infty \rightarrow E_\infty$, which depend on the uncanonical choice of a set-theoretic section of the map of abstract groups underlying $D_\infty \rightarrow D_\infty/M_\infty$.
	\end{proof}
	
	We summarise all this in the following:
	\begin{theorem}
		We have a commutative diagram with short exact rows and columns
		\begin{center}
			\begin{tikzcd}[column sep = {2cm,between origins}]
				& 0 \arrow[d] & 0 \arrow[d] &  &  \\
				0 \arrow[r] & M_\infty \arrow[d] \arrow[d] \arrow[r] & E_\infty \arrow[d] \arrow[r] & E/M^{1/p^\infty} \arrow[d,equal] \arrow[r] & 0 \\
				0 \arrow[r] & D_\infty \arrow[r] \arrow[d] & A_\infty \arrow[d] \arrow[r] & E/M^{1/p^\infty} \arrow[r] & 0 \\
				& D_\infty/M_\infty \arrow[r,equal] \arrow[d] & D_\infty/M_\infty \arrow[d] &  &  \\
				& 0 & 0 &  & 
			\end{tikzcd}
		\end{center}
		Moreover, the square on the upper right is a pullback as well as a pushout square. In particular, we have a natural short exact sequence of perfectoid groups
		\[0\rightarrow M_\infty\rightarrow D_\infty \times E_\infty \rightarrow A_\infty\rightarrow 0\]
		where the map on the left is the diagonal embedding of $M_\infty$ into $D_\infty\times E_\infty$. In particular, we can describe $A_\infty$ as the quotient $(D_\infty\times E_\infty)/M_\infty$.
		
	\end{theorem}
	\begin{proof}
		The diagram of short exact sequences follow from Proposition~\ref{the morphism E->E/M^{1/p^n} in the limit}, Proposition~\ref{the morphism A->E/M^{1/p^n} in the limit} and Proposition~\ref{the morphism E->A in the limit}.
		That the square is Cartesian is an immediate consequence of the universal property of the kernel.
		That the square satisfies the pushout universal property can be verified using the isomorphism in equation~\ref{uncanonical splitting map of embedding of E_inf into A_inf}.
		
		The universal property of the pushout then implies that $A_\infty$ has the universal property of the quotient of the diagonal map $M\rightarrow D_\infty\times E_\infty$.
	\end{proof}
	
	\appendix
	\section{Fibre bundles of formal and rigid spaces}
	In this chapter we review the theory of fibre bundles with structure group $T$ and in particular of principal $T$-bundles in the setting of formal and rigid geometry.
		
	
	
	In this chapter we denote by $T$ a commutative formal group scheme over $\mathcal O_K$. We denote the multiplication map by $m:T\times T\rightarrow T$. By a $T$-action on a formal scheme $X$ we mean a morphism $m_X:T\times X\rightarrow X$ such that the usual associativity diagram commutes. 
	\begin{definition}
		By a \textbf{$T$-linear map} of schemes $X$ and $Y$ with $T$-actions we mean a morphism $\phi:X\rightarrow Y$ such that the following diagram commutes
		\begin{center}
			\begin{tikzcd}
				T\times X \arrow[d, "m_X"] \arrow[r, "\operatorname{id}_T\times \phi"] & T\times Y \arrow[d, "m_Y"] \\
				X \arrow[r, "\phi"] & Y
			\end{tikzcd}
		\end{center}
		We denote by $\mathbf{FormAct}_T$ the category of formal schemes with action by $T$.
	\end{definition}
	
	
	The definition of a principal $T$-bundle is just what we get when we take the definition of a principal $G$-bundle and replace the category of topological spaces by the category of formal schemes.
	\begin{notation}
		In the following, if $\pi:E\rightarrow B$ is a morphism of formal schemes, then for a formal open subscheme $U\subseteq B$ we denote $E|_U:=\pi^{-1}(U)\subseteq E$.
	\end{notation}
	\begin{definition}\label{definition principal T-bundle}
		Let $T$ be a formal group scheme. Let $F$ be a formal scheme with an action $m:T\times F\rightarrow F$.
		A morphism $\pi:E\rightarrow B$ of formal schemes is called a \textbf{fibre bundle with fibre $F$ and structure group $T$} if there is a cover $\mathfrak U$ of $B$ of open formal subschemes $U_i\subseteq B$ with isomorphisms $\varphi_i:F\times U_i \xrightarrow{\sim} E|_{U_i}$ which satisfy the following conditions:
		\begin{enumerate}[label=(\alph*)]
			\item For every $U_i\in \mathfrak U$, the following diagram commutes:
			\begin{center}
				\begin{tikzcd}
					F\times U_{i} \arrow[r, "\varphi_i"] \arrow[rd, "p_2"] & E|_{U_{i}} \arrow[d, "\pi"] & \phantom{T\times U_{ij}} \\
					& U_{i} & 
				\end{tikzcd}
			\end{center}
			\item For every two $U_i,U_j\in \mathfrak U$ with intersection $U_{ij}$, the commutative diagram
			\begin{center}
				\begin{tikzcd}
					F\times U_{ij} \arrow[r, "\varphi_i"] \arrow[rd, "p_2"] & E|_{U_{ij}} \arrow[d, "\pi"] & F\times U_{ij} \arrow[ld] \arrow[l, "\varphi_j"'] \\
					& U_{ij} & 
				\end{tikzcd}
			\end{center}
			produces an isomorphism $\phi_{ij}:=\varphi_j^{-1}\circ\varphi_i: F\times U_{ij}\rightarrow F\times U_{ij}$ with the following property: There exists a morphism $\psi_{ij}:U_{ij}\rightarrow T$ such that
			\[\phi_{ij}=F\times U_{ij} \xrightarrow{\psi_{ij}\times \operatorname{id}\times\operatorname{id}} T\times F\times U_{ij}\xrightarrow{m\times \operatorname{id}} F\times U_{ij}\]
		\end{enumerate}
	\end{definition}
	\begin{definition}
		When we take $F$ equal to the formal scheme $T$ with the action on itself by left multiplication, then a fibre bundle $\pi:E\rightarrow B$ with fibre $T$ and structure group $T$ is called a \textbf{principal $T$-bundle}. This is also called a $T$-torsor.
	\end{definition}
	
	\begin{example}
		For the short exact sequence $0\rightarrow \overline{T}\rightarrow \overline{E}\xrightarrow{\pi} B\rightarrow 0$ from the last section, $\overline{E}\xrightarrow{\pi} B$ defines a principal $\overline{T}$-bundle by Lemma~\ref{formal Raynaud sequence is locally split}. Moreover, for any formal open subscheme $U\subseteq B$, the map $E|_U\rightarrow U$ is still a principal $\overline{T}$-bundle. This is what we mean when we say that the notion of principal $\overline{T}$-bundles is better suited for studying $E$ locally on $B$ than the notion of short exact sequences is.
	\end{example}
	
	The morphism $\phi_{ij}$ from condition (b) is fully determined by the morphism $\psi_{ij}:U_{ij}\rightarrow T$. By a glueing argument, one shows:
	\begin{lemma}\label{equivalent characterisation of principal $T$-bundle}
		Suppose we are given formal schemes $F$ and $B$ and a formal group scheme $T$ with an action on $F$. Then fibre bundles $\pi:E\rightarrow B$ with fibre $F$ and structure group $T$ are equivalent to the data (up to refinement) of a cover $\mathfrak U$ by formal open subschemes and morphisms $\psi_{ij}:U_{ij}\rightarrow T$ for all $U_i,U_j\in \mathfrak U$ that satisfy the cocycle condition $\psi_{jk}\cdot \psi_{ij}=\psi_{ik}$, by which we mean that the following diagram commutes:
		
		\begin{center}\begin{equation}		\label{cocycle condition of fibre bundle}
			\begin{tikzcd}
			U_{ijk} \arrow[r, "\psi_{ij}\times\psi_{jk}"] \arrow[d,equal] & T\times T \arrow[d, "m"] \\
			U_{ijk} \arrow[r, "\psi_{ik}"] & T.
			\end{tikzcd}
			\end{equation}
		\end{center}
	\end{lemma}
	
	In order to define the category of  fibre bundles, we also need the following:
	\begin{lemma}
		Let $E\rightarrow B$ be a fibre bundle with fibre $F$ and structure group $T$. With notations like in Definition~\ref{definition principal T-bundle} we have a natural $T$-actions on $F\times U_{i}$ when we let $T$ act trivially on $U_{i}$. These actions glue together to a $T$-action on $E$.
	\end{lemma}
	\begin{proof}
		This is immediate from condition (b).
	\end{proof}
	\begin{definition}
		Let $\pi:E\rightarrow B$ be a fibre bundle with fibre $F$ and structure group $T$ and let $\pi':E'\rightarrow B'$ be a fibre bundle with fibre $F'$ and structure group $T$. Then a \textbf{morphism of fibre bundles} $f:(E',B',\pi')\rightarrow (E,B,\pi)$ is a commutative diagram of formal schemes
		\begin{center}
			\begin{tikzcd}
				E' \arrow[d] \arrow[d, "f_E"] \arrow[r, "\pi'"] & B' \arrow[d, "f_B"] \\
				E \arrow[r, "\pi"] & B
			\end{tikzcd}
		\end{center}
		in which the morphism $f_E$ is also $T$-linear (we often abbreviate this by writing $f:E'\rightarrow E$). We thus obtain the category of fibre bundles over $T$ that we denote by $\mathbf{FormFibBun}_T$ and the full subcategory of principle $T$-bundles, that we denote by $\mathbf{FormPrinBun}_T$.
	\end{definition}
	
	In the case of principal $T$-bundles, this data can be given as follows: Let $\mathfrak U$ be a cover over which $E$ is trivialised. Then we can always refine $U$ in such a way that for all $U\in \mathfrak{U}$ the fibre bundle $E'$ is trivial over $U':=f_B^{-1}(U)$. The induced map $f_E:T\times U\rightarrow T\times U'$ is then $T$-linear and thus can be reconstructed from the induced map
	\[\theta:U'=1\times U'\hookrightarrow T\times U\xrightarrow{f_E} T\times U\xrightarrow{p_1} T.\]
	
	\begin{lemma}\label{equivalent characterisation of morphisms of principal T-bundles}
		Given a morphism $f_B:B'\rightarrow B$, and using notation as above, the data of a morphism $f=(f_E,f_B)$ of principal $T$-bundles is equivalent to the data of morphisms $\theta_i: U'_i\rightarrow T$ for all $U_i\in \mathfrak U$ such that for all $i,j$ th following diagram commutes:
		\begin{center}
			\begin{tikzcd}
				T\times U'_{ij} \arrow[d, "f_E"] \arrow[r, "\phi'_{ij}"] & T\times U'_{ij} \arrow[d, "f_E"] \\
				T\times U_{ij} \arrow[r, "\phi_{ij}"] & T\times U_{ij}.
			\end{tikzcd}
		\end{center}
		Moreover, commutativity of the above diagram is equivalent to commutativity of 
		\begin{center}
			\begin{tikzcd}
				U'_{ij} \arrow[r, "\psi'_{ij}\times\theta_j"] \arrow[d, "(\psi_{ij}\circ f)\times\theta_i"'] & T\times T \arrow[d] \arrow[d, "m"] \\
				T\times T \arrow[r] \arrow[r, "m"] & T.
			\end{tikzcd}
		\end{center}
	\end{lemma}
	Or in short hand notation,
	\[\psi'_{ij}(u)\theta_j(u)=\psi_{ij}(f(u))\cdot \theta_i(u)\]
	\begin{proof}
		One direction is clear. For the other, the first part follows from glueing. The second part is a consequence of all maps in the first diagram being $T$-linear.
	\end{proof}
	
	\begin{definition}\label{definition of Borel construction}
		Let $\pi:E\rightarrow B$ be a principial $T$-bundle. Let $F$ be a formal scheme with an action by $T$. Since the data in the equivalent characterisation of Lemma~\ref{equivalent characterisation of principal $T$-bundle} is completely independent of the fibre, the morphisms $\psi_{ij}:U_{ij}\rightarrow T$ by Lemma~\ref{equivalent characterisation of principal $T$-bundle} define a fibre bundle with fibre $F$ and structure group $T$ that we denote by $F\times^T E$. This is called the \textbf{associated bundle} or Borel-Weil construction.
	\end{definition}
	
	Note that in many authors in differential geometry and topology denote the associated bundle by "$F\times^T E$" instead of $F\times^T E$. In our setting, however, this is slightly confusing since we often have natural maps from $T$ to $F$ and $E$, but $F\times^T E$ is usually \textit{not} their fibre product. In fact it behaves more like a pushout, for instance in the case that $E$ comes from a short exact sequence.
	
	\begin{proposition}\label{associated bundle construction is bifunctor}
		The associated bundle construction is a bifunctor \[-\times^T-:\mathbf{FormAct}_T\times \mathbf{FormPrinBun}_T\rightarrow \mathbf{FormFibBun}_T\] 
		from the categories of formal schemes with $T$-action $\times$ the category of principal $T$-bundles to the category of fibre bundles with structure group $T$.
	\end{proposition}
	\begin{proof}
		Let $E$ and $E'$ be principal $T$-bundles and let $f:E'\rightarrow E$ be a morphism of $T$-bundles. Let $F$ and $F'$ be formal schemes with $T$-action and let $\gamma:F'\rightarrow F$ be a $T$-equivariant morphism. Then we can find compatible covers $\mathfrak U'$ of $E'$ and $\mathfrak U$ of $E$ such that locally we have diagrams like in Lemma~\ref{equivalent characterisation of morphisms of principal T-bundles}. Then locally $F\times^T E$ and $F'\times^T E'$ are of the form $F\times U_i$ and $F'\times U'_i$ such that we obtain a natural map
		\[F'\times U'_i\xrightarrow{(\lambda\times^T\pi)} F\times U_i, \quad (f,u)\mapsto (\lambda(f)\theta_i(u),\pi(u))\]
		(of course this description is just a short hand for a diagram of maps, and not a description in terms of "points"). These maps glue together over the cover, since on intersection Lemma~\ref{equivalent characterisation of morphisms of principal T-bundles} implies that we have a commutative diagram
		\begin{center}
			\begin{tikzcd}
				F'\times U'_{ij} \arrow[r, "\lambda\times^T\pi"] & F\times U_{ij} \\
				F'\times U'_{ij} \arrow[u, "\psi'_{ij}\times \operatorname{id}"] \arrow[r, "\lambda\times^T\pi"] & F\times U_{ij} \arrow[u, "\psi_{ij}\times \operatorname{id}"'].
			\end{tikzcd}
		\end{center}
		One easily checks that this is functorial in both components.
	\end{proof}
	
	\begin{lemma}
		Let $S$ be another formal group scheme that receives an action of $T$ from a group homomorphism $g:T\rightarrow S$. Then for any principal $T$-bundle $E$, the Borel construction $S\times^T E$ is a principal $S$-bundle.
	\end{lemma}
	\begin{proof}
		This follows from Lemma~\ref{equivalent characterisation of principal $T$-bundle}. The only thing we need to check is that the cocycle condition from diagram (\ref{cocycle condition of fibre bundle}) also holds with respect to $S$. But $g$ is a homomorphism and therefore the following diagram commutes:
		\begin{center}
			\begin{tikzcd}
				T\times T \arrow[d, "m"] \arrow[r, "g\times g"] & S\times S \arrow[d, "m"] \\
				T \arrow[r, "g"] & S.
			\end{tikzcd}
		\end{center}
		
		
	\end{proof}
	
	
	\begin{lemma}\label{change of fibre is functorial}
		The Borel construction is a functor $S\times^T -$ from principal $T$-bundles to principal $S$-bundles.
	\end{lemma}
	\begin{proof}
		This is a consequence of Lemma~\ref{equivalent characterisation of morphisms of principal T-bundles}. One obtains the necessary data by composing the morphisms $\theta':U_i'\rightarrow T$ with the morphism $T\rightarrow S$. These morphisms glue together because the second diagram of Lemma~\ref{equivalent characterisation of morphisms of principal T-bundles} commutes, as one easily sees from the fact that $T\rightarrow S$ is a morphism of formal groups. 
	\end{proof}
	
	The Borel construction satisfies the following universal property:
	\begin{lemma}\label{universal property of associated fibre construction for principal bundles}
		Let $g:T\rightarrow S$ be a homomorphism of formal group schemes and let $E\rightarrow B$ be a principal $T$-bundle. Let $X$ be any principle $S$-bundle. Note that $X$ receives a $T$-action from $g$. Then there is a functorial one-to-one correspondence between $T$-linear morphisms $E\rightarrow X$ and morphisms of principal $S$-bundles $S\times^T E\rightarrow X$.
	\end{lemma}
	
	\subsection{The semi-linear case}
	We later want to consider morphisms of fibre bundles that are induces from morphisms of short exact sequences. In this situation. in order to describe the morphism of the kernels, we need to incorporate morphisms of the structure group into the notion of morphisms of fibre bundles. For this we need semi-linear group actions.
	\begin{definition}
		Let $T$ and $S$ be formal group schemes and let $g:T\rightarrow S$ be a homomorphism. Let $X$ and $Y$ be formal schemes with actions $m:T\times X\rightarrow X$ and $m:S\times Y\rightarrow Y$ respectively. Then by a $g$-linear morphism $f:X\rightarrow Y$ we mean a morphism of formal schemes such that the following diagram commutes
		\begin{center}
			\begin{tikzcd}
				T\times X \arrow[r, "g\times f"] \arrow[d, "m"'] & S\times Y \arrow[d, "m"] \\
				X \arrow[r, "f"] & Y.
			\end{tikzcd}
		\end{center}
	\end{definition}
	
	\begin{definition}
		We denote by $\mathbf{FormAct}$ the category of pairs $(T,X)$ where $T$ is a formal group scheme and $X$ is a formal scheme with $T$ action, and morphisms are pairs of $(g,f)$ where $g$ is a group homomorphism and $f$ is a $g$-linear morphism. It has a natural forgetful functor to $
		\mathbf{FormGrp}$, the category of formal group schemes.
	\end{definition}
	
	\begin{definition}
		Let $g:T'\rightarrow T$ be a homomorphisms of formal group schemes. Let $\pi:E\rightarrow B$ be a fibre bundle with fibre $F$ and structure group $T$ and let $\pi':E'\rightarrow B'$ be a fibre bundle with fibre $F'$ and structure group $T'$ . Then a $g$-linear morphism of principal bundles is a diagram
		\begin{center}
			\begin{tikzcd}
				E' \arrow[d] \arrow[d, "f_E"] \arrow[r, "\pi'"] & B' \arrow[d, "f_B"] \\
				E \arrow[r, "\pi"] & B
			\end{tikzcd}
		\end{center}		
		such that $f_E$ is $g$-linear. We denote by $\mathbf{ FormPrinBun}$ the category of fibre bundles $(E,B,\pi,T,F)$ with arrows being the morphisms of principal bundles. It has a natural forgetful functor $(E,B,\pi,T,F) \mapsto T$ to the category $\mathbf{FormGrp}$ of formal group schemes
	\end{definition}
	We get the natural analogue of Lemma~\ref{equivalent characterisation of morphisms of principal T-bundles}:
	
	\begin{lemma}
		With the notations from Lemma~\ref{equivalent characterisation of morphisms of principal T-bundles}, a $g$-linear morphism of a principal $T'$-bundle to a principal $T$-bundle is equivalent to the data of morphisms $\theta: U_i'\rightarrow T$ such that the following diagram commutes on intersections:
		\begin{center}
			\begin{tikzcd}
				U'_{ij} \arrow[r, "\psi'_{ij}\times\theta_j"] \arrow[d, "(\psi_{ij}\circ f) \times \theta_i"'] & T'\times T \arrow[r, "g\times \operatorname{id}"] & T\times T  \arrow[d, "m"] \\
				T\times T \arrow[rr, "m"] &  & T
			\end{tikzcd}
		\end{center}
	\end{lemma}
	Or in short hand notation,
	\begin{equation}\label{shorthand for description of semi-linear morphism of fibre  bundles}
	g(\psi'_{ij}(u))\cdot\theta_j(u)=\psi_{ij}(f(u))\cdot \theta_i(u).
	\end{equation}
	
	
	Similarly as in Proposition~\ref{associated bundle construction is bifunctor} one can conclude from this that change of fibre is functorial in the following sense:
	
	\begin{proposition}\label{associated bundle construction in the semi-linear case is a sort of fibered bifunctor}
		
		Given any homomorphism of group schemes $g:T'\rightarrow T$ and a $g$-linear homomorphism $h:F'\rightarrow F$ of formal schemes with $T'$ and $T$-actions respectively, and a homomorphism $f:E'\rightarrow E$ of principal $T'$ and $T$-bundles over $g$, one obtains a morphism
		\[h\times^g f : F'\times^{T'}E'\rightarrow F\times^T E\]
		of fibre bundles over $g$, in a way that is functorial in $h,g,f$. 
		More precisely, the associated bundle construction is a fibered bifunctor
		\[-\times^{-}-: \mathbf{FormAct} \times_{\mathbf{FormGrp}} \mathbf{FormPrinBun}\rightarrow \mathbf{FormBun}. \]
	\end{proposition}
	\begin{proof}
		Let ($E$,$B$,$\pi$,$T$) and ($E'$,$B'$,$\pi'$,$T'$) be principal bundles. Let $F$ and $F'$ be formal schemes with $T$-action and $T'$ action respectively. Let $g:T\rightarrow T'$ be a group homomorphism and let $h:F'\rightarrow F$ be a $g$-equivariant morphism.
		Let $f:E'\rightarrow E$ be a morphism of principle fibre bundles over $g$.
		Then we can find compatible covers $\mathfrak U'$ of $E'$ and $\mathfrak U$ of $E$ such that locally we have diagrams like in Lemma~\ref{equivalent characterisation of morphisms of principal T-bundles}. Then locally $F\times^T E$ and $F'\times^T E'$ are of the form $F\times U_i$ and $F'\times U'_i$ such that we obtain a natural map
		\[F'\times U'_i\xrightarrow{(h\times^T\pi)} F\times U_i, \quad (x,u)\mapsto (h(x)\theta_i(u),f_\pi(u))\]
		(as before this description is just a short hand for a diagram of maps, and not a description in terms of "points"). These maps glue together over the cover, since on intersection Lemma~\ref{equivalent characterisation of morphisms of principal T-bundles} implies that we have a commutative diagram
		\begin{center}
			\begin{tikzcd}
				F'\times U'_{ij} \arrow[r, "h\times^T\pi"] & F\times U_{ij} \\
				F'\times U'_{ij} \arrow[u, "\psi'_{ij}\times \operatorname{id}"] \arrow[r, "h\times^T\pi"] & F\times U_{ij} \arrow[u, "\psi_{ij}\times \operatorname{id}"'].
			\end{tikzcd}
		\end{center}
		More precisely, by $g$-linearity of $h$ one has
		\[h(x\cdot\psi_{ij}'(u))\cdot \theta_j(u)  =  h(x)\cdot g(\psi_{ij}'(u))\cdot \theta_j(u)  \stackrel{(\ref{shorthand for description of semi-linear morphism of fibre  bundles})}{=} h(x)\cdot \psi_{ij}(f(u))\cdot \theta_i(u).\]
		This shows that the maps glue to a morphism $h\times^g f$ as desired.
		
		By refining covers, one easily checks that this is functorial in both components.
	\end{proof}
	We obtain a variant of Lemma~\ref{universal property of associated fibre construction for principal bundles} in the semilinear case:
	
	\begin{lemma}\label{universal property of associated fibre construction in the semilinear case}
		Let $E'$ be a principal $T'$ bundle and let $E$ be a principal $T$-bundle. Let $H'$ and $H$ be formal group schemes and assume there is a commutative diagram of group homomorphisms
		\begin{center}
			\begin{tikzcd}
					H' \arrow[r, "h"] & H \\
					T' \arrow[r, "g"] \arrow[u] & T \arrow[u].
			\end{tikzcd}
		\end{center}
		Let moreover $f:E'\rightarrow E$ be a $g$-linear morphism of fibre bundles.
		Then the map $h\times^g f$ from Proposition~\ref{associated bundle construction in the semi-linear case is a sort of fibered bifunctor} is the unique $h$-linear morphism of fibre bundles making the following diagram commute:
		\begin{center}
			\begin{tikzcd}
				H'\times^{T'}E' \arrow[r, "h\times^{f}g"] & H\times^{T}E \\
				E' \arrow[r, "f"] \arrow[u] & E \arrow[u].
			\end{tikzcd}
		\end{center}
	\end{lemma}
	\begin{proof}
		The morphism exists by Proposition~\ref{associated bundle construction in the semi-linear case is a sort of fibered bifunctor}. The vertical maps in the commutative diagram exist by functoriality via $E=T\times^{T}E\rightarrow H\times^{T}E$. 
		
		On the other hand, on any compatible trivialisation $T'\times U'\rightarrow T\times U$ of $f:E'\rightarrow E$ there is clearly only one way to extend this to $H'\times U'\rightarrow H\times U$ in a $h$-linear way.
	\end{proof}
	
	\begin{remark}\label{appendix in the case of rigid spaces and schemes}
	All that we have done in this chapter can be done in completely the same way with formal schemes replaced by rigid spaces (covers being replaced by admissible covers) and also for schemes, or in fact for any site. We have preferred to use formal schemes to make things more explicit. 
	The different categories of fibre bundles are well-behaved with respect to the usual functors between the different categories: For instance, by functoriality of fibre products there are natural rigidification and reduction functors from formal principal $T$-bundles over $\mathcal O_K$ to rigid principal $T_\eta$-bundles over $K$ on the generic fibre, and to principal $\overline{T}$-bundles on the reduction $\mathcal O_K/p$. Moreover, these generic fibre and reduction functors commute with the associated fibre construction:
	\end{remark}
	\begin{lemma}\label{associated bundle commutes with generic fibre}
		Let $T$ be a formal group scheme and let $\pi:E\rightarrow B$ be a principial $T$-bundle. Let $F$ be a formal scheme with an action by $T$. Then
		\[(E\times^T B)_\eta = E_\eta\times^{T_\eta} B_\eta \]
	\end{lemma}
	\begin{proof}
		This can be checked locally on any trivialising cover, where it is clear.
	\end{proof}
	
	
	\begin{thebibliography}{9}
		
		\bibitem{Bosch lectures} 
		Siegfried Bosch
		\textit{Lectures in formal and rigid geometry}.
		
		\bibitem{Bosch defines formal rigid spaces} 
		Siegfried Bosch
		\textit{Zur Kohomologietheorie rigid analytischer R\"aume }.
		
		\bibitem{BL} 
		Bosch, L\"utkebohmert
		\textit{Degenerating abelian varieties}.
		
		\bibitem{FvdP}
		Jean Fresnel, Marius van der Put
		\textit{Rigid Analytic Geometry and its Applications}
		
		
		\bibitem{rigid geometry of curves} 
		Werner L\"utkebohmert
		\textit{Rigid Geometry of Curves and Their Jacobians}. 
		
		\bibitem{torsion} 
		James S. Milne
		\textit{Abelian varieties}.
		In: Arithmetic Geometry, Gary Cornell and Joseph H Silverman
			
		\bibitem{perfectoid} 
		Peter Scholze
		\textit{Perfectoid spaces}. 
		
		\bibitem{perfectoid IHES} 
		Peter Scholze
		\textit{Perfectoid spaces}. The other one, in publications de l'IHES 
		
		\bibitem{p-adic Hodge} 
		Peter Scholze
		\textit{p-adic Hodge theory for rigid analytic varieties}.
		
		\bibitem{torsion} 
		Peter Scholze
		\textit{Torsion in cohomology of locally symmetric varieties}.
		
		\bibitem{SW} 
		Peter Scholze, Jared Weinstein
		\textit{moduli spaces of p-divisible groups}.
		
		
		
	\end{thebibliography}
	
	
	
	
	
	
	
	
	
	
	
	
	
	
\end{document}
